
\let\negmedspace\undefined
\let\negthickspace\undefined
\documentclass[journal]{IEEEtran}
\usepackage[a5paper, margin=10mm, onecolumn]{geometry}
\usepackage{lmodern} 
\usepackage{tfrupee} 

\setlength{\headheight}{1cm} % Set the height of the header box
\setlength{\headsep}{0mm}     % Set the distance between the header box and the top of the text

\usepackage{gvv-book}
\usepackage{gvv}
\usepackage{cite}
\usepackage{amsmath,amssymb,amsfonts,amsthm}
\usepackage{algorithmic}
\usepackage{graphicx}
\usepackage{textcomp}
\usepackage{xcolor}
\usepackage{txfonts}
\usepackage{listings}
\usepackage{enumitem}
\usepackage{mathtools}
\usepackage{gensymb}
\usepackage{comment}
\usepackage[breaklinks=true]{hyperref}
\usepackage{tkz-euclide} 
\usepackage{listings}                                      
\def\inputGnumericTable{}                                 
\usepackage[latin1]{inputenc}                                
\usepackage{color}                                            
\usepackage{array}                                            
\usepackage{longtable}
\usepackage{multicol}
\usepackage{calc}                                             
\usepackage{multirow}                                         
\usepackage{hhline}                                           
\usepackage{ifthen}                                           
\usepackage{lscape}
\begin{document}

\bibliographystyle{IEEEtran}
\vspace{3cm}

\title{5.4.26}
\author {EE25BTECH11031 - Sai Sreevallabh}
% \maketitle
% \newpage
% \bigskip
{\let\newpage\relax\maketitle}

\renewcommand{\thefigure}{\theenumi}
\renewcommand{\thetable}{\theenumi}
\setlength{\intextsep}{10pt} % Space between text and floats


\numberwithin{equation}{enumi}
\numberwithin{figure}{enumi}
\renewcommand{\thetable}{\theenumi}

\textbf{Question: }\\

 Using elementary transformations find the inverse of the given matrix\\
 \begin{center}
     $\myvec{1&-3&2\\
     -3&0&-5\\
     2&5&0}$
 \end{center}

\textbf{Solution: }\\ 

Given 
\begin{align}
    \vec{A} = 
     \myvec{1&-3&2\\
     -3&0&-5\\
     2&5&0}
\end{align}

To find the inverse, $\vec{A}^{-1}$, we augment the identity matrix $\vec{I}$ to $\vec{A}$ and apply row operations to this augmented matrix. 

\begin{align}
    \augvec{3}{3}{1&-3&2&1&0&0\\
                -3&0&-5&0&1&0\\
                2&5&0&0&0&1}
    \xleftrightarrow[]{R_2\xrightarrow{}R_2+3R_1}
    \augvec{3}{3}{1&-3&2&1&0&0\\
                    0&9&-11&3&1&0\\
                    2&5&0&0&0&1}
\end{align}

\begin{align}
    \augvec{3}{3}{1&-3&2&1&0&0\\
                    0&9&-11&3&1&0\\
                    2&5&0&0&0&1}
    \xleftrightarrow[]{R_3\xrightarrow[]{}{R_3-2R_1}}
    \augvec{3}{3}{1&-3&2&1&0&0\\
                    0&9&-11&3&1&0\\
                    0&-1&4&-2&0&1}
\end{align}

\begin{align}
    \augvec{3}{3}{1&-3&2&1&0&0\\
                    0&9&-11&3&1&0\\
                    0&-1&4&-2&0&1}
    \xleftrightarrow[]{R_2\xrightarrow[]{}\frac{1}{9}R_2}
    \augvec{3}{3}{1&-3&2&1&0&0\\
                    0&1&\frac{-11}{9}&\frac{1}{3}&
                    \frac{1}{9}&0\\
                    0&-1&4&-2&0&1}
\end{align}

\begin{align}
    \augvec{3}{3}{1&-3&2&1&0&0\\
                    0&1&\frac{-11}{9}&\frac{1}{3}&
                    \frac{1}{9}&0\\
                    0&-1&4&-2&0&1}
    \xleftarrow[]{R_1\xrightarrow{}{R_1-3R_2}}
    \augvec{3}{3}{1&0&\frac{5}{3}&0&\frac{-1}{3}&0\\
                    0&1&\frac{-11}{9}&\frac{1}{3}&
                    \frac{1}{9}&0\\
                    0&-1&4&-2&0&1}
\end{align}

\begin{align}
    \augvec{3}{3}{1&0&\frac{5}{3}&0&\frac{-1}{3}&0\\[1ex]
                    0&1&\frac{-11}{9}&\frac{1}{3}&
                    \frac{1}{9}&0\\[1ex]
                    0&-1&4&-2&0&1}
    \xleftarrow[]{R_3\xrightarrow{}{R_3+R_2}}
    \augvec{3}{3}{1&0&\frac{5}{3}&0&\frac{-1}{3}&0\\[1ex]
                    0&1&\frac{-11}{9}&\frac{1}{3}&
                    \frac{1}{9}&0\\[1ex]
                    0&0&\frac{25}{9}&\frac{-5}{3}&\frac{1}{9}&1}
\end{align}

\begin{align}
    \augvec{3}{3}{1&0&\frac{5}{3}&0&\frac{-1}{3}&0\\[1ex]
                    0&1&\frac{-11}{9}&\frac{1}{3}&
                    \frac{1}{9}&0\\[1ex]
                    0&0&\frac{25}{9}&\frac{-5}{3}&\frac{1}{9}&1}
    \xleftarrow[]{R_3\xrightarrow{}{\frac{9}{25}R_3}}
    \augvec{3}{3}{1&0&\frac{5}{3}&0&\frac{-1}{3}&0\\[1ex]
                    0&1&\frac{-11}{9}&\frac{1}{3}&
                    \frac{1}{9}&0\\[1ex]
                    0&0&1&\frac{-3}{5}&\frac{1}{25}&\frac{9}{25}}
\end{align}

\begin{align}
    \augvec{3}{3}{1&0&\frac{5}{3}&0&\frac{-1}{3}&0\\[1ex]
                    0&1&\frac{-11}{9}&\frac{1}{3}&
                    \frac{1}{9}&0\\[1ex]
                    0&0&1&\frac{-3}{5}&\frac{1}{25}&\frac{9}{25}}
    \xleftarrow[]{R_2\xrightarrow{}{R_2+\frac{11}{9}R_3}}
    \augvec{3}{3}{1&0&\frac{5}{3}&0&\frac{-1}{3}&0\\[1ex]
                    0&1&0&\frac{-2}{5}&
                    \frac{4}{25}&\frac{11}{25}\\[1ex]
                    0&0&1&\frac{-3}{5}&\frac{1}{25}&\frac{9}{25}}
\end{align}

\begin{align}
    \augvec{3}{3}{1&0&\frac{5}{3}&0&\frac{-1}{3}&0\\[1ex]
                    0&1&0&\frac{-2}{5}&
                    \frac{4}{25}&\frac{11}{25}\\[1ex]
                    0&0&1&\frac{-3}{5}&\frac{1}{25}&\frac{9}{25}}
    \xleftarrow[]{R_1\xrightarrow{}{R_1+\frac{5}{3}R_3}}
    \augvec{3}{3}{1&0&0&1&\frac{-2}{5}&\frac{-3}{5}\\[1ex]
                    0&1&0&\frac{-2}{5}&
                    \frac{4}{25}&\frac{11}{25}\\[1ex]
                    0&0&1&\frac{-3}{5}&\frac{1}{25}&\frac{9}{25}}
\end{align}\\

Therefore, 
\begin{center}
    $\vec{A}^{-1} = \myvec{1&\frac{-2}{5}&\frac{-3}{5}\\[1ex]
                    \frac{-2}{5}&
                    \frac{4}{25}&\frac{11}{25}\\[1ex]
                    \frac{-3}{5}&\frac{1}{25}&\frac{9}{25}}$
\end{center}

\end{document}
