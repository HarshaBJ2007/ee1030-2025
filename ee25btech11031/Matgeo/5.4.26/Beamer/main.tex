\documentclass{beamer}
\usepackage[utf8]{inputenc}

\usetheme{Madrid}
\usecolortheme{default}
\usepackage{txfonts}
\usepackage{listings}
\usepackage{adjustbox}
\usepackage{tabularx}
\usepackage{lmodern}
\usepackage{circuitikz}
\usepackage{tikz}

\usepackage{gvv}
\usepackage{cite}
\usepackage{amsmath,amssymb,amsfonts,amsthm}
\usepackage{algorithmic}
\usepackage{graphicx}
\usepackage{textcomp}
\usepackage{xcolor}
\usepackage{txfonts}
\usepackage{listings}
\usepackage{enumitem}
\usepackage{mathtools}
\usepackage{gensymb}
\usepackage{comment}
\usepackage{tkz-euclide} 
\usepackage{listings}                                      
\def\inputGnumericTable{}                                
\usepackage{color}                                            
\usepackage{array}                                            
\usepackage{longtable}
\usepackage{multicol}
\usepackage{calc}                                             
\usepackage{multirow}                                         
\usepackage{hhline}                                           
\usepackage{ifthen}

\setbeamertemplate{page number in head/foot}[totalframenumber]

\usepackage{tcolorbox}
\tcbuselibrary{minted,breakable,xparse,skins}



\definecolor{bg}{gray}{0.95}
\DeclareTCBListing{mintedbox}{O{}m!O{}}{%
  breakable=true,
  listing engine=minted,
  listing only,
  minted language=#2,
  minted style=default,
  minted options={%
    linenos,
    gobble=0,
    breaklines=true,
    breakafter=,,
    fontsize=\small,
    numbersep=8pt,
    #1},
  boxsep=0pt,
  left skip=0pt,
  right skip=0pt,
  left=25pt,
  right=0pt,
  top=3pt,
  bottom=3pt,
  arc=5pt,
  leftrule=0pt,
  rightrule=0pt,
  bottomrule=2pt,
  toprule=2pt,
  colback=bg,
  colframe=orange!70,
  enhanced,
  overlay={
    \begin{tcbclipinterior}
    \fill[orange!20!white] (frame.south west) rectangle ([xshift=20pt]frame.north west);
    \end{tcbclipinterior}},
  #3,
}
\lstset{
    language=C,
    basicstyle=\ttfamily\small,
    keywordstyle=\color{blue},
    stringstyle=\color{orange},
    commentstyle=\color{green!60!black},
    numbers=left,
    numberstyle=\tiny\color{gray},
    breaklines=true,
    showstringspaces=false,
}

\title 
{5.4.26}
\date{October 11, 2025}


\author 
{Sai Sreevallabh - EE25BTECH11031}



\begin{document}


\frame{\titlepage}
\begin{frame}{Question}
 Using elementary transformations find the inverse of the given matrix\\
    \begin{center}
     $\myvec{1&-3&2\\
     -3&0&-5\\
     2&5&0}$
    \end{center}
\end{frame}


\begin{frame}{Theoretical Solution}
Given 
\begin{align}
    \vec{A} = 
     \myvec{1&-3&2\\
     -3&0&-5\\
     2&5&0}
\end{align}

To find the inverse, $\vec{A}^{-1}$, we augment the identity matrix $\vec{I}$ to $\vec{A}$ and apply row operations to this augmented matrix. 

\end{frame}

\begin{frame}{Theoretical Solution}

\begin{align}
    \augvec{3}{3}{1&-3&2&1&0&0\\
                -3&0&-5&0&1&0\\
                2&5&0&0&0&1}
    \xleftrightarrow[]{R_2\xrightarrow{}R_2+3R_1}
    \augvec{3}{3}{1&-3&2&1&0&0\\
                    0&9&-11&3&1&0\\
                    2&5&0&0&0&1}
\end{align}

\begin{align}
    \augvec{3}{3}{1&-3&2&1&0&0\\
                    0&9&-11&3&1&0\\
                    2&5&0&0&0&1}
    \xleftrightarrow[]{R_3\xrightarrow[]{}{R_3-2R_1}}
    \augvec{3}{3}{1&-3&2&1&0&0\\
                    0&9&-11&3&1&0\\
                    0&-1&4&-2&0&1}
\end{align}
\end{frame}

\begin{frame}{Theoretical Solution}
\begin{align}
    \augvec{3}{3}{1&-3&2&1&0&0\\
                    0&9&-11&3&1&0\\
                    0&-1&4&-2&0&1}
    \xleftrightarrow[]{R_2\xrightarrow[]{}\frac{1}{9}R_2}
    \augvec{3}{3}{1&-3&2&1&0&0\\
                    0&1&\frac{-11}{9}&\frac{1}{3}&
                    \frac{1}{9}&0\\
                    0&-1&4&-2&0&1}
\end{align}

\begin{align}
    \augvec{3}{3}{1&-3&2&1&0&0\\
                    0&1&\frac{-11}{9}&\frac{1}{3}&
                    \frac{1}{9}&0\\
                    0&-1&4&-2&0&1}
    \xleftarrow[]{R_1\xrightarrow{}{R_1-3R_2}}
    \augvec{3}{3}{1&0&\frac{5}{3}&0&\frac{-1}{3}&0\\
                    0&1&\frac{-11}{9}&\frac{1}{3}&
                    \frac{1}{9}&0\\
                    0&-1&4&-2&0&1}
\end{align}

\end{frame}

\begin{frame}{Theoretical Solution}
    \begin{align}
    \augvec{3}{3}{1&0&\frac{5}{3}&0&\frac{-1}{3}&0\\[1ex]
                    0&1&\frac{-11}{9}&\frac{1}{3}&
                    \frac{1}{9}&0\\[1ex]
                    0&-1&4&-2&0&1}
    \xleftarrow[]{R_3\xrightarrow{}{R_3+R_2}}
    \augvec{3}{3}{1&0&\frac{5}{3}&0&\frac{-1}{3}&0\\[1ex]
                    0&1&\frac{-11}{9}&\frac{1}{3}&
                    \frac{1}{9}&0\\[1ex]
                    0&0&\frac{25}{9}&\frac{-5}{3}&\frac{1}{9}&1}
\end{align}

\begin{align}
    \augvec{3}{3}{1&0&\frac{5}{3}&0&\frac{-1}{3}&0\\[1ex]
                    0&1&\frac{-11}{9}&\frac{1}{3}&
                    \frac{1}{9}&0\\[1ex]
                    0&0&\frac{25}{9}&\frac{-5}{3}&\frac{1}{9}&1}
    \xleftarrow[]{R_3\xrightarrow{}{\frac{9}{25}R_3}}
    \augvec{3}{3}{1&0&\frac{5}{3}&0&\frac{-1}{3}&0\\[1ex]
                    0&1&\frac{-11}{9}&\frac{1}{3}&
                    \frac{1}{9}&0\\[1ex]
                    0&0&1&\frac{-3}{5}&\frac{1}{25}&\frac{9}{25}}
\end{align}
\end{frame}

\begin{frame}{Theoretical Solution}
    \begin{align}
    \augvec{3}{3}{1&0&\frac{5}{3}&0&\frac{-1}{3}&0\\[1ex]
                    0&1&\frac{-11}{9}&\frac{1}{3}&
                    \frac{1}{9}&0\\[1ex]
                    0&0&1&\frac{-3}{5}&\frac{1}{25}&\frac{9}{25}}
    \xleftarrow[]{R_2\xrightarrow{}{R_2+\frac{11}{9}R_3}}
    \augvec{3}{3}{1&0&\frac{5}{3}&0&\frac{-1}{3}&0\\[1ex]
                    0&1&0&\frac{-2}{5}&
                    \frac{4}{25}&\frac{11}{25}\\[1ex]
                    0&0&1&\frac{-3}{5}&\frac{1}{25}&\frac{9}{25}}
\end{align}

\begin{align}
    \augvec{3}{3}{1&0&\frac{5}{3}&0&\frac{-1}{3}&0\\[1ex]
                    0&1&0&\frac{-2}{5}&
                    \frac{4}{25}&\frac{11}{25}\\[1ex]
                    0&0&1&\frac{-3}{5}&\frac{1}{25}&\frac{9}{25}}
    \xleftarrow[]{R_1\xrightarrow{}{R_1+\frac{5}{3}R_3}}
    \augvec{3}{3}{1&0&0&1&\frac{-2}{5}&\frac{-3}{5}\\[1ex]
                    0&1&0&\frac{-2}{5}&
                    \frac{4}{25}&\frac{11}{25}\\[1ex]
                    0&0&1&\frac{-3}{5}&\frac{1}{25}&\frac{9}{25}}
\end{align}\\
\end{frame}

\begin{frame}{Theoretical Solution}
Therefore, 
\begin{center}
    $\vec{A}^{-1} = \myvec{1&\frac{-2}{5}&\frac{-3}{5}\\[1ex]
                    \frac{-2}{5}&
                    \frac{4}{25}&\frac{11}{25}\\[1ex]
                    \frac{-3}{5}&\frac{1}{25}&\frac{9}{25}}$
\end{center}
\end{frame}


\begin{frame}[fragile]
    \frametitle{C Code - Finding Inverse}

    \begin{lstlisting}
#include <stdio.h>

void inverse(double (*matrix)[3]) {
    double I[3][3] = {
        {1,0,0},
        {0,1,0},
        {0,0,1}
    };
   
    double pivot = matrix[0][0];
    for(int i = 0; i < 3; i++) {
        matrix[0][i] /= pivot;
        I[0][i] /= pivot;
    }


    \end{lstlisting}

\end{frame}

\begin{frame}[fragile]
    \frametitle{C Code - Finding Inverse}

    \begin{lstlisting}
    
    double factor = matrix[1][0];
    for(int i = 0; i < 3; i++) {
        matrix[1][i] -= factor * matrix[0][i];
        I[1][i] -= factor * I[0][i];
    }
   
    factor = matrix[2][0];
    for(int i = 0; i < 3; i++) {
        matrix[2][i] -= factor * matrix[0][i];
        I[2][i] -= factor * I[0][i];
    }

     pivot = matrix[1][1];
    for(int i = 0; i < 3; i++) {
        matrix[1][i] /= pivot;
        I[1][i] /= pivot;
    }

    \end{lstlisting}

\end{frame}


\begin{frame}[fragile]
    \frametitle{C Code - Finding Inverse}

    \begin{lstlisting}
    factor = matrix[0][1];
    for(int i = 0; i < 3; i++) {
        matrix[0][i] -= factor * matrix[1][i];
        I[0][i] -= factor * I[1][i];
    }

    factor = matrix[2][1];
    for(int i = 0; i < 3; i++) {
        matrix[2][i] -= factor * matrix[1][i];
        I[2][i] -= factor * I[1][i];
    }
   

    pivot = matrix[2][2];
    for(int i = 0; i < 3; i++) {
        matrix[2][i] /= pivot;
        I[2][i] /= pivot;
    }


    \end{lstlisting}

\end{frame}

\begin{frame}[fragile]
    \frametitle{C Code - Finding Inverse}

    \begin{lstlisting}
    factor = matrix[0][2];
    for(int i = 0; i < 3; i++) {
        matrix[0][i] -= factor * matrix[2][i];
        I[0][i] -= factor * I[2][i];
    }

    factor = matrix[1][2];
    for(int i = 0; i < 3; i++) {
        matrix[1][i] -= factor * matrix[2][i];
        I[1][i] -= factor * I[2][i];
    }

    for (int i = 0; i < 3; i++) {
    for (int j = 0; j < 3; j++) {
        matrix[i][j] = I[i][j];
    }
}

}
    \end{lstlisting}
    
\end{frame}


\begin{frame}[fragile]
    \frametitle{Python Code - Using Shared Object}
    \begin{lstlisting}
import numpy as np
import ctypes

c_lib = ctypes.CDLL("./code.so")

c_lib.inverse.argtypes = [ctypes.POINTER((ctypes.c_double * 3))]

A = np.array([
    [1.0, -3.0, 2.0],
    [-3.0, 0.0, -5.0],
    [2.0, 5.0, 0.0]
], dtype=np.float64)


\end{lstlisting}
\end{frame}

\begin{frame}[fragile]
    \frametitle{Python Code - Using Shared Object}
    \begin{lstlisting}

B = A.ctypes.data_as(ctypes.POINTER((ctypes.c_double * 3)))

c_lib.inverse(B)

np.set_printoptions(precision=2)

print(A)


\end{lstlisting}
\end{frame}

%-------End of Python+C-------------


\begin{frame}[fragile]
    \frametitle{Python Code}
    \begin{lstlisting}
import numpy as np
import numpy.linalg as LA

A = np.array([
    [1.0, -3.0, 2.0],
    [-3.0, 0.0, -5.0],
    [2.0, 5.0, 0.0]
])

A_inv = LA.inv(A)

print(A_inv)


\end{lstlisting}
\end{frame}


\end{document}
