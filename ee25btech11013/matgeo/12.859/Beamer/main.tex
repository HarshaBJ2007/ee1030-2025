
\documentclass{beamer}
\usepackage[utf8]{inputenc}

\usetheme{Madrid}
\usecolortheme{default}
\usepackage{amsmath,amssymb,amsfonts,amsthm}
\usepackage{mathtools}
\usepackage{txfonts}
\usepackage{tkz-euclide}
\usepackage{listings}
\usepackage{adjustbox}
\usepackage{tfrupee}
\usepackage{array}
\usepackage{gensymb}
\usepackage{tabularx}
\usepackage{gvv}
\usepackage{lmodern}
\usepackage{circuitikz}
\usepackage{tikz}
\lstset{literate={·}{{$\cdot$}}1 {λ}{{$\lambda$}}1 {→}{{$\to$}}1}
\usepackage{graphicx}

\setbeamertemplate{page number in head/foot}[totalframenumber]

\usepackage{tcolorbox}
\tcbuselibrary{minted,breakable,xparse,skins}

\definecolor{bg}{gray}{0.95}
\DeclareTCBListing{mintedbox}{O{}m!O{}}{%
  breakable=true,
  listing engine=minted,
  listing only,
  minted language=#2,
  minted style=default,
  minted options={%
    linenos,
    gobble=0,
    breaklines=true,
    breakafter=,,
    fontsize=\small,
    numbersep=8pt,
    #1},
  boxsep=0pt,
  left skip=0pt,
  right skip=0pt,
  left=25pt,
  right=0pt,
  top=3pt,
  bottom=3pt,
  arc=5pt,
  leftrule=0pt,
  rightrule=0pt,
  bottomrule=2pt,
  toprule=2pt,
  colback=bg,
  colframe=orange!70,
  enhanced,
  overlay={%
    \begin{tcbclipinterior}
    \fill[orange!20!white] (frame.south west) rectangle ([xshift=20pt]frame.north west);
    \end{tcbclipinterior}},
  #3,
}
\lstset{
    language=C,
    basicstyle=\ttfamily\small,
    keywordstyle=\color{blue},
    stringstyle=\color{orange},
    commentstyle=\color{green!60!black},
    numbers=left,
    numberstyle=\tiny\color{gray},
    breaklines=true,
    showstringspaces=false,
}

\title{12.859}
\date{October 10, 2025}
\author{Bhargav - EE25BTECH11013}

\begin{document}

\frame{\titlepage}

\begin{frame}{Question}
\textbf{Question}: \\
Let $\vec{O} = \{\vec{P} : \vec{P} \text{ is a } 3 \times 3 \text{ real matrix with } \vec{P}^T\vec{P} = \vec{I}_3, \det(\vec{P}) = 1\}$. Which of the following options is/are correct?

\begin{enumerate}
    \item[a)] There exists $\vec{P} \in \vec{O}$ with $\lambda = \frac{1}{2}$ as an eigenvalue.
    \item[b)] There exists $\vec{P} \in \vec{O}$ with $\lambda = 2$ as an eigenvalue.
    \item[c)] If $\lambda$ is the only real eigenvalue of $\vec{P} \in 
    \vec{O}$, then $\lambda = 1$.
    \item[d)] There exists $\vec{P} \in O$ with $\lambda = -1$ as an eigenvalue.
\end{enumerate}
\end{frame}

\begin{frame}{Solution}
Let $\vec{v}$ be the eigenvector corresponding to the eigenvalue $\lambda$.
\begin{align}
\vec{P}\vec{v} = \lambda\vec{v} \label{eqq}
\end{align}
Orthogonal transformations preserve the length of vectors  ($\abs{\vec{P}} = 1$)
\begin{align}
\norm{\vec{P}\vec{v}}=\norm{\vec{v}} \label{eq1}
\end{align}
This can be proved in this way:
\begin{align}
\norm{\vec{P}\vec{v}}^2 = \brak{\vec{P}\vec{v}}^\top\brak{\vec{P}\vec{v}}=\vec{v}^\top\vec{P}^\top\vec{P}\vec{v}
\end{align}
Since $\vec{P}^\top \vec{P} = \vec{I}$
\begin{align}
\norm{\vec{P}\vec{v}}^2 = \vec{v}^\top\vec{v} = \norm{\vec{v}}^2
\end{align}
\begin{align}
\implies \norm{\vec{P}\vec{v}} = \norm{\vec{v}}
\end{align}
From \eqref{eqq},
\begin{align}
\norm{\vec{P}\vec{v}} = \abs{\lambda}\norm{\vec{v}} \label{eq2}
\end{align}


\end{frame}

\begin{frame}{Solution}
Using the equations \eqref{eq1} and \eqref{eq2},
\begin{align}
\norm{\vec{P}\vec{v}} = \norm{\vec{v}} = \abs{\lambda}\norm{\vec{v}}
\end{align}
\begin{align}
\implies \norm{\vec{v}} = \abs{\lambda}\norm{\vec{v}}
\end{align}
Thus, $\abs{\lambda} = 1$

So, eigenvalue satisfies the condition that $\abs{\lambda}=1$\\


Thus, options (c) and (d) are correct. \\ \\ \\
\end{frame}

\begin{frame}{Solution}
This can be verified by examples.\\ 
1. For $\lambda_1$ = 1 \\ \\
$\vec{P} = \myvec{1 & 0 & 0 \\ 0 & 1 & 0 \\ 0 & 0 & 1}$ \\
$\vec{P}^T\vec{P} = \vec{I}$ \\
Eigenvalue of $\vec{P}$ is 1. \\ \\
2. For $\lambda_2 = -1$ \\ \\
$\vec{P} = \myvec{-1 & 0 & 0 \\ 0 & 1 & 0 \\ 0 & 0 & -1}$ \\
$\vec{P}^T\vec{P} = \vec{I}$ \\
Eigenvalue of $\vec{P}$ is -1. \\ \\
\end{frame}

\begin{frame}[fragile]
    \frametitle{C Code}
    \begin{lstlisting}
#include <stdio.h>

void matmul_transpose(double A[3][3], double result[3][3]) {
    // Compute result = A^T * A
    for (int i = 0; i < 3; i++) {
        for (int j = 0; j < 3; j++) {
            result[i][j] = 0.0;
            for (int k = 0; k < 3; k++) {
                result[i][j] += A[k][i] * A[k][j];
            }
        }
    }
}




    \end{lstlisting}
\end{frame}


\begin{frame}[fragile]
    \frametitle{Python + C Code}
    \begin{lstlisting}
import numpy as np
import ctypes


lib = ctypes.CDLL("./libcode.so")
# Define function argument and return types
lib.matmul_transpose.argtypes = [((ctypes.c_double * 3) * 3), ((ctypes.c_double * 3) * 3)]

# Convert numpy array to C 2D array
def to_c_matrix(A):
    c_mat = ((ctypes.c_double * 3) * 3)()
    for i in range(3):
        for j in range(3):
            c_mat[i][j] = A[i][j]
    return c_mat

# Convert C 2D array back to numpy
def from_c_matrix(c_mat):
    return np.array([[c_mat[i][j] for j in range(3)] for i in range(3)])






    \end{lstlisting}
\end{frame}
\begin{frame}[fragile]
    \frametitle{Python + C Code}
    \begin{lstlisting}

# Function to verify example
def verify_matrix(P, name):
    print(f"\n--- {name} ---")
    print("Matrix P:\n", P)

    P_c = to_c_matrix(P)
    result_c = ((ctypes.c_double * 3) * 3)()
    lib.matmul_transpose(P_c, result_c)

    PT_P = from_c_matrix(result_c)
    print("\nP^T * P =\n", PT_P)
    print("\nIs P orthogonal?", np.allclose(PT_P, np.eye(3)))

    eigenvalues, _ = np.linalg.eig(P)
    print("\nEigenvalues of P:", eigenvalues)


    \end{lstlisting}
\end{frame}
\begin{frame}[fragile]
    \frametitle{Python + C Code}
    \begin{lstlisting}

# Example 1: λ = 1
P1 = np.array([[1, 0, 0],
               [0, 1, 0],
               [0, 0, 1]], dtype=float)

# Example 2: λ = -1
P2 = np.array([[-1, 0, 0],
               [0, 1, 0],
               [0, 0, -1]], dtype=float)

verify_matrix(P1, "Example 1 (λ = 1)")
verify_matrix(P2, "Example 2 (λ = -1)")
    \end{lstlisting}
\end{frame}


\begin{frame}[fragile]
    \frametitle{Python Code}
    \begin{lstlisting}
import numpy as np

# Example 1: λ1 = 1
P1 = np.array([
    [1, 0, 0],
    [0, 1, 0],
    [0, 0, 1]
])

# Example 2: λ2 = -1
P2 = np.array([
    [-1, 0, 0],
    [0, 1, 0],
    [0, 0, -1]
])






    \end{lstlisting}
\end{frame}
\begin{frame}[fragile]
    \frametitle{Python Code}
    \begin{lstlisting}
# Function to verify orthogonality and eigenvalues
def verify_matrix(P, name):
    print(f"--- {name} ---")
    print("Matrix P:\n", P)

    # Check orthogonality
    PT_P = np.dot(P.T, P)
    print("\nP^T * P =\n", PT_P)

    # Check if PT_P is identity
    print("\nIs P^T * P = I ?", np.allclose(PT_P, np.eye(P.shape[0])))

    # Eigenvalues of P
    eigenvalues, _ = np.linalg.eig(P)
    print("\nEigenvalues of P:", eigenvalues)
    print()



    \end{lstlisting}
\end{frame}


\end{document}



