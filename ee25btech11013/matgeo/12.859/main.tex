\let\negmedspace\undefined
\let\negthickspace\undefined
\documentclass[journal]{IEEEtran}
\usepackage[a5paper, margin=10mm, onecolumn]{geometry}
%\usepackage{lmodern} % Ensure lmodern is loaded for pdflatex
\usepackage{tfrupee} % Include tfrupee package

\setlength{\headheight}{1cm} % Set the height of the header box
\setlength{\headsep}{0mm}     % Set the distance between the header box and the top of the text

\usepackage{gvv-book}
\usepackage{comment}
\usepackage{gvv}
\usepackage{cite}
\usepackage{amsmath,amssymb,amsfonts,amsthm}
\usepackage{algorithmic}
\usepackage{graphicx}
\usepackage{textcomp}
\usepackage{xcolor}
%\usepackage{txfonts}
\usepackage{listings}
\usepackage{enumitem}
\usepackage{mathtools}
\usepackage{gensymb}
\usepackage{comment}
\usepackage[breaklinks=true]{hyperref}
\usepackage{tkz-euclide} 
\usepackage{listings}
% \usepackage{gvv}                                        
\def\inputGnumericTable{}                                 
\usepackage[latin1]{inputenc}                                
\usepackage{color}                                            
\usepackage{array}                                            
\usepackage{longtable}                                       
\usepackage{calc}                                             
\usepackage{multirow}                                         
\usepackage{hhline}                                           
\usepackage{ifthen}                                           
\usepackage{lscape}
\usepackage{circuitikz}
\tikzstyle{block} = [rectangle, draw, fill=blue!20, 
    text width=4em, text centered, rounded corners, minimum height=3em]
\tikzstyle{sum} = [draw, fill=blue!10, circle, minimum size=1cm, node distance=1.5cm]
\tikzstyle{input} = [coordinate]
\tikzstyle{output} = [coordinate]


\begin{document}

\bibliographystyle{IEEEtran}
\vspace{3cm}

\title{12.859}
\author{EE25BTECH11013 - Bhargav}
\maketitle
    {\let\newpage\relax\maketitle}

\renewcommand{\thefigure}{\theenumi}
\renewcommand{\thetable}{\theenumi}
\setlength{\intextsep}{10pt} % Space between text and floats

\numberwithin{equation}{enumi}
\numberwithin{figure}{enumi}
\renewcommand{\thetable}{\theenumi}

\textbf{Question}: \\
Let $\vec{O} = \{\vec{P} : \vec{P} \text{ is a } 3 \times 3 \text{ real matrix with } \vec{P}^T\vec{P} = \vec{I}_3, \det(\vec{P}) = 1\}$. Which of the following options is/are correct?

\begin{enumerate}
    \item There exists $\vec{P} \in \vec{O}$ with $\lambda = \frac{1}{2}$ as an eigenvalue.
    \item There exists $\vec{P} \in \vec{O}$ with $\lambda = 2$ as an eigenvalue.
    \item If $\lambda$ is the only real eigenvalue of $\vec{P} \in 
    \vec{O}$, then $\lambda = 1$.
    \item There exists $\vec{P} \in O$ with $\lambda = -1$ as an eigenvalue.
\end{enumerate}

\solution \\
Let $\vec{v}$ be the eigenvector corresponding to the eigenvalue $\lambda$.
\begin{align}
\vec{P}\vec{v} = \lambda\vec{v} \label{eqq}
\end{align}
Orthogonal transformations preserve the length of vectors  ($\abs{\vec{P}} = 1$)
\begin{align}
\norm{\vec{P}\vec{v}}=\norm{\vec{v}} \label{eq1}
\end{align}
This can be proved in this way:
\begin{align}
\norm{\vec{P}\vec{v}}^2 = \brak{\vec{P}\vec{v}}^\top\brak{\vec{P}\vec{v}}=\vec{v}^\top\vec{P}^\top\vec{P}\vec{v}
\end{align}
Since $\vec{P}^\top \vec{P} = \vec{I}$
\begin{align}
\norm{\vec{P}\vec{v}}^2 = \vec{v}^\top\vec{v} = \norm{\vec{v}}^2
\end{align}
\begin{align}
\implies \norm{\vec{P}\vec{v}} = \norm{\vec{v}}
\end{align}
From \eqref{eqq}, 
\begin{align}
\norm{\vec{P}\vec{v}} = \abs{\lambda}\norm{\vec{v}} \label{eq2}
\end{align}

Using the equations \eqref{eq1} and \eqref{eq2},
\begin{align}
\norm{\vec{P}\vec{v}} = \norm{\vec{v}} = \abs{\lambda}\norm{\vec{v}}
\end{align}
\begin{align}
\implies \norm{\vec{v}} = \abs{\lambda}\norm{\vec{v}}
\end{align}
Thus, $\abs{\lambda} = 1$

So, eigenvalue satisfies the condition that $\abs{\lambda}=1$\\

Thus, options (3) and (4) are correct. \\ \\ \\

This can be verified by examples.\\ \\ \\
1. For $\lambda_1$ = 1 \\ \\
$\vec{P} = \myvec{1 & 0 & 0 \\ 0 & 1 & 0 \\ 0 & 0 & 1}$ \\
$\vec{P}^T\vec{P} = \vec{I}$ \\
Eigenvalue of $\vec{P}$ is 1. \\ \\

2. For $\lambda_2 = -1$ \\ \\
$\vec{P} = \myvec{-1 & 0 & 0 \\ 0 & 1 & 0 \\ 0 & 0 & -1}$ \\
$\vec{P}^T\vec{P} = \vec{I}$ \\
Eigenvalue of $\vec{P}$ is -1. \\ \\
\end{document}