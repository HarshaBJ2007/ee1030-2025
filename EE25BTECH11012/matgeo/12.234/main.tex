\let\negmedspace\undefined
\let\negthickspace\undefined
\documentclass[journal]{IEEEtran}
\usepackage[a5paper, margin=10mm, onecolumn]{geometry}
%\usepackage{lmodern} % Ensure lmodern is loaded for pdflatex
\usepackage{tfrupee} % Include tfrupee package

\setlength{\headheight}{1cm} % Set the height of the header box
\setlength{\headsep}{0mm}     % Set the distance between the header box and the top of the text

\usepackage{gvv-book}
\usepackage{gvv}
\usepackage{cite}
\usepackage{amsmath,amssymb,amsfonts,amsthm}
\usepackage{algorithmic}
\usepackage{graphicx}
\usepackage{textcomp}
\usepackage{xcolor}
\usepackage{txfonts}
\usepackage{listings}
\usepackage{enumitem}
\usepackage{mathtools}
\usepackage{gensymb}
\usepackage{comment}
\usepackage[breaklinks=true]{hyperref}
\usepackage{tkz-euclide} 
\usepackage{listings}
% \usepackage{gvv}                                        
\def\inputGnumericTable{}                                 
\usepackage[latin1]{inputenc}                                
\usepackage{color}                                            
\usepackage{array}                                            
\usepackage{longtable}                                       
\usepackage{calc}                                             
\usepackage{multirow}                                         
\usepackage{hhline}                                           
\usepackage{ifthen}                                           
\usepackage{lscape}
\usepackage{multicol}
\begin{document}

\bibliographystyle{IEEEtran}
\vspace{3cm}

\title{12.234}
\author{EE25BTECH11012-BEERAM MADHURI}
% \maketitle
% \newpage
% \bigskip
{\let\newpage\relax\maketitle}

\renewcommand{\thefigure}{\theenumi}
\renewcommand{\thetable}{\theenumi}
\setlength{\intextsep}{10pt} % Space between text and floats


\numberwithin{equation}{enumi}
\numberwithin{figure}{enumi}
\renewcommand{\thetable}{\theenumi}


\textbf{Question}:\\
Consider the set of vectors in three-dimensional real vector space $\mathbb{R}^3$,
\[S = \{(1, 1, 1), (1, -1, 1), (1, 1, -1)\}.\]

Which one of the following statements is true?

\begin{enumerate}
\item[a)] $S$ is not a linearly independent set.
\item[b)] $S$ is a basis for $\mathbb{R}^3$.
\item[c)] The vectors in $S$ are orthogonal.
\item[d)] An orthogonal set of vectors cannot be generated from $S$.
\end{enumerate}
\textbf{Solution:}\\
let the vectors in S be:
\begin{table}[H]
    \centering
    \begin{tabular}[12pt]{ |c| c|}
    \hline
    \textbf{Point} & \textbf{Vector}\\ 
    \hline
    $\mathbf{v_1}$ &  $\myvec{1\\1\\1}$\\
    \hline
    $\mathbf{v_2}$ &   $\myvec{1\\-1\\1}$\\
    \hline
     $\mathbf{v_3}$ &  $\myvec{1\\1\\-1}$\\
    \hline
    \end{tabular}
    \caption{Variables used}
    \label{table 12.234}
\end{table}
Let $A$ be the matrix with its columns as vectors of $S$

\begin{align}
A = \begin{pmatrix}1 & 1 & 1 \\1 & -1 & 1 \\1 & 1 & -1\end{pmatrix}
\end{align}
\textbf{Option a:}\\
The column vectors of a matrix $A$ are \textbf{linearly independent} if and only if the equation
\begin{align}
A\vec{x} = \vec{0}\\
\text{has only the trivial solution}\\
(\vec{x} = \vec{0})
\end{align}
We can find the solution by guassian elimination of $A$.
\begin{align}
\text{Perform row operations to get it into row echelon form:}\\\\
\begin{pmatrix}1 & 1 & 1 \\1 & -1 & 1 \\1 & 1 & -1\end{pmatrix}\xrightarrow[
R_2 \to R_2 - R_1]{R_3 \to R_3 - R_1}
    \begin{pmatrix} 1 & 1 & 1 \\ 0 & -2 & 0 \\ 0 & 0 & -2 \end{pmatrix}\\\\
\begin{pmatrix} 1 & 1 & 1 \\ 0 & -2 & 0 \\ 0 & 0 & -2 \end{pmatrix}\xrightarrow[
R_2 \to R_2 / -2]{R_3 \to R_3 / -2}
    \begin{pmatrix} 1 & 1 & 1 \\ 0 & 1 & 0 \\ 0 & 0 & 1 \end{pmatrix}\\\\
 \begin{pmatrix} 1 & 1 & 1 \\ 0 & 1 & 0 \\ 0 & 0 & 1 \end{pmatrix}\xrightarrow{
    R_1 \to R_1 - R_3}
    \begin{pmatrix} 1 & 1 & 0 \\ 0 & 1 & 0 \\ 0 & 0 & 1 \end{pmatrix}\xrightarrow{  R_1 \to R_1 - R_2}
    \begin{pmatrix} 1 & 0 & 0 \\ 0 & 1 & 0 \\ 0 & 0 & 1 \end{pmatrix}
    \end{align}\\
The reduced row echelon form of $A$ is the Identity matrix $I$.\\\\
$\therefore$ The only possible solution is the trivialsolution:
$\vec{x}=0$\\
$\therefore$ Vectors are linearly independent\\\\
\textbf{Option b:}\\
Since there are 3 linearly independent vectors in $\mathbb{R}^3$ \\they form a basis for $\mathbb{R}^3$\\\\
\textbf{Option c:}\\
    Let the vector be $\vec{v_1, v_2, v_3}$.
    \begin{align}
    \vec{v_1^\top v_2} &\neq 0 \\
    \vec{v_1^\top v_3 }&\neq 0 \\
    \vec{v_2^\top v_3 }&\neq 0
    \end{align}
    $\therefore$ These vectors are not orthogonal\\\\
\textbf{Option d:}\\
\text{Applying Gram-Schmidt process:} \\
\text{let the orthogonal vectors be $\vec{u_1, u_2, u_3}$ generated from $\vec{v_1, v_2, v_3}$}\\
\begin{align}\vec{u_1} = \vec{v_1} = \begin{pmatrix} 1 \\ 1\\1 \end{pmatrix}\\
\vec{u_2} &= \vec{v_2} - (\vec{u_1^\top v_2}) \hat{\vec{u}}_1 \\
\vec{u_2} &= \vec{v_2} - \left(\frac{\vec{v_2}^\top \vec{u_1}}{\vec{u_1}^\top \vec{u_1}} \right) \vec{u_1}\\ &= \begin{pmatrix} 2/3 \\ -4/3 \\ 2/3 \end{pmatrix} \\
\vec{u_3} &= \vec{v_3} - (\hat{\vec{u}}_2^\top \vec{v_3}) \hat{\vec{u}}_2 \\&= \vec{v_3} - \left( \frac{\vec{u_2}^\top \vec{v_3}}{\vec{u_2}^\top \vec{u_2}} \right) \vec{u_2} \\&= \begin{pmatrix} 1 \\ 0 \\ -1 \end{pmatrix}\\
\vec{u_1}^\top \vec{u_2} &= 0 \\\vec{u_2}^\top \vec{u_3} &= 0 \\\vec{u_1}^\top \vec{u_3} &= 0
\end{align}
$\therefore$ an orthogonal set of vectors can be generated from S.\\
$\therefore$ Options b and d are correct.
\end{document}
