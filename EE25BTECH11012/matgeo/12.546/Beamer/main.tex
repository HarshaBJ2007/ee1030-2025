\documentclass{beamer}
\usepackage[utf8]{inputenc}

\usetheme{Madrid}
\usecolortheme{default}
\usepackage{amsmath,amssymb,amsfonts,amsthm}
\usepackage{txfonts}
\usepackage{tkz-euclide}
\usepackage{listings}
\usepackage{adjustbox}
\usepackage{array}
\usepackage{tabularx}
\usepackage{gvv}
\usepackage{lmodern}
\usepackage{circuitikz}
\usepackage{tikz}
\usepackage{graphicx}
\usepackage{multicol}
\setbeamertemplate{page number in head/foot}[totalframenumber]

\usepackage{tcolorbox}
\tcbuselibrary{minted,breakable,xparse,skins}



\definecolor{bg}{gray}{0.95}
\DeclareTCBListing{mintedbox}{O{}m!O{}}{%
  breakable=true,
  listing engine=minted,
  listing only,
  minted language=#2,
  minted style=default,
  minted options={%
    linenos,
    gobble=0,
    breaklines=true,
    breakafter=,,
    fontsize=\small,
    numbersep=8pt,
    #1},
  boxsep=0pt,
  left skip=0pt,
  right skip=0pt,
  left=25pt,
  right=0pt,
  top=3pt,
  bottom=3pt,
  arc=5pt,
  leftrule=0pt,
  rightrule=0pt,
  bottomrule=2pt,

  colback=bg,
  colframe=orange!70,
  enhanced,
  overlay={%
    \begin{tcbclipinterior}
    \fill[orange!20!white] (frame.south west) rectangle ([xshift=20pt]frame.north west);
    \end{tcbclipinterior}},
  #3,
}
\lstset{
    language=C,
    basicstyle=\ttfamily\small,
    keywordstyle=\color{blue},
    stringstyle=\color{orange},
    commentstyle=\color{green!60!black},
    numbers=left,
    numberstyle=\tiny\color{gray},
    breaklines=true,
    showstringspaces=false,
}
%------------------------------------------------------------
%This block of code defines the information to appear in the
%Title page
\title %optional
{12.546}
\date{October  2025}
%\subtitle{A short story}

\author % (optional)
{BEERAM MADHURI - EE25BTECH11012}



\begin{document}


\frame{\titlepage}
\begin{frame}{Question}
Consider the following two statements

P: $\begin{pmatrix} 0 & 5 \\ 0 & 7 \end{pmatrix}$ has infinitely many LU factorizations, where $\mathbf{L}$ is lower triangular with each diagonal entry 1 and $\mathbf{U}$ is upper triangular.

Q: $\begin{pmatrix} 0 & 0 \\ 2 & 5 \end{pmatrix}$ has no LU factorization, where $\mathbf{L}$ is lower triangular with each diagonal entry 1 and $\mathbf{U}$ is upper triangular.

Then which one of the following options is correct? \hfill (MA 2018)

\begin{enumerate}
    \item[a)] P is TRUE and Q is FALSE
    \item[b)] Both P and Q are TRUE
    \item[c)] P is FALSE and Q is TRUE
    \item[d)] Both P and Q are FALSE
\end{enumerate}
\end{frame}
 
\begin{frame}{given data}
\begin{table}[h!]
    \centering
    \begin{tabular}[12pt]{ |c| c|}
    \hline
    \textbf{Point} & \textbf{Vector}\\ 
    \hline
    $\mathbf{v_1}$ &  $\myvec{1\\1\\1}$\\
    \hline
    $\mathbf{v_2}$ &   $\myvec{1\\-1\\1}$\\
    \hline
     $\mathbf{v_3}$ &  $\myvec{1\\1\\-1}$\\
    \hline
    \end{tabular}
    \caption{Variables used}
    \label{table 12.546}
\end{table}
\end{frame}

\begin{frame}{solution}
    \frametitle{solution}
Let 
\begin{align}
L = \begin{pmatrix} 1 & 0 \\ l_{21} & 1 \end{pmatrix}\\
U = \begin{pmatrix} u_{11} & u_{12} \\ 0 & u_{22} \end{pmatrix}\\
LU = \begin{pmatrix} u_{11} & u_{12} \\ l_{21}u_{11} & l_{21}u_{12} + u_{22} \end{pmatrix}
\end{align}
\end{frame}
\begin{frame}
\textbf{Statement P:}
\begin{align}
LU = \begin{pmatrix} 0 & 5 \\ 0 & 7 \end{pmatrix} \\
u_{11} = 0, u_{12} = 5\\
l_{21}u_{12} + u_{22} = 7
\end{align}
has infinite solutions.

$\therefore$ there are infinitely many pairs $(l_{21}, u_{22})$

Statement P is true\\
\end{frame}
\begin{frame}
\textbf{Statement Q:}
\begin{align}
LU = \begin{pmatrix} 0 & 0 \\ 2 & 5 \end{pmatrix}\\
u_{11} = 0\\
l_21u_11=0\\
\text{but} l_21u_11=2
\end{align}
$\therefore$ no pairs(L,U) exists.\\
$\therefore$ Statement Q is true.\\
Option b is correct.
\end{frame}


\begin{frame}[fragile]
\frametitle{Python Code}
\begin{lstlisting}
def print_matrix(name: str, matrix: list[list[float]]):
    """Prints a 2x2 matrix with a given name in a formatted way."""
    print(f"Matrix {name}:")
    for row in matrix:
        print("  [ ", end="")
        for val in row:
            # Format to 2 decimal places with a width of 5 characters
            print(f"{val:5.2f} ", end="")
        print("]")
    print()
\end{lstlisting}
\end{frame}

\begin{frame}[fragile]
\frametitle{Python Code}
\begin{lstlisting}
def analyze_lu_factorization(matrix_name: str, A: list[list[float]]):
    """
    Analyzes the LU factorization for a 2x2 matrix A.

    It checks the conditions derived from A = LU, where L is lower
    triangular with 1s on the diagonal and U is upper triangular.
\end{lstlisting}
\end{frame}

\begin{frame}[fragile]
\frametitle{Python Code}
\begin{lstlisting}
    """
    print(f"--- Analyzing Matrix {matrix_name} ---")
    print_matrix(matrix_name, A)

    # For a matrix A = [[a, b], [c, d]], we want to find L and U such that A = LU:
    # L = [[1, 0], [l21, 1]] and U = [[u11, u12], [0, u22]]
    #
    # Multiplying L and U and equating to A gives the system of equations:
\end{lstlisting}
\end{frame}

\begin{frame}[fragile]
\frametitle{Python Code}
\begin{lstlisting}
    # 1) u11 = A[0][0]
    # 2) u12 = A[0][1]
    # 3) l21 * u11 = A[1][0]  <-- This is the critical equation.
    # 4) l21 * u12 + u22 = A[1][1]
    u11 = A[0][0]
    a10 = A[1][0] # The element at row 1, column 0
    print("From the definition, the critical equation is: l21 * u11 = A[1][0]")
\end{lstlisting}
\end{frame}

\begin{frame}[fragile]
\frametitle{Python Code}
\begin{lstlisting}
    print(f"Substituting known values from matrix {matrix_name}:")
    print(f"  -> l21 * {u11:.2f} = {a10:.2f}\n")

    # Check the condition of the critical equation (3)
    if u11 == 0:
        if a10 == 0:
            # This is the case for Matrix P: l21 * 0 = 0
            print(f"Result for {matrix_name}:")
            print("The equation becomes 0 = 0, which is always true.")
\end{lstlisting}
\end{frame}

\begin{frame}[fragile]
\frametitle{Python Code}
\begin{lstlisting}
            print("This means 'l21' can be any real number, leading to infinitely many solutions.")
            print(f"Conclusion: Statement {matrix_name} is TRUE.")
        else:
            # This is the case for Matrix Q: l21 * 0 = 2
            print(f"Result for {matrix_name}:")
            print(f"The equation becomes 0 = {a10:.2f}, which is a contradiction.")
            print("No value of 'l21' can satisfy this, so no LU factorization exists.")
            print(f"Conclusion: Statement {matrix_name} is TRUE.")
\end{lstlisting}
\end{frame}

\begin{frame}[fragile]
\frametitle{Python Code}
\begin{lstlisting}
    else:
        # This is the standard case where a unique solution would exist
        print(f"Result for {matrix_name}:")
        print(f"Since u11 ({u11:.2f}) is non-zero, a unique solution for l21 could be found.")
        print(f"This case does not apply to matrices P or Q.")
# This block ensures the code runs only when the script is executed directly
\end{lstlisting}
\end{frame}

\begin{frame}[fragile]
\frametitle{Python Code}
\begin{lstlisting}
if _name_ == "_main_":
    # Matrix from Statement P
    P = [
        [0.0, 5.0],
        [0.0, 7.0]
    ]

    # Matrix from Statement Q
    Q = [
        [0.0, 0.0],
        [2.0, 5.0]
    ]
\end{lstlisting}
\end{frame}

\begin{frame}[fragile]
\frametitle{Python Code}
\begin{lstlisting} 
    print("This program analyzes the LU factorization for the two given matrices.\n")
    # Analyze Matrix P
    analyze_lu_factorization("P", P)
    # Analyze Matrix Q
    analyze_lu_factorization("Q", Q)
    print("Final Conclusion:")
    print("Both statements P and Q are TRUE. The correct option is (b).")
\end{lstlisting}
\end{frame}

\begin{frame}[fragile]
\frametitle{C Code}
\begin{lstlisting}
#include <stdio.h>

/**
 * @file compile_time_lu.c
 * @brief Analyzes LU factorization at compile time using the preprocessor.
 *
 * The logic to determine the truthiness of statements P and Q is resolved
 * before the program is compiled. The runtime executable only contains the
 * final, pre-determined answer.
 */
\end{lstlisting}
\end{frame}

\begin{frame}[fragile]
\frametitle{C Code}
\begin{lstlisting}
// --- Matrix P Definition ---
// A = [[P_00, P_01], [P_10, P_11]]
#define P_00 0
#define P_01 5
#define P_10 0
#define P_11 7
// --- Matrix Q Definition ---
// A = [[Q_00, Q_01], [Q_10, Q_11]]
#define Q_00 0
#define Q_01 0
#define Q_10 2
#define Q_11 5
\end{lstlisting}
\end{frame}

\begin{frame}[fragile]
\frametitle{C Code}
\begin{lstlisting}
// --- COMPILE-TIME ANALYSIS ---
// The preprocessor will evaluate these #if statements.
// Analyze Statement P: "P has infinitely many LU factorizations"
// This is TRUE if A[0][0] is 0 AND A[1][0] is 0, leading to 0 = 0.
#if P_00 == 0 && P_10 == 0
  #define P_IS_TRUE 1
#else
  #define P_IS_TRUE 0
#endif
\end{lstlisting}
\end{frame}

\begin{frame}[fragile]
\frametitle{C Code}
\begin{lstlisting}
// Analyze Statement Q: "Q has no LU factorization"
// This is TRUE if A[0][0] is 0 AND A[1][0] is not 0, leading to 0 = non-zero.
#if Q_00 == 0 && Q_10 != 0
  #define Q_IS_TRUE 1
#else
  #define Q_IS_TRUE 0
#endif
\end{lstlisting}
\end{frame}

\begin{frame}[fragile]
\frametitle{C Code}
\begin{lstlisting}
int main() {
    printf("This conclusion was determined entirely at COMPILE TIME.\n");
    printf("The running program is just printing the pre-calculated result.\n\n");

    // The preprocessor uses the P_IS_TRUE and Q_IS_TRUE macros
    // to select ONLY ONE of the following printf statements to
    // include in the final compiled program.
\end{lstlisting}
\end{frame}

\begin{frame}[fragile]
\frametitle{C Code}
\begin{lstlisting}
    #if P_IS_TRUE && Q_IS_TRUE
      printf("Conclusion: Both P and Q are TRUE. The correct option is (b).\n");
    #elif P_IS_TRUE && !Q_IS_TRUE
      printf("Conclusion: P is TRUE and Q is FALSE. The correct option is (a).\n");
\end{lstlisting}
\end{frame}

\begin{frame}[fragile]
\frametitle{C Code}
\begin{lstlisting}
    #elif !P_IS_TRUE && Q_IS_TRUE
      printf("Conclusion: P is FALSE and Q is TRUE. The correct option is (c).\n");
    #else
      printf("Conclusion: Both P and Q are FALSE. The correct option is (d).\n");
    #endif
    return 0;
}
\end{lstlisting}
\end{frame}

\begin{frame}[fragile]
\frametitle{Python and C Code}
\begin{lstlisting}
from ctypes import c_int
"""
Simulating C preprocessor-based compile-time LU factorization analysis
in Python using ctypes and top-level evaluation.
"""
# --- Matrix P Definition ---
P_00 = c_int(0)
P_01 = c_int(5)
P_10 = c_int(0)
P_11 = c_int(7)
\end{lstlisting}
\end{frame}

\begin{frame}[fragile]
\frametitle{Python and C Code}
\begin{lstlisting}
# --- Matrix Q Definition ---
Q_00 = c_int(0)
Q_01 = c_int(0)
Q_10 = c_int(2)
Q_11 = c_int(5)
# --- "Compile-time" evaluation simulated at import time ---
# Analyze Statement P: "P has infinitely many LU factorizations"
\end{lstlisting}
\end{frame}

\begin{frame}[fragile]
\frametitle{Python and C Code}
\begin{lstlisting}
# TRUE if A[0][0] == 0 and A[1][0] == 0
if P_00.value == 0 and P_10.value == 0:
    P_IS_TRUE = True
else:
    P_IS_TRUE = False

# Analyze Statement Q: "Q has no LU factorization"
# TRUE if A[0][0] == 0 and A[1][0] != 0
if Q_00.value == 0 and Q_10.value != 0:
    Q_IS_TRUE = True
\end{lstlisting}
\end{frame}

\begin{frame}[fragile]
\frametitle{Python and C Code}
\begin{lstlisting}
else:
    Q_IS_TRUE = False


def main():
    print("This conclusion was determined entirely at 'import time' (simulating compile time).")
    print("The running program is just printing the pre-determined result.\n")
\end{lstlisting}
\end{frame}

\begin{frame}[fragile]
\frametitle{Python and C Code}
\begin{lstlisting}
    if P_IS_TRUE and Q_IS_TRUE:
        print("Conclusion: Both P and Q are TRUE. The correct option is (b).")
    elif P_IS_TRUE and not Q_IS_TRUE:
        print("Conclusion: P is TRUE and Q is FALSE. The correct option is (a).")
    elif not P_IS_TRUE and Q_IS_TRUE:
        print("Conclusion: P is FALSE and Q is TRUE. The correct option is (c).")
\end{lstlisting}
\end{frame}

\begin{frame}[fragile]
\frametitle{Python and C Code}
\begin{lstlisting}
    else:
        print("Conclusion: Both P and Q are FALSE. The correct option is (d).")


if __name__ == "__main__":
    main()
\end{lstlisting}
\end{frame}

\end{document}