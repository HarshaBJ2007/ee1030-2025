\let\negmedspace\undefined
\let\negthickspace\undefined
\documentclass[journal]{IEEEtran}
\usepackage[a5paper, margin=10mm, onecolumn]{geometry}
%\usepackage{lmodern} % Ensure lmodern is loaded for pdflatex
\usepackage{tfrupee} % Include tfrupee package

\setlength{\headheight}{1cm} % Set the height of the header box
\setlength{\headsep}{0mm}     % Set the distance between the header box and the top of the text

\usepackage{gvv-book}
\usepackage{gvv}
\usepackage{cite}
\usepackage{amsmath,amssymb,amsfonts,amsthm}
\usepackage{algorithmic}
\usepackage{graphicx}
\usepackage{textcomp}
\usepackage{xcolor}
\usepackage{txfonts}
\usepackage{listings}
\usepackage{enumitem}
\usepackage{mathtools}
\usepackage{gensymb}
\usepackage{comment}
\usepackage[breaklinks=true]{hyperref}
\usepackage{tkz-euclide} 
\usepackage{listings}
% \usepackage{gvv}                                        
\def\inputGnumericTable{}                                 
\usepackage[latin1]{inputenc}                                
\usepackage{color}                                            
\usepackage{array}                                            
\usepackage{longtable}                                       
\usepackage{calc}                                             
\usepackage{multirow}                                         
\usepackage{hhline}                                           
\usepackage{ifthen}                                           
\usepackage{lscape}
\begin{document}

\bibliographystyle{IEEEtran}
\vspace{3cm}

\title{5.4.7}
\author{EE25BTECH11012-BEERAM MADHURI}
% \maketitle
% \newpage
% \bigskip
{\let\newpage\relax\maketitle}

\renewcommand{\thefigure}{\theenumi}
\renewcommand{\thetable}{\theenumi}
\setlength{\intextsep}{10pt} % Space between text and floats


\numberwithin{equation}{enumi}
\numberwithin{figure}{enumi}
\renewcommand{\thetable}{\theenumi}

\textbf{Question}:\\
Using elementary transformations, find the inverse of the following matrix\\
\begin{align*}
\begin{pmatrix}
$2$ & $2$\\
$4$ & $3$
\end{pmatrix}
\end{align*}

\textbf{Solution:}\\
We know that
\begin{align}
\vec{A}^{-1}\vec{A} = \vec{I} 
\end{align}

\text where $\vec{I}$ is the 2$\times$2 identity matrix\\
Now we get the augmented matrix\\
\begin{align}
\begin{pmatrix}2 & 2 & | & 1 & 0 \\4 & 3 & | & 0 & 1\end{pmatrix}\xrightarrow{R_2 \to R_2 -{2}R_1}\begin{pmatrix}2 & 2 & | & 1 & 0 \\0 & -1 & | & -{2} & 1\end{pmatrix}\\
\xrightarrow[\substack{R_2 \to -R_2}]{R_1 \to \frac{R_1}{2}}\begin{pmatrix}1 & 1 & | & \frac{1}{2} & 0 \\0 & 1 & | & {2} & -1\end{pmatrix}\xrightarrow{R_1 \to R_1 - R_2}\begin{pmatrix}1 & 0 & | & -\frac{3}{2} & 1 \\0 & 1 & | &{2} & -1\end{pmatrix}
\end{align}
Therefore
\begin{align}
\vec{A}^{-1} = \begin{pmatrix}-\frac{3}{2} & 1 \\{2} & -1\end{pmatrix}
\end{align}
This can be verified by $\vec{A}^{-1}\vec{A} = \vec{I}$
\end{document}
