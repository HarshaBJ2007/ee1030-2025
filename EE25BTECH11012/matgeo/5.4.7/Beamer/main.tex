\documentclass{beamer}
\usepackage[utf8]{inputenc}

\usetheme{Madrid}
\usecolortheme{default}
\usepackage{amsmath,amssymb,amsfonts,amsthm}
\usepackage{txfonts}
\usepackage{tkz-euclide}
\usepackage{listings}
\usepackage{adjustbox}
\usepackage{array}
\usepackage{tabularx}
\usepackage{gvv}
\usepackage{lmodern}
\usepackage{circuitikz}
\usepackage{tikz}
\usepackage{graphicx}

\setbeamertemplate{page number in head/foot}[totalframenumber]

\usepackage{tcolorbox}
\tcbuselibrary{minted,breakable,xparse,skins}



\definecolor{bg}{gray}{0.95}
\DeclareTCBListing{mintedbox}{O{}m!O{}}{%
  breakable=true,
  listing engine=minted,
  listing only,
  minted language=#2,
  minted style=default,
  minted options={%
    linenos,
    gobble=0,
    breaklines=true,
    breakafter=,,
    fontsize=\small,
    numbersep=8pt,
    #1},
  boxsep=0pt,
  left skip=0pt,
  right skip=0pt,
  left=25pt,
  right=0pt,
  top=3pt,
  bottom=3pt,
  arc=5pt,
  leftrule=0pt,
  rightrule=0pt,
  bottomrule=2pt,

  colback=bg,
  colframe=orange!70,
  enhanced,
  overlay={%
    \begin{tcbclipinterior}
    \fill[orange!20!white] (frame.south west) rectangle ([xshift=20pt]frame.north west);
    \end{tcbclipinterior}},
  #3,
}
\lstset{
    language=C,
    basicstyle=\ttfamily\small,
    keywordstyle=\color{blue},
    stringstyle=\color{orange},
    commentstyle=\color{green!60!black},
    numbers=left,
    numberstyle=\tiny\color{gray},
    breaklines=true,
    showstringspaces=false,
}
%------------------------------------------------------------
%This block of code defines the information to appear in the
%Title page
\title %optional
{4.5.14}
\date{September  2025}
%\subtitle{A short story}

\author % (optional)
{BEERAM MADHURI - EE25BTECH11012}


\begin{document}


\frame{\titlepage}
\begin{frame}{Question}
 Using elementary transformations, find the inverse of the following matrix\\
\begin{align*}
\begin{pmatrix}
$2$ & $2$\\
$4$ & $3$
\end{pmatrix}
\end{align*}
\end{frame}
 

\begin{frame}{solution}
    \frametitle{finding the Inverse of matrix: }
We know that
\begin{align}
\vec{A}^{-1}\vec{A} = \vec{I} 
\end{align}

\text where $\vec{I}$ is the 2$\times$2 identity matrix\\
Now we get the augmented matrix\\
\begin{align}
\begin{pmatrix}2 & 2 & | & 1 & 0 \\4 & 3 & | & 0 & 1\end{pmatrix}\xrightarrow{R_2 \to R_2 -{2}R_1}\begin{pmatrix}2 & 2 & | & 1 & 0 \\0 & -1 & | & -{2} & 1\end{pmatrix}\\
\end{align}
\end{frame}
\begin{frame}
\begin{align}
\xrightarrow[\substack{R_2 \to -R_2}]{R_1 \to \frac{R_1}{2}}\begin{pmatrix}1 & 1 & | & \frac{1}{2} & 0 \\0 & 1 & | & {2} & -1\end{pmatrix}\xrightarrow{R_1 \to R_1 - R_2}\begin{pmatrix}1 & 0 & | & -\frac{3}{2} & 1 \\0 & 1 & | &{2} & -1\end{pmatrix}
\end{align}
Therefore
\begin{align}
\vec{A}^{-1} = \begin{pmatrix}-\frac{3}{2} & 1 \\{2} & -1\end{pmatrix}
\end{align}
This can be verified by $\vec{A}^{-1}\vec{A} = \vec{I}$
\end{frame}

\begin{frame}[fragile]
    \frametitle{Python Code}
    \begin{lstlisting}
import numpy as np


a = np.array([[2, 2], [4, 3]])
inverse_a = np.array([[],[]])

b = a@inverse_a
print(b)
    \end{lstlisting}
\end{frame}



\begin{frame}[fragile]
\frametitle{C Code}
\begin{lstlisting}
#include <stdio.h>
#include <stdlib.h> // For exit()

// A function to print a 2x2 matrix
void printMatrix(double mat[2][2]) {
    for (int i = 0; i < 2; i++) {
        printf(" |");
        for (int j = 0; j < 2; j++) {
            // %.2f prints the float with 2 decimal places
            printf("%8.2f", mat[i][j]);
        }
        printf(" |\n");
    }
}
\end{lstlisting}
\end{frame}

\begin{frame}[fragile]
\frametitle{C Code}
\begin{lstlisting}
   
int main() {
    // The original matrix from the question
    double matrix[2][2] = {
        {2.0, 2.0},
        {4.0, 3.0}
    };

    printf("Original Matrix A:\n");
    printMatrix(matrix);
\end{lstlisting}
\end{frame}

\begin{frame}[fragile]
\frametitle{C Code}
\begin{lstlisting}
   
    // --- Step 1: Check if the inverse exists (determinant != 0) ---
    // Determinant = ad - bc
    double det = matrix[0][0] * matrix[1][1] - matrix[0][1] * matrix[1][0];
    if (det == 0) {
        printf("\nInverse does not exist because the determinant is zero.\n");
        return 1; // Exit with an error
    }
\end{lstlisting}
\end{frame}

\begin{frame}[fragile]
\frametitle{C Code}
\begin{lstlisting}
    // --- Step 2: Create an augmented matrix [A|I] ---
    // 'I' is the 2x2 identity matrix
    double augmented[2][4];
    for (int i = 0; i < 2; i++) {
        for (int j = 0; j < 2; j++) {
            augmented[i][j] = matrix[i][j]; // Copy matrix A
        }
    }
    // Append the identity matrix
    augmented[0][2] = 1.0;
    augmented[0][3] = 0.0;
    augmented[1][2] = 0.0;
    augmented[1][3] = 1.0;
\end{lstlisting}
\end{frame}

\begin{frame}[fragile]
\frametitle{C Code}
\begin{lstlisting}
    // --- Step 3: Apply Gauss-Jordan elimination ---
    // Goal: Transform the left side of the augmented matrix into the identity matrix.

    // Make the first element of the first row (pivot) equal to 1
    // R1 -> R1 / 2
    double pivot1 = augmented[0][0];
    for (int j = 0; j < 4; j++) {
        augmented[0][j] /= pivot1;
    }
\end{lstlisting}
\end{frame}

\begin{frame}[fragile]
\frametitle{C Code}
\begin{lstlisting}
    // Make the first element of the second row equal to 0
    // R2 -> R2 - 4 * R1
    double factor1 = augmented[1][0];
    for (int j = 0; j < 4; j++) {
        augmented[1][j] -= factor1 * augmented[0][j];
    }
    // Make the second element of the second row (pivot) equal to 1
    // R2 -> R2 / -1
    double pivot2 = augmented[1][1];
    for (int j = 0; j < 4; j++) {
        augmented[1][j] /= pivot2;
    }
\end{lstlisting}
\end{frame}

\begin{frame}[fragile]
\frametitle{C Code}
\begin{lstlisting}
    // Make the second element of the first row equal to 0
    // R1 -> R1 - 1 * R2
    double factor2 = augmented[0][1];
    for (int j = 0; j < 4; j++) {
        augmented[0][j] -= factor2 * augmented[1][j];
    }

    // --- Step 4: Extract the inverse matrix ---
    // The inverse is now on the right side of the augmented matrix
\end{lstlisting}
\end{frame}

\begin{frame}[fragile]
\frametitle{C Code}
\begin{lstlisting}
    double inverse[2][2];
    for (int i = 0; i < 2; i++) {
        for (int j = 0; j < 2; j++) {
            inverse[i][j] = augmented[i][j + 2];
        }
    }
    printf("\nFound Inverse Matrix A⁻¹:\n");
    printMatrix(inverse);
    return 0;
}
\end{lstlisting}
\end{frame}

\begin{frame}[fragile]
\frametitle{Python and C Code}
\begin{lstlisting}
import ctypes

# Define the C double type
DoubleArray2x2 = ctypes.c_double * 2
Matrix2x2 = DoubleArray2x2 * 2  # 2x2 matrix

def print_matrix(mat):
    for i in range(2):
        print(" |", end='')
        for j in range(2):
            print(f"{mat[i][j]:8.2f}", end='')
        print(" |")
\end{lstlisting}
\end{frame}

\begin{frame}[fragile]
\frametitle{Python and C Code}
\begin{lstlisting}
# Initialize matrix A
matrix = Matrix2x2(
    DoubleArray2x2(2.0, 2.0),
    DoubleArray2x2(4.0, 3.0)
)

print("Original Matrix A:")
print_matrix(matrix)
\end{lstlisting}
\end{frame}

\begin{frame}[fragile]
\frametitle{Python and C Code}
\begin{lstlisting}
# Step 1: Calculate determinant
det = matrix[0][0] * matrix[1][1] - matrix[0][1] * matrix[1][0]
if det == 0:
    print("\nInverse does not exist because the determinant is zero.")
    exit(1)
# Step 2: Create augmented matrix [A | I] using ctypes
AugmentedRow = ctypes.c_double * 4
AugmentedMatrix = AugmentedRow * 2
augmented = AugmentedMatrix(
    AugmentedRow(matrix[0][0], matrix[0][1], 1.0, 0.0),
    AugmentedRow(matrix[1][0], matrix[1][1], 0.0, 1.0)
)
\end{lstlisting}
\end{frame}

\begin{frame}[fragile]
\frametitle{Python and C Code}
\begin{lstlisting}
# Step 3: Gauss-Jordan Elimination
# Row 1 normalization
pivot1 = augmented[0][0]
for j in range(4):
    augmented[0][j] /= pivot1
# Row 2 elimination
factor1 = augmented[1][0]
for j in range(4):
    augmented[1][j] -= factor1 * augmented[0][j]
\end{lstlisting}
\end{frame}

\begin{frame}[fragile]
\frametitle{Python and C Code}
\begin{lstlisting}
# Row 2 normalization
pivot2 = augmented[1][1]
for j in range(4):
    augmented[1][j] /= pivot2

# Row 1 elimination
factor2 = augmented[0][1]
for j in range(4):
    augmented[0][j] -= factor2 * augmented[1][j]
\end{lstlisting}
\end{frame}

\begin{frame}[fragile]
\frametitle{Python and C Code}
\begin{lstlisting}
# Step 4: Extract inverse
inverse = Matrix2x2(
    DoubleArray2x2(augmented[0][2], augmented[0][3]),
    DoubleArray2x2(augmented[1][2], augmented[1][3])
)

print("\nFound Inverse Matrix A⁻¹:")
print_matrix(inverse)

\end{lstlisting}
\end{frame}

\end{document}