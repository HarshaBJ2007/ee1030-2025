\documentclass{beamer}
\usepackage[utf8]{inputenc}

\usetheme{Madrid}
\usecolortheme{default}
\usepackage{amsmath,amssymb,amsfonts,amsthm}
\usepackage{txfonts}
\usepackage{tkz-euclide}
\usepackage{listings}
\usepackage{adjustbox}
\usepackage{array}
\usepackage{tabularx}
\usepackage{gvv}
\usepackage{lmodern}
\usepackage{circuitikz}
\usepackage{tikz}
\usepackage{graphicx}
\usepackage{multicol}
\setbeamertemplate{page number in head/foot}[totalframenumber]

\usepackage{tcolorbox}
\tcbuselibrary{minted,breakable,xparse,skins}



\definecolor{bg}{gray}{0.95}
\DeclareTCBListing{mintedbox}{O{}m!O{}}{%
  breakable=true,
  listing engine=minted,
  listing only,
  minted language=#2,
  minted style=default,
  minted options={%
    linenos,
    gobble=0,
    breaklines=true,
    breakafter=,,
    fontsize=\small,
    numbersep=8pt,
    #1},
  boxsep=0pt,
  left skip=0pt,
  right skip=0pt,
  left=25pt,
  right=0pt,
  top=3pt,
  bottom=3pt,
  arc=5pt,
  leftrule=0pt,
  rightrule=0pt,
  bottomrule=2pt,

  colback=bg,
  colframe=orange!70,
  enhanced,
  overlay={%
    \begin{tcbclipinterior}
    \fill[orange!20!white] (frame.south west) rectangle ([xshift=20pt]frame.north west);
    \end{tcbclipinterior}},
  #3,
}
\lstset{
    language=C,
    basicstyle=\ttfamily\small,
    keywordstyle=\color{blue},
    stringstyle=\color{orange},
    commentstyle=\color{green!60!black},
    numbers=left,
    numberstyle=\tiny\color{gray},
    breaklines=true,
    showstringspaces=false,
}
%------------------------------------------------------------
%This block of code defines the information to appear in the
%Title page
\title %optional
{12.754}
\date{October  2025}
%\subtitle{A short story}

\author % (optional)
{BEERAM MADHURI - EE25BTECH11012}



\begin{document}


\frame{\titlepage}
\begin{frame}{Question}
Let $\mathbf{Q} = \begin{pmatrix} 1 & -2 \\ 2 & 1 \end{pmatrix}$ be a $2\times2$ matrix. Which one of the following statements is \textbf{TRUE}?

\begin{enumerate}
    \item[a)] $\mathbf{Q}$ is equal to its transpose.
    \item[b)] $\mathbf{Q}$ is equal to its inverse.
    \item[c)] $\mathbf{Q}$ is full rank.
    \item[d)] $\mathbf{Q}$ has linearly dependent columns.
\end{enumerate}
\end{frame}
 
\begin{frame}{solution}
    \frametitle{finding the properties of Q:}
    given
    \begin{align}
    \vec{Q}=\begin{pmatrix}
        1&-2\\2&1
    \end{pmatrix}
    \end{align}
 \textbf{a)}
    \begin{align}\vec{Q}^\top = \begin{pmatrix} 1 & 2 \\ -2 & 1 \end{pmatrix}\\
    \vec{Q} \neq \vec{Q}^\top
    \end{align}
\end{frame}
\begin{frame}
    \textbf{b)}
    \begin{align}
    \vec{Q}= \begin{pmatrix} 1 & -2 \\ 2 & 1 \end{pmatrix}\\
    \text{If } \vec{Q} = \vec{Q}^-1 \text{ then } \vec{Q}^2 = I\\
    \vec{Q}^2 = \begin{pmatrix} 1 & 2 \\ -2 & 1 \end{pmatrix} \begin{pmatrix} 1 & 2 \\ -2 & 1 \end{pmatrix}\\
    = \begin{pmatrix} -3 & 4 \\ -4 & -3 \end{pmatrix}
    \neq \vec{I}
    \end{align}
    \end{frame}
    \begin{frame}
\text{c)} 
\begin{align}
\vec{Q}= \begin{pmatrix} 1 & -2 \\ 2 & 1 \end{pmatrix}
\end{align}
    Using Row reduction:-
\begin{align}
\begin{pmatrix} 1 & -2 \\ 2 & 1 \end{pmatrix}\xrightarrow{R_2-2(R_1)}
\begin{pmatrix} 1 & -2 \\ 0 & -3
\end{pmatrix}\\
rank=2
\end{align}
$\therefore$ $\vec{Q}$ is a full rank Matrix.
\end{frame}
\begin{frame}
\textbf{d)}
Columns of $\vec{Q}$ are linearly dependent if
\begin{align}
\vec{c_1} = \lambda \vec{c_2} \quad (\lambda \neq 0)
\end{align}
where $\vec{c_1} =$ first column of $\vec{Q}$ \\
$\vec{c_2} =$ second column of $\vec{Q}$.
\begin{align}
\vec{c_1} = \begin{pmatrix} 1 \\ 2 \end{pmatrix}, \vec{c_2} = \begin{pmatrix} -2 \\ 1 \end{pmatrix}\\
\vec{c_1} \neq \lambda \vec{c_2} \text{ for any } \lambda \neq 0
\end{align}
$\therefore$ columns of $\vec{Q}$ are linearly independent.\\\\
$\therefore$ Option C is correct.
\end{frame}

\begin{frame}[fragile]
    \frametitle{Python Code}
    \begin{lstlisting}
import numpy as np
from numpy.linalg import inv, det, LinAlgError

def solve_matrix_problem():
    """
    Analyzes the matrix Q and verifies the given statements.
    """
    # Define the 2x2 matrix Q as a NumPy array
    Q = np.array([[1, -2],
                  [2, 1]])

    print(f"Given Matrix Q:\n{Q}\n")
\end{lstlisting}
\end{frame}

\begin{frame}[fragile]
\frametitle{Python Code}
\begin{lstlisting}
    # --- a) Check if Q is equal to its transpose ---
    print("a) Checking if Q is equal to its transpose...")
    # In NumPy, the .T attribute gets the transpose
    Q_T = Q.T
    print(f"Transpose Q^T:\n{Q_T}")
    # np.array_equal safely checks if two arrays are element-wise equal
    is_equal_to_transpose = np.array_equal(Q, Q_T)
    print(f"Result: Statement (a) is {is_equal_to_transpose}.\n")
\end{lstlisting}
\end{frame}

\begin{frame}[fragile]
\frametitle{Python Code}
\begin{lstlisting}
    # --- b) Check if Q is equal to its inverse ---
    print("b) Checking if Q is equal to its inverse...")
    try:
        # np.linalg.inv calculates the inverse
        Q_inv = inv(Q)
        print(f"Inverse Q^-1:\n{np.round(Q_inv, 2)}") # Round for cleaner display
        is_equal_to_inverse = np.array_equal(Q, Q_inv)
        print(f"Result: Statement (b) is {is_equal_to_inverse}.\n")
    except LinAlgError:
        # This block runs if the matrix has no inverse (is singular)
        print("Inverse does not exist.")
        print("Result: Statement (b) is False.\n")
\end{lstlisting}
\end{frame}

\begin{frame}[fragile]
\frametitle{Python Code}
\begin{lstlisting}
    # --- c) & d) Check rank and column dependency via the determinant ---
    print("c/d) Checking for full rank and column dependency...")
    # np.linalg.det calculates the determinant
    q_det = det(Q)
    print(f"Determinant of Q = {q_det:.2f}")

    # A square matrix has full rank if its determinant is non-zero.
    # Its columns are linearly dependent if the determinant is zero.
    if abs(q_det) > 1e-9:  # Use tolerance for floating point comparison
        print("Result: Statement (c) is True (Determinant is non-zero, so Q is of full rank).")
        print("Result: Statement (d) is False (Columns are linearly independent).\n")
\end{lstlisting}
\end{frame}

\begin{frame}[fragile]
\frametitle{Python Code}
\begin{lstlisting}
else:
        print("Result: Statement (c) is False (Determinant is zero, so Q is not of full rank).")
        print("Result: Statement (d) is True (Columns are linearly dependent).\n")
    print("Conclusion: The only TRUE statement is (c).")
    
if _name_ == "_main_":
    solve_matrix_problem()
\end{lstlisting}
\end{frame}

\begin{frame}[fragile]
\frametitle{C Code}
\begin{lstlisting}
#include <stdio.h>
#include <math.h> // Required for fabs() for floating-point comparisons

// --- Function Prototypes ---
void printMatrix(const char* name, double matrix[2][2]);
double determinant(double matrix[2][2]);
void transpose(double in[2][2], double out[2][2]);
int inverse(double in[2][2], double out[2][2]); // Returns 1 on success, 0 on failure
int areMatricesEqual(double A[2][2], double B[2][2]);
\end{lstlisting}
\end{frame}

\begin{frame}[fragile]
\frametitle{C Code}
\begin{lstlisting}
int main() {
    // Define the 2x2 matrix Q
    double Q[2][2] = {{1.0, -2.0}, {2.0, 1.0}};

    printf("Given Matrix Q:\n");
    printMatrix("Q", Q);
    // a) Check if Q is equal to its transpose.
    printf("a) Checking if Q == Q^T ...\n");
    double Q_T[2][2];
    transpose(Q, Q_T);
    printMatrix("Transpose Q^T", Q_T);
    printf("Result: Statement (a) is %s.\n\n", areMatricesEqual(Q, Q_T) ? "TRUE" : "FALSE");
\end{lstlisting}
\end{frame}

\begin{frame}[fragile]
\frametitle{C Code}
\begin{lstlisting}
    // b) Check if Q is equal to its inverse.
    printf("b) Checking if Q == Q^-1 ...\n");
    double Q_inv[2][2];
    if (inverse(Q, Q_inv)) { // Check if inverse exists before using it
        printMatrix("Inverse Q^-1", Q_inv);
        printf("Result: Statement (b) is %s.\n\n", areMatricesEqual(Q, Q_inv) ? "TRUE" : "FALSE");
    } else {
        printf("Inverse does not exist.\n");
        printf("Result: Statement (b) is FALSE.\n\n");
    }
\end{lstlisting}
\end{frame}

\begin{frame}[fragile]
\frametitle{C Code}
\begin{lstlisting}
    // c) & d) Check for full rank and column dependency using the determinant.
    printf("c/d) Checking rank and column dependency...\n");
    double det = determinant(Q);
    printf("Determinant of Q = %.2f\n", det);
    // If determinant is non-zero, it has full rank and independent columns.
    if (fabs(det) > 1e-9) {
        printf("Result: Statement (c) is TRUE (Determinant is non-zero, so Q is of full rank).\n");
        printf("Result: Statement (d) is FALSE (Columns are linearly independent).\n\n");
    } 
\end{lstlisting}
\end{frame}

\begin{frame}[fragile]
\frametitle{C Code}
\begin{lstlisting}
else {
        printf("Result: Statement (c) is FALSE (Determinant is zero, so Q is not of full rank).\n");
        printf("Result: Statement (d) is TRUE (Columns are linearly dependent).\n\n");
    }
    printf("Conclusion: The only TRUE statement is (c).\n");

    return 0;
}
\end{lstlisting}
\end{frame}

\begin{frame}[fragile]
\frametitle{C Code}
\begin{lstlisting}
void printMatrix(const char* name, double matrix[2][2]) {
    printf("%s = \n", name);
    printf("  | %6.2f  %6.2f |\n", matrix[0][0], matrix[0][1]);
    printf("  | %6.2f  %6.2f |\n", matrix[1][0], matrix[1][1]);
}

double determinant(double matrix[2][2]) {
    return matrix[0][0] * matrix[1][1] - matrix[0][1] * matrix[1][0];
}
\end{lstlisting}
\end{frame}

\begin{frame}[fragile]
\frametitle{C Code}
\begin{lstlisting}
void transpose(double in[2][2], double out[2][2]) {
    out[0][0] = in[0][0];
    out[0][1] = in[1][0];
    out[1][0] = in[0][1];
    out[1][1] = in[1][1];
}
int inverse(double in[2][2], double out[2][2]) {
    double det = determinant(in);

    // A matrix is invertible if and only if its determinant is non-zero.
    if (fabs(det) < 1e-9) {
        return 0; // No inverse exists
    }
\end{lstlisting}
\end{frame}

\begin{frame}[fragile]
\frametitle{C Code}
\begin{lstlisting}
    double inv_det = 1.0 / det;
    out[0][0] =  in[1][1] * inv_det;
    out[0][1] = -in[0][1] * inv_det;
    out[1][0] = -in[1][0] * inv_det;
    out[1][1] =  in[0][0] * inv_det;

    return 1; // Success
}
\end{lstlisting}
\end{frame}

\begin{frame}[fragile]
\frametitle{C Code}
\begin{lstlisting}
int areMatricesEqual(double A[2][2], double B[2][2]) {
    for (int i = 0; i < 2; i++) {
        for (int j = 0; j < 2; j++) {
            // Use a small tolerance for floating-point comparison
            if (fabs(A[i][j] - B[i][j]) > 1e-9) {
                return 0; // Not equal
            }
        }
    }
    return 1; // Equal
}
\end{lstlisting}
\end{frame}

\begin{frame}[fragile]
\frametitle{Python and C Code}
\begin{lstlisting}
import ctypes
import math

# Define a 2x2 matrix type using ctypes (array of arrays)
Matrix2x2 = (ctypes.c_double * 2) * 2

def print_matrix(name, matrix):
    print(f"{name} = ")
    for i in range(2):
        print(f"  | {matrix[i][0]:6.2f}  {matrix[i][1]:6.2f} |")
def determinant(matrix):
    return matrix[0][0] * matrix[1][1] - matrix[0][1] * matrix[1][0]
\end{lstlisting}
\end{frame}

\begin{frame}[fragile]
\frametitle{Python and C Code}
\begin{lstlisting}
def transpose(in_matrix, out_matrix):
    out_matrix[0][0] = in_matrix[0][0]
    out_matrix[0][1] = in_matrix[1][0]
    out_matrix[1][0] = in_matrix[0][1]
    out_matrix[1][1] = in_matrix[1][1]

def inverse(in_matrix, out_matrix):
    det = determinant(in_matrix)
    if abs(det) < 1e-9:
        return False
    inv_det = 1.0 / det
\end{lstlisting}
\end{frame}

\begin{frame}[fragile]
\frametitle{Python and C Code}
\begin{lstlisting}
    out_matrix[0][0] =  in_matrix[1][1] * inv_det
    out_matrix[0][1] = -in_matrix[0][1] * inv_det
    out_matrix[1][0] = -in_matrix[1][0] * inv_det
    out_matrix[1][1] =  in_matrix[0][0] * inv_det
    return True
def are_matrices_equal(A, B):
    for i in range(2):
        for j in range(2):
            if abs(A[i][j] - B[i][j]) > 1e-9:
                return False
    return True
\end{lstlisting}
\end{frame}

\begin{frame}[fragile]
\frametitle{Python and C Code}
\begin{lstlisting}
def main():
    # Define matrix Q
    Q = Matrix2x2()
    Q[0][0], Q[0][1] = 1.0, -2.0
    Q[1][0], Q[1][1] = 2.0,  1.0
    print("Given Matrix Q:")
    print_matrix("Q", Q)

    # a) Check if Q == Q^T
    print("a) Checking if Q == Q^T ...")
    Q_T = Matrix2x2()
    transpose(Q, Q_T)
\end{lstlisting}
\end{frame}

\begin{frame}[fragile]
\frametitle{Python and C Code}
\begin{lstlisting}
    print_matrix("Transpose Q^T", Q_T)
    print(f"Result: Statement (a) is {'TRUE' if are_matrices_equal(Q, Q_T) else 'FALSE'}.\n")
    # b) Check if Q == Q^-1
    print("b) Checking if Q == Q^-1 ...")
    Q_inv = Matrix2x2()
    if inverse(Q, Q_inv):
        print_matrix("Inverse Q^-1", Q_inv)
        print(f"Result: Statement (b) is {'TRUE' if are_matrices_equal(Q, Q_inv) else 'FALSE'}.\n")
\end{lstlisting}
\end{frame}

\begin{frame}[fragile]
\frametitle{Python and C Code}
\begin{lstlisting}
    else:
        print("Inverse does not exist.")
        print("Result: Statement (b) is FALSE.\n")
    # c) & d) Check full rank and column dependency using determinant
    print("c/d) Checking rank and column dependency...")
    det = determinant(Q)
    print(f"Determinant of Q = {det:.2f}")
    if abs(det) > 1e-9:
        print("Result: Statement (c) is TRUE (Determinant is non-zero, so Q is of full rank).")
        print("Result: Statement (d) is FALSE (Columns are linearly independent).\n")
\end{lstlisting}
\end{frame}

\begin{frame}[fragile]
\frametitle{Python and C Code}
\begin{lstlisting}
    else:
        print("Result: Statement (c) is FALSE (Determinant is zero, so Q is not of full rank).")
        print("Result: Statement (d) is TRUE (Columns are linearly dependent).\n")
    print("Conclusion: The only TRUE statement is (c).")

if __name__ == "__main__":
    main()
\end{lstlisting}
\end{frame}
\end{document}