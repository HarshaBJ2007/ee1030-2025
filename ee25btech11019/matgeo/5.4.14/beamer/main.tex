\documentclass{beamer}

% Theme
\usetheme{Madrid}
\usecolortheme{default}

% Packages
\usepackage{amsmath,amssymb,amsfonts}
\usepackage{graphicx}
\usepackage{xcolor}
\usepackage{tikz}
\usepackage{tkz-euclide}
\usepackage{array}
\usepackage{multirow}
\usepackage{longtable}
\usepackage{lscape}
\usepackage{listings}
\lstset{
    basicstyle=\ttfamily\small,
    keywordstyle=\color{blue},
    commentstyle=\color{gray},
    stringstyle=\color{red},
    showstringspaces=false,
    breaklines=true
}

% Custom macros
\newcommand{\myvec}[1]{\begin{pmatrix}#1\end{pmatrix}}
\newcommand{\brak}[1]{\left( #1 \right)}
\newcommand{\augvec}[3]{%
  \left(\!\begin{array}{@{}*{#1}{r}|*{#2}{r}@{}}#3\end{array}\!\right)
}




% Redefine \vec to bold letters only (no arrow)
\renewcommand{\vec}[1]{\mathbf{#1}}

\title{5.4.14}
\author{EE25BTECH11019 -- Darji Vivek M.}
\date{}

\begin{document}

\begin{frame}
\begin{titlepage}

\end{titlepage}
\end{frame}

\begin{frame}{Question}
Using elementary transformations, find the inverse of the following matrix. 
\begin{align*}
    \myvec{2&&3\\5&&7}
\end{align*}
\end{frame}

\begin{frame}{Theoretical Solution}
To find the inverse of a matrix using the Gauss–Jordan method:

\begin{equation}
\resizebox{1\textwidth}{!}{$
\augvec{2}{2}{2 & 3 & 1 & 0\\ 5 & 7 & 0 & 1}
\xrightarrow{R_1 \gets R_1/2,\ R_2 \gets R_2 - 5 R_1}
\augvec{2}{2}{1 & \tfrac{3}{2} & \tfrac{1}{2} & 0\\ 0 & -\tfrac{1}{2} & -\tfrac{5}{2} & 1}
\xrightarrow{R_2 \gets -2 R_2,\ R_1 \gets R_1 - \tfrac{3}{2} R_2}
\augvec{2}{2}{1 & 0 & -7 & 3\\ 0 & 1 & 5 & -2}
$}
\tag{1}
\end{equation}

Thus, the inverse is:

\begin{equation}
A^{-1} =
\myvec{-7 & 3\\ 5 & -2}.
\tag{2}
\end{equation}
\end{frame}

\begin{frame}[fragile]{C Code: parallel\_funcs.c}
\begin{lstlisting}
#include <stdio.h>
#define N 2   // matrix size (you can generalize)
void inverse(double A[N][N], double inv[N][N]) {
    // Step 1: Create augmented matrix [A|I]
    double aug[N][2*N];
    for (int i = 0; i < N; i++) {
        for (int j = 0; j < N; j++) {
            aug[i][j] = A[i][j];          // copy A
            aug[i][j+N] = (i == j) ? 1 : 0; // identity
        }
    }
    // Step 2: Gauss-Jordan elimination
    for (int i = 0; i < N; i++) {
        // Make pivot = 1
        double pivot = aug[i][i];
        for (int j = 0; j < 2*N; j++) {
            aug[i][j] /= pivot;
        }

\end{lstlisting}
\end{frame}
\begin{frame}[fragile]{C Code: parallel\_funcs.c}
\begin{lstlisting}
        // Eliminate other rows
        for (int k = 0; k < N; k++) {
            if (k != i) {
                double factor = aug[k][i];
                for (int j = 0; j < 2*N; j++) {
                    aug[k][j] -= factor * aug[i][j];
                }
            }
        }
    }
    // Step 3: Extract inverse from augmented matrix
    for (int i = 0; i < N; i++) {
        for (int j = 0; j < N; j++) {
            inv[i][j] = aug[i][j+N];
        }
    }
}
\end{lstlisting}
\end{frame}
\begin{frame}[fragile]{Python: Plotting Lines}
\begin{lstlisting}[language=Python, basicstyle=\ttfamily\scriptsize, keywordstyle=\color{blue}]
import ctypes
import numpy as np
import sympy as sp

# ====== Load Shared Library (.so) ======
lib = ctypes.CDLL("./libinverse.so")

# Define argument types for C function
lib.inverse.argtypes = [
    ctypes.POINTER((ctypes.c_double * 2) * 2),  # Input matrix A
    ctypes.POINTER((ctypes.c_double * 2) * 2)   # Output inverse matrix
]

# ====== Define the Matrix ======
A = np.array([[2, -6],
              [1, -2]], dtype=np.double)

inv = np.zeros((2, 2), dtype=np.double)

\end{lstlisting}
\end{frame}
\begin{frame}[fragile]{Python: Plotting Lines}
\begin{lstlisting}[language=Python, basicstyle=\ttfamily\scriptsize, keywordstyle=\color{blue}]

# ====== Call the C Function ======
lib.inverse(
    A.ctypes.data_as(ctypes.POINTER((ctypes.c_double * 2) * 2)),
    inv.ctypes.data_as(ctypes.POINTER((ctypes.c_double * 2) * 2))
)

# ====== Convert and Display Results ======
A_sym = sp.Matrix(A)
A_inv_sym = A_sym.inv()           # Sympy inverse (for verification)
A_inv_c = sp.Matrix(inv)          # Inverse from C code

print("Matrix A:")
sp.pprint(A_sym)

print("\nInverse computed in C:")
sp.pprint(A_inv_c)

print("\nInverse computed in Sympy:")
sp.pprint(A_inv_sym)
\end{lstlisting}
\end{frame}
\end{document}
