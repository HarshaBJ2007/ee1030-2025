\let\negmedspace\undefined
\let\negthickspace\undefined
\documentclass[journal]{IEEEtran}
\usepackage[a5paper, margin=10mm, onecolumn]{geometry}
%\usepackage{lmodern} % Ensure lmodern is loaded for pdflatex
\usepackage{tfrupee} % Include tfrupee package

\setlength{\headheight}{1cm} % Set the height of the header box
\setlength{\headsep}{0mm}     % Set the distance between the header box and the top of the text

\usepackage{gvv-book}
\usepackage{gvv}
\usepackage{cite}
\usepackage{amsmath,amssymb,amsfonts,amsthm}
\usepackage{algorithmic}
\usepackage{graphicx}
\usepackage{textcomp}
\usepackage{xcolor}
\usepackage{txfonts}
\usepackage{listings}
\usepackage{enumitem}
\usepackage{mathtools}
\usepackage{gensymb}
\usepackage{tfrupee}
\usepackage[breaklinks=true]{hyperref}
\usepackage{tkz-euclide} 
\usepackage{listings}
% \usepackage{gvv}                                        
\def\inputGnumericTable{}                                 
\usepackage[latin1]{inputenc}                                
\usepackage{color}                                            
\usepackage{array}                                            
\usepackage{longtable}                                       
\usepackage{calc}                                             
\usepackage{multirow}                                         
\usepackage{hhline}                                           
\usepackage{ifthen}                                           
\usepackage{lscape}
\usepackage{circuitikz}
\usepackage{comment}
\tikzstyle{block} = [rectangle, draw, fill=blue!20, 
    text width=4em, text centered, rounded corners, minimum height=3em]
\tikzstyle{sum} = [draw, fill=blue!10, circle, minimum size=1cm, node distance=1.5cm]
\tikzstyle{input} = [coordinate]
\tikzstyle{output} = [coordinate]


\begin{document}

\bibliographystyle{IEEEtran}
\vspace{3cm}

\title{5.13.44}
\author{EE25BTECH11018-Darisy Sreetej}
 \maketitle
% \newpage
% \bigskip
{\let\newpage\relax\maketitle}

\renewcommand{\thefigure}{\theenumi}
\renewcommand{\thetable}{\theenumi}
\setlength{\intextsep}{10pt} % Space between text and floats


\numberwithin{equation}{enumi}
\numberwithin{figure}{enumi}
\renewcommand{\thetable}{\theenumi}

\textbf{Question}:
Let  

$$
\vec{\vec{P_1}} = \vec{I} = 
\myvec{1 & 0 & 0 \\ 0 & 1 & 0 \\ 0 & 0 & 1}, \quad
\vec{\vec{P_2}} = 
\myvec{1 & 0 & 0 \\ 0 & 0 & 1 \\ 0 & 1 & 0}, \quad
\vec{\vec{P_3}} = 
\myvec{0 & 1 & 0 \\ 1 & 0 & 0 \\ 0 & 0 & 1},
$$

$$
\vec{\vec{P_4}} = 
\myvec{0 & 1 & 0 \\ 0 & 0 & 1 \\ 1 & 0 & 0}, \quad
\vec{\vec{P_5}} = 
\myvec{0 & 0 & 1 \\ 1 & 0 & 0 \\ 0 & 1 & 0}, \quad
\vec{\vec{P_6}} = 
\myvec{0 & 0 & 1 \\ 0 & 1 & 0 \\ 1 & 0 & 0}.
$$
$$
\quad \text{and} \quad
\vec{X} = \sum_{k=1}^{6} \vec{P_k} \myvec{2 & 1 & 3 \\ 1 & 0 & 2 \\ 3 & 2 & 1} \vec{P_k}^\top.
$$

Where $\vec{P_k}^\top $ denotes the transpose of matrix $ \vec{P_k} $.Then which of the following options is/are correct? 
\begin{enumerate}
    \item $\vec{X}$ is a symmetric matrix
    \item The sum of diagonal elements of $\vec{X}$ is 18
    \item $\vec{X} - 30\vec{I}$ is an invertible matrix
    \item If $\vec{X}\myvec{1\\1\\1}=\alpha\myvec{1\\1\\1}$, then $\alpha$ is 30
\end{enumerate}
\textbf{Solution}:\\
From the question,
$
\vec{\vec{P_1}}^\top = \vec{\vec{P_1}}, \quad \vec{\vec{P_2}}^\top = \vec{\vec{P_2}}, \quad \vec{\vec{P_3}}^\top = \vec{\vec{P_3}}, \quad \vec{\vec{P_4}}^\top = \vec{\vec{P_5}}, \quad \vec{\vec{P_5}}^\top = \vec{\vec{P_4}}, \quad \vec{\vec{P_6}}^\top = \vec{\vec{P_6}}$
and Let 
\begin{align}
\vec{Q} = \myvec{
2 & 1 & 3 \\
1 & 0 & 2 \\
3 & 2 & 1 }
\end{align}
and $ \vec{Q}^\top = \vec{Q}$

Now,

\begin{align}
\vec{X} = (\vec{P_1} Q \vec{P_1}^\top) + (\vec{P_2} Q \vec{P_2}^\top) + (\vec{P_3} Q \vec{P_3}^\top) + (\vec{P_4} Q \vec{P_4}^\top) + (\vec{P_5} Q \vec{P_5}^\top) + (\vec{P_6} Q \vec{P_6}^\top)
\end{align}

So,
\begin{align}
\vec{X}^\top = (\vec{P_1} Q \vec{P_1}^\top)^\top + (\vec{P_2} Q \vec{P_2}^\top)^\top + (\vec{P_3} Q \vec{P_3}^\top)^\top + (\vec{P_4} Q \vec{P_4}^\top)^\top + (\vec{P_5} Q \vec{P_5}^\top)^\top + (\vec{P_6} Q \vec{P_6}^\top)^\top
\end{align}
\begin{align}
= \vec{P_1} Q \vec{P_1}^\top + \vec{P_2} Q \vec{P_2}^\top + \vec{P_3} Q \vec{P_3}^\top + \vec{P_4} Q \vec{P_4}^\top + \vec{P_5} Q \vec{P_5}^\top + \vec{P_6} Q \vec{P_6}^\top
\end{align}
\begin{align}
\Rightarrow \vec{X}^\top = \vec{X}
\end{align}
$$
\Rightarrow \vec{X} \text{ is a symmetric matrix.}
$$

The sum of diagonal entries of $\vec{X} = \operatorname{Tr}(\vec{X})$:
\begin{align}
\operatorname{Tr}(\vec{X}) = \sum_{i=1}^6 \operatorname{Tr}(\vec{P_i} \vec{Q} \vec{P_i}^\top) = \sum_{i=1}^6 \operatorname{Tr}(\vec{QP_i^\top P_i})
\end{align}
$$
(\because \operatorname{Tr}(ABC) = \operatorname{Tr}(BCA))
$$
\begin{align}
= \sum_{i=1}^6 \operatorname{Tr}(\vec{Q I})
\end{align}
$$
(\because \vec{P_i}'s \quad  \text{are orthogonal matrices})
$$
\begin{align}
= \sum_{i=1}^6 \operatorname{Tr}(\vec{Q})
\end{align}
\begin{align}
= 6\operatorname{Tr}(\vec{Q})\\
= 6\times 3 \\
=18 \quad \
\end{align}
Now , let
\begin{align}
\vec{R} = \myvec{1\\1\\1}, \text{ then}
\end{align}
\begin{align}
\vec{XR} = \sum_{k=1}^{6} \vec{P_k Q P_k}^\top \vec{R} = \sum_{k=1}^{6} \vec{P_k Q P_k}^\top \vec{R}
\end{align}
\begin{align}
= \sum_{k=1}^{6} \vec{P_k} ( \vec{Q R}) \qquad [\because \vec{P_k}^\top \vec{R} = \vec{R}]
\end{align}
\begin{align}
= \sum_{k=1}^{6} \vec{P_k} \myvec{6\\3\\6}\\
= \sum_{k=1}^{6} \vec{P_k} \myvec{6\\3\\6}\\
= \myvec{2 & 2 & 2\\2 & 2 & 2\\2 & 2 & 2}\myvec{6\\3\\6}
\end{align}
\begin{align}
\implies \vec{XR} = 
\myvec{30\\30\\30}\\
\implies \vec{XR} = 30\vec{R}\\
\implies \vec{X}\myvec{1\\1\\1} = 30\myvec{1\\1\\1}
\end{align}
Thus , the value of $ \alpha = 30$.\\
From (4.19),
\begin{align}
 (\vec{X} - 30\vec{I})\vec{R} = 0 \implies \mydet{\vec{X}-30\vec{I}} = 0 \\
\text{So, } (\vec{X} - 30\vec{I}) \text{ is not invertible }
\end{align}
\text{Hence, options (a), (b) and (d) are correct.}

\end{document}
