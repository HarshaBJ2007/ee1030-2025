\documentclass{beamer}
\usepackage[utf8]{inputenc}
\usetheme{Madrid}
\usecolortheme{default}
\usepackage{amsmath,amssymb,amsfonts,amsthm}
\usepackage{mathtools}
\usepackage{txfonts}
\usepackage{tkz-euclide}
\usepackage{listings}
\usepackage{adjustbox}
\usepackage{array}
\usepackage{tabularx}
\usepackage{gvv}
\usepackage{lmodern}
\usepackage{circuitikz}
\usepackage{tikz}
\usepackage{graphicx}
\setbeamertemplate{page number in head/foot}[totalframenumber]
\usepackage[T1]{fontenc}
\usepackage{tcolorbox}
\tcbuselibrary{minted,breakable,xparse,skins}

\definecolor{bg}{gray}{0.95}
\DeclareTCBListing{mintedbox}{O{}m!O{}}{%
  breakable=true,
  listing engine=minted,
  listing only,
  minted language=#2,
  minted style=default,
  minted options={%
    linenos,
    gobble=0,
    breaklines=true,
    breakafter=,,
    fontsize=\small,
    numbersep=8pt,
    #1},
  boxsep=0pt,
  left skip=0pt,
  right skip=0pt,
  left=25pt,
  right=0pt,
  top=3pt,
  bottom=3pt,
  arc=5pt,
  leftrule=0pt,
  rightrule=0pt,
  bottomrule=2pt,
  toprule=2pt,
  colback=bg,
  colframe=orange!70,
  enhanced,
  overlay={%
    \begin{tcbclipinterior}
    \fill[orange!20!white] (frame.south west) rectangle ([xshift=20pt]frame.north west);
    \end{tcbclipinterior}},
  #3,
}

% Code style
\lstset{
    language=C,
    basicstyle=\ttfamily\small,
    keywordstyle=\color{blue},
    stringstyle=\color{orange},
    commentstyle=\color{green!60!black},
    numbers=left,
    numberstyle=\tiny\color{gray},
    breaklines=true,
    showstringspaces=false,
}

% Title info
\title %optional
{5.13.44}
\date{October 10, 2025}
\author % (optional)
{EE25BTECH11018 - Darisy Sreetej}

\begin{document}

\frame{\titlepage}

\begin{frame}{Question}
Let  

$$
\vec{\vec{P_1}} = \vec{I} = 
\myvec{1 & 0 & 0 \\ 0 & 1 & 0 \\ 0 & 0 & 1}, \quad
\vec{\vec{P_2}} = 
\myvec{1 & 0 & 0 \\ 0 & 0 & 1 \\ 0 & 1 & 0}, \quad
\vec{\vec{P_3}} = 
\myvec{0 & 1 & 0 \\ 1 & 0 & 0 \\ 0 & 0 & 1},
$$

$$
\vec{\vec{P_4}} = 
\myvec{0 & 1 & 0 \\ 0 & 0 & 1 \\ 1 & 0 & 0}, \quad
\vec{\vec{P_5}} = 
\myvec{0 & 0 & 1 \\ 1 & 0 & 0 \\ 0 & 1 & 0}, \quad
\vec{\vec{P_6}} = 
\myvec{0 & 0 & 1 \\ 0 & 1 & 0 \\ 1 & 0 & 0}.
$$
$$
\quad \text{and} \quad
\vec{X} = \sum_{k=1}^{6} \vec{P_k} \myvec{2 & 1 & 3 \\ 1 & 0 & 2 \\ 3 & 2 & 1} \vec{P_k}^\top.
$$

Where $\vec{P_k}^\top $ denotes the transpose of matrix $ \vec{P_k} $.Then which of the following options is/are correct? 
\end{frame}
\begin{frame}{Question}
    \begin{enumerate}
    \item $\vec{X}$ is a symmetric matrix
    \item The sum of diagonal elements of $\vec{X}$ is 18
    \item $\vec{X} - 30\vec{I}$ is an invertible matrix
    \item If $\vec{X}\myvec{1\\1\\1}=\alpha\myvec{1\\1\\1}$, then $\alpha$ is 30
\end{enumerate}
\end{frame}
\begin{frame}{Solution}
From the question,
$
\vec{\vec{P_1}}^\top = \vec{\vec{P_1}}, \quad \vec{\vec{P_2}}^\top = \vec{\vec{P_2}}, \quad \vec{\vec{P_3}}^\top = \vec{\vec{P_3}}, \quad \vec{\vec{P_4}}^\top = \vec{\vec{P_5}}, \quad \vec{\vec{P_5}}^\top = \vec{\vec{P_4}}, \quad \vec{\vec{P_6}}^\top = \vec{\vec{P_6}}$
and Let 
\begin{align}
\vec{Q} = \myvec{
2 & 1 & 3 \\
1 & 0 & 2 \\
3 & 2 & 1 }
\end{align}
and $ \vec{Q}^\top = \vec{Q}$

Now,

\begin{align}
\vec{X} = (\vec{P_1} Q \vec{P_1}^\top) + (\vec{P_2} Q \vec{P_2}^\top) + (\vec{P_3} Q \vec{P_3}^\top) + (\vec{P_4} Q \vec{P_4}^\top) + (\vec{P_5} Q \vec{P_5}^\top) + 
\end{align}
$$
(\vec{P_6} Q \vec{P_6}^\top)
$$
\end{frame}
\begin{frame}
So,
\begin{align}
\vec{X}^\top = (\vec{P_1} Q \vec{P_1}^\top)^\top + (\vec{P_2} Q \vec{P_2}^\top)^\top + (\vec{P_3} Q \vec{P_3}^\top)^\top + (\vec{P_4} Q \vec{P_4}^\top)^\top + 
\end{align}
$(\vec{P_5} Q \vec{P_5}^\top)^\top + (\vec{P_6} Q \vec{P_6}^\top)^\top$
\begin{align}
= \vec{P_1} Q \vec{P_1}^\top + \vec{P_2} Q \vec{P_2}^\top + \vec{P_3} Q \vec{P_3}^\top + \vec{P_4} Q \vec{P_4}^\top + \vec{P_5} Q \vec{P_5}^\top + \vec{P_6} Q \vec{P_6}^\top
\end{align}
\begin{align}
\Rightarrow \vec{X}^\top = \vec{X}
\end{align}
$$
\Rightarrow \vec{X} \text{ is a symmetric matrix.}
$$
The sum of diagonal entries of $\vec{X} = \operatorname{Tr}(\vec{X})$:
\begin{align}
\operatorname{Tr}(\vec{X}) = \sum_{i=1}^6 \operatorname{Tr}(\vec{P_i} \vec{Q} \vec{P_i}^\top) = \sum_{i=1}^6 \operatorname{Tr}(\vec{QP_i^\top P_i})
\end{align}
\end{frame}
\begin{frame}
   $$
(\because \operatorname{Tr}(ABC) = \operatorname{Tr}(BCA))
$$
\begin{align}
= \sum_{i=1}^6 \operatorname{Tr}(\vec{Q I})
\end{align}
$$
(\because \vec{P_i}'s \quad  \text{are orthogonal matrices})
$$
\begin{align}
= \sum_{i=1}^6 \operatorname{Tr}(\vec{Q})
\end{align}
\begin{align}
= 6\operatorname{Tr}(\vec{Q})\\
= 6\times 3 \\
=18 \quad \
\end{align}
\end{frame}
\begin{frame}
  Now , let
\begin{align}
\vec{R} = \myvec{1\\1\\1}, \text{ then}
\end{align}
\begin{align}
\vec{XR} = \sum_{k=1}^{6} \vec{P_k Q P_k}^\top \vec{R} = \sum_{k=1}^{6} \vec{P_k Q P_k}^\top \vec{R}
\end{align}
\begin{align}
= \sum_{k=1}^{6} \vec{P_k} ( \vec{Q R}) \qquad [\because \vec{P_k}^\top \vec{R} = \vec{R}]
\end{align}
\begin{align}
= \sum_{k=1}^{6} \vec{P_k} \myvec{6\\3\\6}\\
= \sum_{k=1}^{6} \vec{P_k} \myvec{6\\3\\6}\\
\end{align}
\end{frame}
\begin{frame}
\begin{align}
    = \myvec{2 & 2 & 2\\2 & 2 & 2\\2 & 2 & 2}\myvec{6\\3\\6}
\end{align}
\begin{align}
\implies \vec{XR} = 
\myvec{30\\30\\30}\\
\implies \vec{XR} = 30\vec{R}\\
\implies \vec{X}\myvec{1\\1\\1} = 30\myvec{1\\1\\1}
\end{align}
Thus , the value of $ \alpha = 30$.\\
From (4.19),
\begin{align}
 (\vec{X} - 30\vec{I})\vec{R} = 0 \implies \mydet{\vec{X}-30\vec{I}} = 0 \\
\text{So, } (\vec{X} - 30\vec{I}) \text{ is not invertible }
\end{align}
\end{frame}
\begin{frame}{Conclusion}
    \text{Hence, options (a), (b) and (d) are correct.}
\end{frame}
\begin{frame}[fragile]
\frametitle{C code}
    \begin{lstlisting}[language=C]
#include <stdio.h>
#include <math.h>
#define N 3

void compute_X(double X[N][N]) {
    for (int i=0; i<N; ++i)
        for (int j=0; j<N; ++j)
            X[i][j] = (i==j) ? 6.0 : 12.0;
}

double trace(double X[N][N]) {
    double t = 0.0;
    for (int i=0; i<N; ++i)
        t += X[i][i];
    return t;
}
\end{lstlisting}
\end{frame}
\begin{frame}[fragile]
    \frametitle{C Code }
    \begin{lstlisting}[language=C]
int is_symmetric(double X[N][N], double tol) {
    for (int i=0; i<N; ++i)
        for (int j=i+1; j<N; ++j)
            if (fabs(X[i][j]-X[j][i])>tol)
                return 0;
    return 1;
}

void mat_vec_mul(double X[N][N], double v[N], double y[N]) {
    for (int i=0; i<N; ++i){
        y[i]=0;
        for (int j=0; j<N; ++j)
            y[i]+=X[i][j]*v[j];
    }
}
     \end{lstlisting}
\end{frame}
\begin{frame}[fragile]
    \frametitle{C Code }
    \begin{lstlisting}[language=C]
void subtract_scalar_I(double X[N][N], double scalar, double Y[N][N]) {
    for (int i=0;i<N;++i)
        for (int j=0;j<N;++j)
            Y[i][j] = X[i][j] - (i==j ? scalar : 0);
}

double determinant(double X[N][N]) {
    return X[0][0]*(X[1][1]*X[2][2]-X[1][2]*X[2][1])
         - X[0][1]*(X[1][0]*X[2][2]-X[1][2]*X[2][0])
         + X[0][2]*(X[1][0]*X[2][1]-X[1][1]*X[2][0]);
}    
     \end{lstlisting}
\end{frame}
\begin{frame}[fragile]
    \frametitle{Python + C code}

    \begin{lstlisting}[language=Python]
import ctypes
import numpy as np

# Load shared library
lib = ctypes.CDLL("./libmatrix_X.so")

# Define types
N = 3
DoubleArray3 = ctypes.c_double * N
DoubleMatrix3 = (DoubleArray3 * N)

# Function signatures
lib.compute_X.argtypes = [DoubleMatrix3]
lib.trace.argtypes = [DoubleMatrix3]
lib.trace.restype = ctypes.c_double
lib.is_symmetric.argtypes = [DoubleMatrix3, ctypes.c_double]
lib.is_symmetric.restype = ctypes.c_int
lib.mat_vec_mul.argtypes = [DoubleMatrix3, DoubleArray3, DoubleArray3]
\end{lstlisting}
\end{frame}
\begin{frame}[fragile]
    \frametitle{Python + C code}

    \begin{lstlisting}[language=Python]
lib.subtract_scalar_I.argtypes = [DoubleMatrix3, ctypes.c_double, DoubleMatrix3]
lib.determinant.argtypes = [DoubleMatrix3]
lib.determinant.restype = ctypes.c_double

# Initialize matrices
X = DoubleMatrix3()
lib.compute_X(X)

# Convert to numpy for easier viewing
X_np = np.array([[X[i][j] for j in range(N)] for i in range(N)])
print("Matrix X =\n", X_np)

# Compute trace
trace_val = lib.trace(X)
print("\nTrace(X) =", trace_val)
\end{lstlisting}
\end{frame}
\begin{frame}[fragile]
\frametitle{Python + C code}
    \begin{lstlisting}[language=Python]
    # Check symmetry
sym = lib.is_symmetric(X, 1e-9)
print("Symmetric:", "Yes" if sym else "No")

# Compute α for X*[1 1 1]^T
v = DoubleArray3(1.0, 1.0, 1.0)
y = DoubleArray3()
lib.mat_vec_mul(X, v, y)
y_vals = [y[i] for i in range(N)]
alpha = y_vals[0]
print("\nX*[1 1 1]^T =", y_vals)
print("Alpha =", alpha)

# Compute determinant of (X - 30I)
Y = DoubleMatrix3()
lib.subtract_scalar_I(X, 30.0, Y)
det_val = lib.determinant(Y)
    \end{lstlisting}
\end{frame}
\begin{frame}[fragile]
\frametitle{Python + C code}
    \begin{lstlisting}[language=Python]
print("\nDeterminant of (X - 30I):", det_val)
if abs(det_val) < 1e-9:
    print("=> (X - 30I) is NOT invertible (singular)")
else:
    print("=> (X - 30I) is invertible")
     \end{lstlisting}
\end{frame}
\end{document}
