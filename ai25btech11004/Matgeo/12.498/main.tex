\let\negmedspace\undefined
\let\negthickspace\undefined
\documentclass[journal]{IEEEtran}
\usepackage[a5paper, margin=10mm, onecolumn]{geometry}
%\usepackage{lmodern} % Ensure lmodern is loaded for pdflatex
\usepackage{tfrupee} % Include tfrupee package

\setlength{\headheight}{1cm} % Set the height of the header box
\setlength{\headsep}{0mm}     % Set the distance between the header box and the top of the text

\usepackage{gvv-book}
\usepackage{gvv}
\usepackage{cite}
\usepackage{amsmath,amssymb,amsfonts,amsthm}
\usepackage{algorithmic}
\usepackage{graphicx}
\usepackage{textcomp}
\usepackage{xcolor}
\usepackage{txfonts}
\usepackage{listings}
\usepackage{enumitem}
\usepackage{mathtools}
\usepackage{gensymb}
\usepackage{comment}
\usepackage[breaklinks=true]{hyperref}
\usepackage{tkz-euclide} 
\usepackage{listings}
% \usepackage{gvv}                                        
\def\inputGnumericTable{}                                 
\usepackage[latin1]{inputenc}                                
\usepackage{color}                                            
\usepackage{array}                                            
\usepackage{longtable}                                       
\usepackage{calc}                                             
\usepackage{multirow}                                         
\usepackage{hhline}                                           
\usepackage{ifthen}                                           
\usepackage{lscape}
\usepackage{tikz}
\usetikzlibrary{patterns}
\begin{document}


\bibliographystyle{IEEEtran}
\vspace{3cm}


\numberwithin{equation}{enumi}
\numberwithin{figure}{enumi}
\renewcommand{\thetable}{\theenumi}


% Marks the beginning of the document

\bibliographystyle{IEEEtran}
\vspace{3cm}


\title{12.498}
\author{AI25BTECH11004-B.JASWANTH}
% \maketitle
% \newpage
% \bigskip
{\let\newpage\relax\maketitle}


\renewcommand{\thefigure}{\theenumi}
\renewcommand{\thetable}{\theenumi}
\setlength{\intextsep}{10pt} % Space between text and floats

\textbf{Question}\\
If $\Vec{A}=\myvec{1 & 5 \\
        6 & 2} \text{and}$  $\Vec{B}=\myvec{3 & 7 \\
            8 & 4}$,then $\vec{AB^\top}$ is equal to
\begin{enumerate}[label=(\alph*)]
\begin{multicols}{4}
\item  \myvec{38 & 28 \\ 32 & 56}
\item  \myvec{3 & 40 \\ 42 & 8}
\item  \myvec{43 & 27 \\ 34 & 50}
\item  \myvec{38 & 32 \\ 28 & 56}
 \end{multicols}
\end{enumerate}            

\textbf{Solution}:\\
Given,
\begin{align}
\vec{A} = \myvec{1 & 5 \\ 6 & 2}
\end{align}
\begin{align}
\vec{B} = \myvec{3 & 7 \\ 8 & 4}
\end{align}

\begin{align}
\vec{B^T} = \myvec{3 & 8 \\ 7 & 4}
\end{align}

\begin{align}
\vec{AB^T} = \myvec{1 & 5 \\ 6 & 2}\myvec{3 & 8 \\ 7 & 4}
= \myvec{38 & 28 \\ 32 & 56}\\
\implies \vec{AB^T}=\begin{pmatrix}
    38 & 28\\
    32 & 56
\end{pmatrix} 
\end{align}



\end{document}