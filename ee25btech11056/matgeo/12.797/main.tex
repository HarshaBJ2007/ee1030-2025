\let\negmedspace\undefined
\let\negthickspace\undefined
\documentclass[journal,12pt,onecolumn]{IEEEtran}
\usepackage{cite}
\usepackage{amsmath,amssymb,amsfonts,amsthm}
\usepackage{algorithmic}
\usepackage{graphicx}
\graphicspath{{./figs/}}
\usepackage{textcomp}
\usepackage{xcolor}
\usepackage{txfonts}
\usepackage{listings}
\usepackage{enumitem}
\usepackage{mathtools}
\usepackage{gensymb}
\usepackage{comment}
\usepackage{caption}
\usepackage[breaklinks=true]{hyperref}
\usepackage{tkz-euclide} 
\usepackage{listings}
\usepackage{gvv}                                        
%\def\inputGnumericTable{}                                 
\usepackage[latin1]{inputenc}     
\usepackage{xparse}
\usepackage{color}                                            
\usepackage{array}
\usepackage{longtable}                                       
\usepackage{calc}                                             
\usepackage{multirow}
\usepackage{multicol}
\usepackage{hhline}                                           
\usepackage{ifthen}                                           
\usepackage{lscape}
\usepackage{tabularx}
\usepackage{array}
\usepackage{float}
\newtheorem{theorem}{Theorem}[section]
\newtheorem{problem}{Problem}
\newtheorem{proposition}{Proposition}[section]
\newtheorem{lemma}{Lemma}[section]
\newtheorem{corollary}[theorem]{Corollary}
\newtheorem{example}{Example}[section]
\newtheorem{definition}[problem]{Definition}
\newcommand{\BEQA}{\begin{eqnarray}}
\newcommand{\EEQA}{\end{eqnarray}}
\newcommand{\define}{\stackrel{\triangle}{=}}
\theoremstyle{remark}
\newtheorem{rem}{Remark}

\begin{document}

\title{12.797}
\author{ee25btech11056 - Suraj.N}
\maketitle
\renewcommand{\thefigure}{\theenumi}
\renewcommand{\thetable}{\theenumi}

\begin{document}

\textbf{Question :} Let $\vec{A}$ be an $n \times n$ real matrix. Consider the following statements:

\begin{enumerate}

\item If $\vec{A}$ is symmetric, then there exists $c \geq 0$ such that $\vec{A} + c\vec{I}_n$ is symmetric and positive definite, where $\vec{I}_n$ is the $n \times n$ identity matrix.
\item If $\vec{A}$ is symmetric and positive definite, then there exists a symmetric and positive definite $\vec{B}$ such that $\vec{A} = \vec{B}^2$.

Which of the above statements is/are true?

\begin{enumerate}
\begin{multicols}{4}
\item Only (I) \\
\item Only (II) \\
\item Both (I) and (II) \\
\item Neither (I) nor (II)
\end{multicols}
\end{enumerate}

\end{enumerate}

\textbf{Solution :}

\begin{table}[h!]
  \centering
  \begin{tabular}{|c|c|}
\hline
\textbf{Variable} & \textbf{Value} \\
\hline
$A$ & $(0,-\frac{3}{2})$ \\
\hline
$m$ & $\frac{1}{2}$ \\
\hline
\end{tabular}
  \caption*{Table : Matrix}
  \label{12.797}
\end{table}

Checking statement (\textbf{I}) \\
If $\vec{A}$ is symmetric, its eigenvalues are real. Let the minimum eigenvalue of $\vec{A}$ be $\lambda_{\min}$.Then choose $c > -\lambda_{\min}$.  

The Eigen values of $\vec{A}$ are given as :

\begin{align}
  \mydet{\vec{A}-\lambda_i\vec{I}} = 0
\end{align}

The Eigen values of $\vec{A} + c\vec{I}_n$ are given as :

\begin{align}
\mydet{\vec{A}-(\lambda_k-c)\vec{I}} = 0\\
\lambda_k = \lambda_i + c\\
\lambda_i + c > 0
\end{align}

Since $\lambda_i + c > 0$ for all $i$, $\vec{A} + c\vec{I}_n$ is positive definite and symmetric.  
Hence, statement (\textbf{I}) is \textbf{true}.

Checking statement (\textbf{II}) \\
If $\vec{A}$ is symmetric and positive definite, then it can be diagonalized as:

\begin{align}
\vec{A} = \vec{P}\vec{D}\vec{P}^\top
\end{align}

where $\vec{P}$ is orthogonal and $\vec{D}$ is a diagonal matrix with positive entries (since $\vec{A}$ is positive definite).  
Define

\begin{align}
\vec{B} = \vec{P}\vec{D}^{1/2}\vec{P}^\top
\end{align}

Then,

\begin{align}
\vec{B}^2 = \vec{P}\vec{D}^{1/2}\vec{P}^\top \vec{P}\vec{D}^{1/2}\vec{P}^\top = \vec{P}\vec{D}\vec{P}^\top = \vec{A}
\end{align}

Hence, $\vec{B}$ is symmetric and positive definite.  
Therefore, statement (\textbf{II}) is also \textbf{true}.

\textbf{Final Answer:} (c) Both (I) and (II)

\pagebreak

\textbf{Examples}\\

\textbf{Example a for (I)}

\begin{align}
\vec{A} &= \myvec{0 & 1\\ 1 & 0}
\end{align}

To find eigenvalues, evaluate:

\begin{align}
\mydet{\vec{A}-\lambda \vec{I}}=0\\
\mydet{-\lambda & 1\\1 & -\lambda} &= 0\\
\mydet{-\lambda & 1\\1 & -\lambda} 
\xleftrightarrow{R_2 \to R_2 + \tfrac{1}{\lambda}R_1}
\mydet{-\lambda & 1\\0 & \tfrac{1-\lambda^2}{\lambda}} &= 0\\
\lambda^2 - 1 &= 0\\
\lambda = \pm1\\
\lambda_1=1, \lambda_2=-1
\end{align}

The minimum eigenvalue $\lambda_{\min}=-1$. Choose $c=2$. Then:

\begin{align}
\lambda_1+c=3,\quad \lambda_2+c=1
\end{align}

All eigenvalues are positive, so $\vec{A}+2\vec{I}$ is symmetric positive definite.

\textbf{Example b for (I)}

\begin{align}
\vec{A} &= \myvec{-2 & 0 & 0\\ 0 & 1 & 0\\ 0 & 0 & 3}
\end{align}

Compute eigenvalues from $\mydet{\vec{A}-\lambda\vec{I}}=0$:

\begin{align}
\mydet{-2-\lambda & 0 & 0\\0 & 1-\lambda & 0\\0 & 0 & 3-\lambda} &= 0
\end{align}

\begin{align}
(-2-\lambda)(1-\lambda)(3-\lambda)&=0
\end{align}

Hence
\begin{align}
\lambda_1=-2,\quad \lambda_2=1,\quad \lambda_3=3
\end{align}

Choose $c=3$ so $c>-\lambda_{\min}=2$.  
Then $\lambda_i+c>0$ for all $i$.  
Thus $\vec{A}+3\vec{I}$ is symmetric positive definite.

\pagebreak

\textbf{Example a for (II)}

\begin{align}
\vec{A} &= \myvec{5 & 2\\2 & 5}
\end{align}

Find eigenvalues from $\mydet{\vec{A}-\lambda \vec{I}}=0$:

\begin{align}
\mydet{5-\lambda & 2\\2 & 5-\lambda} &= 0
\end{align}

\begin{align}
\mydet{5-\lambda & 2\\2 & 5-\lambda} 
\xleftrightarrow{R_2 \to R_2 - \tfrac{2}{5-\lambda}R_1}
\mydet{5-\lambda & 2\\0 & (5-\lambda)-\tfrac{4}{5-\lambda}} &= 0
\end{align}

\begin{align}
(5-\lambda)^2 - 4 &= 0\\
\lambda^2 - 10\lambda + 21 &= 0\\
\lambda = 5 \pm 2\\
\lambda_1 = 7,\ \lambda_2 = 3 
\end{align}

For eigenvectors:
\begin{align}
(\vec{A}-7\vec{I})\vec{v}=0,\quad(\vec{A}-3\vec{I})\vec{v}=0
\end{align}

They correspond to $\vec{v}_1=\myvec{1\\1}$ and $\vec{v}_2=\myvec{1\\-1}$.  
Form $\vec{P}$ and $\vec{D}$:

\begin{align}
\vec{P} &= \frac{1}{\sqrt{2}}\myvec{1 & 1\\1 & -1},\quad
\vec{D} = \myvec{7 & 0\\0 & 3}
\end{align}

Then
\begin{align}
\vec{B} = \vec{P}\vec{D}^{1/2}\vec{P}^\top,\quad
\vec{B}^2 = \vec{A}
\end{align}

Hence verified for $2\times2$.

\textbf{Example b for (II)}

\begin{align}
\vec{A} &= \myvec{2 & 0 & 0\\0 & 3 & 0\\0 & 0 & 4}
\end{align}

Compute $\mydet{\vec{A}-\lambda\vec{I}}=0$:

\begin{align}
\mydet{2-\lambda & 0 & 0\\0 & 3-\lambda & 0\\0 & 0 & 4-\lambda} &= 0
\end{align}

\begin{align}
(2-\lambda)(3-\lambda)(4-\lambda) = 0\\
\lambda_1=2,\ \lambda_2=3,\ \lambda_3=4
\end{align}

\pagebreak

Diagonalization:

\begin{align}
\vec{A} = \vec{P}\vec{D}\vec{P}^\top,\quad \vec{P}=\vec{I},\quad
\vec{D}=\myvec{2 & 0 & 0\\0 & 3 & 0\\0 & 0 & 4}
\end{align}

Define
\begin{align}
\vec{B} = \vec{P}\vec{D}^{1/2}\vec{P}^\top
= \myvec{\sqrt{2} & 0 & 0\\0 & \sqrt{3} & 0\\0 & 0 & 2}
\end{align}

Then
\begin{align}
\vec{B}^2 = \vec{A}
\end{align}

Hence $\vec{B}$ is symmetric positive definite.\\

\textbf{Conclusion:}  
In all four examples, both statements (I) and (II) hold true.


\end{document}
