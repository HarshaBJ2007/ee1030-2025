\documentclass[journal,12pt,twocolumn]{IEEEtran}
\usepackage{caption}
\captionsetup{font=small,labelfont=bf}
\usepackage{setspace}
\usepackage{gensymb}
\singlespacing
\usepackage[cmex10]{amsmath}
\usepackage{amsthm}
\usepackage{mathrsfs}
\usepackage{txfonts}
\usepackage{stfloats}
\usepackage{graphicx}
\usepackage{float}   
\usepackage{bm}
\usepackage{cite}
\usepackage{cases}
\usepackage{subfig}
\usepackage{longtable}
\usepackage{multirow}
\usepackage{enumitem}
\usepackage{mathtools}
\usepackage{tikz}
\usepackage{circuitikz}
\usepackage{verbatim}
\usepackage[breaklinks=true]{hyperref}
\usepackage{tkz-euclide} % loads  TikZ and tkz-base
\usepackage{listings}
\usepackage{color}    
\usepackage{array}    
\usepackage{longtable}
\usepackage{calc}     
\usepackage{multirow} 
\usepackage{hhline}   
\usepackage{ifthen}   
\usepackage{lscape}     
\usepackage{chngcntr}
\DeclareMathOperator*{\Res}{Res}
\renewcommand\thesection{\arabic{section}}
\renewcommand\thesubsection{\thesection.\arabic{subsection}}
\renewcommand\thesubsubsection{\thesubsection.\arabic{subsubsection}}

\renewcommand\thesectiondis{\arabic{section}}
\renewcommand\thesubsectiondis{\thesectiondis.\arabic{subsection}}
\renewcommand\thesubsubsectiondis{\thesubsectiondis.\arabic{subsubsection}}
\renewcommand\thetable{\arabic{table}}
% correct bad hyphenation here
\hyphenation{op-tical net-works semi-conduc-tor}
\def\inputGnumericTable{}                                 %%

\lstset{
%language=C,
frame=single, 
breaklines=true,
columns=fullflexible
}
%\lstset{
%language=tex,
%frame=single, 
%breaklines=true
%}

\begin{document}
\newtheorem{theorem}{Theorem}[section]
\newtheorem{problem}{Problem}
\newtheorem{proposition}{Proposition}[section]
\newtheorem{lemma}{Lemma}[section]
\newtheorem{corollary}[theorem]{Corollary}
\newtheorem{example}{Example}[section]
\newtheorem{definition}[problem]{Definition}
\newcommand{\BEQA}{\begin{eqnarray}}
\newcommand{\EEQA}{\end{eqnarray}}
\newcommand{\define}{\stackrel{\triangle}{=}}
\bibliographystyle{IEEEtran}
\providecommand{\mbf}{\mathbf}
\providecommand{\pr}[1]{\ensuremath{\Pr\left(#1\right)}}
\providecommand{\qfunc}[1]{\ensuremath{Q\left(#1\right)}}
\providecommand{\sbrak}[1]{\ensuremath{{}\left[#1\right]}}
\providecommand{\lsbrak}[1]{\ensuremath{{}\left[#1\right.}}
\providecommand{\rsbrak}[1]{\ensuremath{{}\left.#1\right]}}
\providecommand{\brak}[1]{\ensuremath{\left(#1\right)}}
\providecommand{\lbrak}[1]{\ensuremath{\left(#1\right.}}
\providecommand{\rbrak}[1]{\ensuremath{\left.#1\right)}}
\providecommand{\cbrak}[1]{\ensuremath{\left\{#1\right\}}}
\providecommand{\lcbrak}[1]{\ensuremath{\left\{#1\right.}}
\providecommand{\rcbrak}[1]{\ensuremath{\left.#1\right\}}}
\theoremstyle{remark}
\newtheorem{rem}{Remark}
\newcommand{\sgn}{\mathop{\mathrm{sgn}}}
\providecommand{\abs}[1]{\left\vert#1\right\vert}
\providecommand{\res}[1]{\Res\displaylimits_{#1}} 
\providecommand{\norm}[1]{\left\lVert#1\right\rVert}
\providecommand{\mtx}[1]{\mathbf{#1}}
\providecommand{\mean}[1]{E\left[ #1 \right]}
\providecommand{\fourier}{\overset{\mathcal{F}}{ \rightleftharpoons}}
\providecommand{\system}[1]{\overset{\mathcal{#1}}{ \longleftrightarrow}}
\newcommand{\solution}{\noindent \textbf{Solution: }}
\newcommand{\cosec}{\,\text{cosec}\,}
\providecommand{\dec}[2]{\ensuremath{\overset{#1}{\underset{#2}{\gtrless}}}}
\newcommand{\myvec}[1]{\ensuremath{\begin{pmatrix}#1\end{pmatrix}}}
\newcommand{\mydet}[1]{\ensuremath{\begin{vmatrix}#1\end{vmatrix}}}
\renewcommand{\vec}[1]{\boldsymbol{\mathbf{#1}}}
\def\putbox#1#2#3{\makebox[0in][l]{\makebox[#1][l]{}\raisebox{\baselineskip}[0in][0in]{\raisebox{#2}[0in][0in]{#3}}}}
     \def\rightbox#1{\makebox[0in][r]{#1}}
     \def\centbox#1{\makebox[0in]{#1}}
     \def\topbox#1{\raisebox{-\baselineskip}[0in][0in]{#1}}
     \def\midbox#1{\raisebox{-0.5\baselineskip}[0in][0in]{#1}}

\vspace{3cm}
\title{DIGITAL THERMOMETER USING PT-100}
\author{Harsha B J and Bhargav K}
\maketitle
\tableofcontents
\bigskip

\section{Objective}
The objective of this project is to design and implement a digital thermometer that measures the temperature using a PT-100 Resistance Temperature Detector (RTD), processes the signal through an Arduino microcontroller, and displays the temperature on a 16×2 LCD. In this experiment, the relationship between the voltage across
the PT-100 and the temperature is determined using linear regression (least squares
method).

\section{Training Data}
The training data gathered by the PT-100 to train the Arduino is shown in Table
\ref{tab:train}.

\begin{table}[H]
    \centering
    \input{tables/training_data.tex}
    \caption{Training data.}
    \label{tab:train}
\end{table}

The C++ source \texttt{codes/data.cpp} was used along with \textit{platformio}
to drive the Arduino.

The approximation is shown in Fig. \ref{fig:train}.
\begin{figure}[H]
    \centering
    \includegraphics[width=\columnwidth]{figs/train.png}
    \caption{Training the model.}
    \label{fig:train}
\end{figure}

\vspace{2cm}

\section{Theory}
The PT-100 sensor changes resistance with temperature. Its nominal resistance is 100 $\ohm$
at 0 $\degree$C, and the resistance increases approximately linearly with temperature:
\begin{align}
R_T = R_o(1 + \alpha T)
\end{align}
where $\alpha$ = 0.00385 $\degree C^{-1}$. When placed in a Wheatstone bridge circuit, the resistance variation produces a corresponding voltage change, which can be measured and used to infer the temperature.


\section{Linear Regression Model}
To obtain an empirical model relating the measured voltage V to temperature T, we
collect calibration data by measuring both quantities over a range of known temperatures.
Let the measured data points be ($T_i$
, $V_i$), i = 1, 2,. . , n. We assume a quadratic model for the voltage-temperature relationship
\begin{align}
    V(T) &= n_0+n_1T+n_2T^2 \\
\end{align}
\begin{align}
    \implies \vec{C} = \vec{X^T}\vec{n}  \label{eq:model}
\end{align}
where
\begin{align}
   \vec{X^T} = \myvec{1 & T_1 & T_1^2 \\ 1 & T_2 & T_2^2 \\ . & . & .\\ . & . & .\\ . & . & . \\ 1 & T_n & T_n^2}, \vec{C} = \myvec{V_1 \\ V_2 \\ . \\ . \\ . \\ V_n}
    \label{eq:x-y-theta-def}
\end{align}



\section{Solution}
We approximate $\vec{n^T} = \myvec{n_0 & n_1 & n_2}$ using the least squares method. 
Using the pseudo-inverse method, the solution to \eqref{eq:model} 
is
\begin{align}
    \vec{n} = \brak{\vec{XX}^\top}^{-1}\vec{X}\vec{C}
\end{align}
The Python code \texttt{codes/main.py} solves for $\vec{n}$.

The calculated value of $\vec{n}$ is
\begin{align}
    \vec{n} = \myvec{2.7451\\3.5566\times10^{-3}\\-5.0234\times10^{-8}}
    \label{eq:opt-theta}
\end{align}

To obtain temperature as a function of measured voltage (for Arduino implementation),
we rearrange or numerically invert the above relation:
\begin{align}
T(V) = a_0 + a_1V + a_2V^2
\end{align}
The coefficients $a_i$ can again be found by applying the least squares method to the ($V_i$, $T_i$)
data
\section{Validation}
The validation data set is shown in Table \ref{tab:valid}. The results of the
validation are shown in Fig. \ref{fig:valid}.
\begin{table}[H]
    \centering
    \input{tables/validation_data.tex}
    \caption{Validation data.}
    \label{tab:valid}
\end{table}
\begin{figure}[H]
    \centering
    \includegraphics[width=\columnwidth]{figs/valid.png}
    \caption{Validating the model.}
    \label{fig:valid}
\end{figure}

\section{Observation}
The observations made while conducting the experiment are as follows,
\begin{enumerate}
    \item The readings of arduino had an offset of a few millivolts for every cycle of reading.
    Solution: We wrote an algorithm to take an average of a few sample readings and computed the temperature using it which gave a good precision.
    \item The values of the temperature had an offset of $\pm \,3\degree$ celcius due to the hardware \\malfunctioning.
    \item the accuracy of the training model and validation data is close enough with a less validation.
\end{enumerate}

\section{Conclusion}

This project used Python for machine learning to calibrate the temperature sensor with linear regression. Arduino collected sensor data and showed real-time results. Python made the analysis easier and more accurate, while Arduino handled the hardware side. This simple combination of Python and Arduino creates a smart, affordable device by merging data science and embedded systems effectively.


\end{document}