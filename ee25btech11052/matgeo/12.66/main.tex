\let\negmedspace\undefined
\let\negthickspace\undefined
\documentclass[journal]{IEEEtran}
\usepackage[a5paper, margin=10mm, onecolumn]{geometry}

\usepackage{tfrupee} % Include tfrupee package

\setlength{\headheight}{1cm} % Set the height of the header box
\setlength{\headsep}{0mm}     % Set the distance between the header box and the top of the text
\usepackage{multicol}
\usepackage{gvv-book}
\usepackage{gvv}
\usepackage{cite}
\usepackage{amsmath,amssymb,amsfonts,amsthm}
\usepackage{algorithmic}
\usepackage{graphicx}
\usepackage{textcomp}
\usepackage{xcolor}
\usepackage{txfonts}
\usepackage{listings}
\usepackage{enumitem}
\usepackage{mathtools}
\usepackage{gensymb}
\usepackage{comment}
\usepackage[breaklinks=true]{hyperref}
\usepackage{tkz-euclide} 
\usepackage{listings}
% \usepackage{gvv}                                        
\def\inputGnumericTable{}                                 
\usepackage[latin1]{inputenc}                                
\usepackage{color}                                            
\usepackage{array}                                            
\usepackage{longtable}                                       
\usepackage{calc}                                             
\usepackage{multirow}                                         
\usepackage{hhline}                                           
\usepackage{ifthen}                                           
\usepackage{lscape}
\begin{document}
	
	\bibliographystyle{IEEEtran}
	\vspace{3cm}
	
	\title{12.66}
	\author{EE25BTECH11052 - Shriyansh Kalpesh Chawda}
	% \maketitle
	% \newpage
	% \bigskip
	{\let\newpage\relax\maketitle}
	
	\renewcommand{\thefigure}{\theenumi}
	\renewcommand{\thetable}{\theenumi}
	\setlength{\intextsep}{10pt} 
	
	\numberwithin{equation}{enumi}
	\numberwithin{figure}{enumi}
	\renewcommand{\thetable}{\theenumi}
	\textbf{Question}:\\
The matrix $$ A = \frac{1}{\sqrt{3}} \myvec{1 & 1 + i \\ 1 - i & -1} $$ is:\\
\begin{enumerate}
\begin{multicols}{4}
\item orthogonal  
\item symmetric 
\item anti-symmetric 
\item unitary
\end{multicols}
\end{enumerate}
\hfill{(PH 2014)}\\
\solution
Given matrix:
\begin{equation}
	A = \frac{1}{\sqrt{3}}\myvec{1 & 1+i \\ 1-i & -1}
\end{equation}

\subsection*{Check 1: Symmetric ($A = A^\top$)}

The transpose of $A$ is:
\begin{align}
	A^\top &= \frac{1}{\sqrt{3}}\myvec{1 & 1-i \\ 1+i & -1}
\end{align}

Since $A^\top \neq A$ (the off-diagonal elements are different: $1+i \neq 1-i$), 

\textbf{$A$ is NOT symmetric.}

\subsection*{Check 2: Anti-symmetric ($A = -A^\top$)}

For anti-symmetric:
\begin{align}
	-A^\top &= -\frac{1}{\sqrt{3}}\myvec{1 & 1-i \\ 1+i & -1}\\
	&= \frac{1}{\sqrt{3}}\myvec{-1 & -1+i \\ -1-i & 1}
\end{align}

Since $A \neq -A^\top$,

\textbf{$A$ is NOT anti-symmetric.}

\subsection*{Check 3: Orthogonal ($AA^\top = I$)}

Note: For real matrices, orthogonal means $AA^\top = I$. However, $A$ contains complex entries, so we need to check if it satisfies the real orthogonal property.

\begin{align}
	AA^\top &= \frac{1}{\sqrt{3}}\myvec{1 & 1+i \\ 1-i & -1} \cdot \frac{1}{\sqrt{3}}\myvec{1 & 1-i \\ 1+i & -1}\\
	&= \frac{1}{3}\myvec{1 & 1+i \\ 1-i & -1}\myvec{1 & 1-i \\ 1+i & -1}
\end{align}

Computing the $(1,1)$ entry:
\begin{equation}
	(AA^\top)_{11} = \frac{1}{3}\left[1 \cdot 1 + (1+i)(1+i)\right] = \frac{1}{3}\left[1 + 1 + 2i + i^2\right] = \frac{1}{3}[1 + 2i] \neq 1
\end{equation}

Since this is complex (not real), $AA^\top \neq I$,

\textbf{$A$ is NOT orthogonal.}

\subsection*{Check 4: Unitary ($A\overline{A^\top} = I$)}

For a unitary matrix, we need $A\overline{A^\top} = I$, where $\overline{A^\top} = \overline{A^\top}$ is the conjugate transpose.

The conjugate transpose is:
\begin{align}
	\overline{A^\top} &= \overline{A^\top} = \overline{\frac{1}{\sqrt{3}}\myvec{1 & 1-i \\ 1+i & -1}}\\
	&= \frac{1}{\sqrt{3}}\myvec{1 & 1+i \\ 1-i & -1}
\end{align}

Notice that $\overline{A^\top} = A$ , but let's verify unitarity:

\begin{align}
	A\overline{A^\top} &= \frac{1}{3}\myvec{1 & 1+i \\ 1-i & -1}\myvec{1 & 1+i \\ 1-i & -1}
\end{align}
\begin{equation}
	A\overline{A^\top} = \myvec{1 & 0 \\ 0 & 1} = I
\end{equation}
\textbf{$A$ is UNITARY.}\\
\boxed{\text{Option 4: The matrix } A \text{ is unitary}}	
	
	
	
	
	
\end{document}