\documentclass{beamer}

\usepackage[utf8]{inputenc}
\usepackage{lmodern} 
\usepackage{listings}
\usepackage{xcolor} 
\usepackage{graphicx}
\usepackage{amsmath,amssymb,amsfonts}

\definecolor{myblue}{RGB}{48, 63, 159}
\setbeamercolor{palette primary}{bg=myblue, fg=white}
\setbeamercolor{structure}{fg=myblue}
\setbeamercolor{frametitle}{bg=myblue, fg=white}
\setbeamercolor{title}{bg=myblue, fg=white}
\setbeamercolor{footlinecolor}{bg=myblue, fg=white}

\defbeamertemplate*{title page}{mytemplate}{
	\vfill
	\begin{center}
		\begin{beamercolorbox}[wd=0.8\paperwidth, center, rounded=true, shadow=true]{title}
			\usebeamerfont{title}\inserttitle\par
		\end{beamercolorbox}
		\vspace{2cm} 
		\usebeamerfont{author}\insertauthor
		\vspace{1cm} 
		\usebeamerfont{date}\insertdate
	\end{center}
	\vfill
}

\defbeamertemplate*{frametitle}{mytemplate}{
	\begin{beamercolorbox}[wd=\paperwidth, ht=2.5ex, dp=1.5ex, left]{frametitle}
		\hspace{1em}\usebeamerfont{frametitle}\insertframetitle
	\end{beamercolorbox}
}

\setbeamertemplate{footline}{
	\begin{beamercolorbox}[wd=\paperwidth, ht=2.25ex, dp=1ex]{footlinecolor}
		\hspace{1em}\usebeamerfont{author in footline}\insertshortauthor
		\hfill
		\usebeamerfont{title in footline}\insertshorttitle
		\hfill
		\usebeamerfont{date in footline}\insertdate \hspace{1em} \insertframenumber/\inserttotalframenumber \hspace{0.5em}
	\end{beamercolorbox}
}

\setbeamerfont{author in footline}{size=\tiny}
\setbeamerfont{title in footline}{size=\tiny}
\setbeamerfont{date in footline}{size=\tiny}

\newcommand{\myvec}[1]{\ensuremath{\begin{pmatrix}#1\end{pmatrix}}}
\providecommand{\brak}[1]{\ensuremath{\left(#1\right)}}

\title{12.66}
\author{Shriyansh Chawda-EE25BTECH11052}

\begin{document}
	
	\setbeamertemplate{footline}{} 
	\frame{\titlepage}
	
	\begin{frame}{Question}
		The matrix
		\[
		A = \frac{1}{\sqrt{3}}\myvec{1 & 1+i \\ 1-i & -1}
		\]
		is:
		\begin{enumerate}
			\item orthogonal
			\item symmetric
			\item anti-symmetric
			\item unitary
		\end{enumerate}
		\hfill{(PH 2014)}
	\end{frame}
	
	\begin{frame}{Solution - Given}
		Given matrix:
		\begin{equation}
			A = \frac{1}{\sqrt{3}}\myvec{1 & 1+i \\ 1-i & -1}
		\end{equation}
	\end{frame}
	
	\begin{frame}{Solution - Check 1: Symmetric}
		\textbf{Check 1: Symmetric ($A = A^\top$)}
		
		The transpose of $A$ is:
		\begin{align}
			A^\top &= \frac{1}{\sqrt{3}}\myvec{1 & 1-i \\ 1+i & -1}
		\end{align}
		
		Since $A^\top \neq A$ (the off-diagonal elements are different: $1+i \neq 1-i$),
		
		\vspace{0.5em}
		\textbf{$A$ is NOT symmetric.}
	\end{frame}
	
	\begin{frame}{Solution - Check 2: Anti-symmetric}
		\textbf{Check 2: Anti-symmetric ($A = -A^\top$)}
		
		For anti-symmetric:
		\begin{align}
			-A^\top &= -\frac{1}{\sqrt{3}}\myvec{1 & 1-i \\ 1+i & -1}\\
			&= \frac{1}{\sqrt{3}}\myvec{-1 & -1+i \\ -1-i & 1}
		\end{align}
		
		Since $A \neq -A^\top$,
		
		\vspace{0.5em}
		\textbf{$A$ is NOT anti-symmetric.}
	\end{frame}
	
	\begin{frame}{Solution - Check 3: Orthogonal (Part 1)}
		\textbf{Check 3: Orthogonal ($AA^\top = I$)}
		
		Note: For real matrices, orthogonal means $AA^\top = I$. However, $A$ contains complex entries.
		
		\begin{align}
			AA^\top &= \frac{1}{\sqrt{3}}\myvec{1 & 1+i \\ 1-i & -1} \cdot \frac{1}{\sqrt{3}}\myvec{1 & 1-i \\ 1+i & -1}\\
			&= \frac{1}{3}\myvec{1 & 1+i \\ 1-i & -1}\myvec{1 & 1-i \\ 1+i & -1}
		\end{align}
	\end{frame}
	
	\begin{frame}{Solution - Check 3: Orthogonal (Part 2)}
		Computing the $(1,1)$ entry:
		\begin{align}
			(AA^\top)_{11} &= \frac{1}{3}\left[1 \cdot 1 + (1+i)(1+i)\right]\\
			&= \frac{1}{3}\left[1 + 1 + 2i + i^2\right]\\
			&= \frac{1}{3}[1 + 2i] \neq 1
		\end{align}
		
		Since this is complex (not real), $AA^\top \neq I$,
		
		\vspace{0.5em}
		\textbf{$A$ is NOT orthogonal.}
	\end{frame}
	
	\begin{frame}{Solution - Check 4: Unitary (Part 1)}
		\textbf{Check 4: Unitary ($A\overline{A^\top} = I$)}
		
		For a unitary matrix, we need $A\overline{A^\top} = I$, where $\overline{A^\top}$ is the conjugate transpose.
		
		The conjugate transpose is:
		\begin{align}
			\overline{A^\top} &= \overline{\frac{1}{\sqrt{3}}\myvec{1 & 1-i \\ 1+i & -1}}\\
			&= \frac{1}{\sqrt{3}}\myvec{1 & 1+i \\ 1-i & -1}
		\end{align}
		
		Notice that $\overline{A^\top} = A$ (the matrix is Hermitian!)
	\end{frame}
	
	\begin{frame}{Solution - Check 4: Unitary (Part 2)}
		Let's verify unitarity:
		\begin{align}
			A\overline{A^\top} &= \frac{1}{3}\myvec{1 & 1+i \\ 1-i & -1}\myvec{1 & 1+i \\ 1-i & -1}
		\end{align}
		
		Computing each entry:
		\begin{align}
			(A\overline{A^\top})_{11} &= \frac{1}{3}\left[1 \cdot 1 + (1+i)(1-i)\right]\\
			&= \frac{1}{3}\left[1 + 1 - i^2\right]\\
			&= \frac{1}{3}[1 + 1 + 1] = 1
		\end{align}
	\end{frame}
	
	\begin{frame}{Solution - Check 4: Unitary (Part 3)}
		\textbf{Entry $(1,2)$:}
		\begin{equation}
			(A\overline{A^\top})_{12} = \frac{1}{3}\left[1 \cdot (1+i) + (1+i)(-1)\right] = 0
		\end{equation}
		
		\textbf{Entry $(2,1)$:}
		\begin{equation}
			(A\overline{A^\top})_{21} = \frac{1}{3}\left[(1-i) \cdot 1 + (-1)(1-i)\right] = 0
		\end{equation}
		
		\textbf{Entry $(2,2)$:}
		\begin{align}
			(A\overline{A^\top})_{22} &= \frac{1}{3}\left[(1-i)(1+i) + (-1)(-1)\right]\\
			&= \frac{1}{3}[2 + 1] = 1
		\end{align}
	\end{frame}
	
	\begin{frame}{Solution - Final Answer}
		Therefore:
		\begin{equation}
			A\overline{A^\top} = \myvec{1 & 0 \\ 0 & 1} = I
		\end{equation}
		
		\vspace{1em}
		\textbf{$A$ is UNITARY.}
		
		\vspace{2em}
		\boxed{\text{Option 4: The matrix } A \text{ is unitary}}
	\end{frame}
	
\end{document}