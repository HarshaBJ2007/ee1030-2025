\let\negmedspace\undefined
\let\negthickspace\undefined
\documentclass[journal]{IEEEtran}
\usepackage[a5paper, margin=10mm, onecolumn]{geometry}
\usepackage{tfrupee} % Include tfrupee package
\setlength{\headheight}{1cm} % Set the height of the header box
\setlength{\headsep}{0mm}     % Set the distance between the header box and the top of the text
%\usepackage{gvv-book}
%\usepackage{gvv}
\usepackage{multicol}
\usepackage{cite}
\usepackage{amsmath,amssymb,amsfonts,amsthm}
\usepackage{algorithmic}
\usepackage{graphicx}
\usepackage{textcomp}
\usepackage{xcolor}
\usepackage{txfonts}
\usepackage{listings}
\usepackage{enumitem}
\usepackage{mathtools}
\usepackage{gensymb}
\usepackage{comment}
\usepackage[breaklinks=true]{hyperref}
\usepackage{tkz-euclide} 
\usepackage{listings}
% \usepackage{gvv}                                        
\def\inputGnumericTable{}                                 
\usepackage[latin1]{inputenc}                                
\usepackage{color}                                            
\usepackage{array}                                            
\usepackage{longtable}                                       
\usepackage{calc}                                             
\usepackage{multirow}                                         
\usepackage{hhline}                                           
\usepackage{ifthen}                                           
\usepackage{lscape}

% Define the \myvec command locally since the package was removed
\newcommand{\myvec}[1]{\ensuremath{\begin{pmatrix}#1\end{pmatrix}}}

\begin{document}
	
	\bibliographystyle{IEEEtran}
	\vspace{3cm}
	
	\title{12.482}
	\author{EE25BTECH11052 - Shriyansh Kalpesh Chawda}
	% \maketitle
	% \newpage
	% \bigskip
	{\let\newpage\relax\maketitle}
	
	\renewcommand{\thefigure}{\theenumi}
	\renewcommand{\thetable}{\theenumi}
	\setlength{\intextsep}{10pt} 
	
	%\numberwithin{equation}{enumi}
	\numberwithin{figure}{enumi}
	\renewcommand{\thetable}{\theenumi}
	\textbf{Question}:\\
	Consider the matrix $\mathbf{M} = \myvec{5 & 3\\3 & 5}$. The normalized eigen-vector corresponding to the smallest eigen-value of the matrix $\mathbf{M}$ is:
	
	\begin{multicols}{4}
		\begin{enumerate}
			\item $\begin{pmatrix} \frac{\sqrt{3}}{2} \\ \frac{1}{2} \end{pmatrix}$
			\item $\begin{pmatrix} \frac{\sqrt{3}}{2} \\ -\frac{1}{2} \end{pmatrix}$
			\item $\begin{pmatrix} \frac{1}{\sqrt{2}} \\ -\frac{1}{\sqrt{2}} \end{pmatrix}$
			\item $\begin{pmatrix} \frac{1}{\sqrt{2}} \\ \frac{1}{\sqrt{2}} \end{pmatrix}$
			\end{enumerate}
	\end{multicols}
	
	\textbf{Solution}\\
	Given matrix:
	\begin{align}
		\mathbf{M} = \myvec{5 & 3\\3 & 5}
	\end{align}
	The characteristic equation is:
	\begin{align}
		\det(\mathbf{M} - \lambda\mathbf{I}) &= 0\\
		\det\myvec{5-\lambda & 3\\3 & 5-\lambda} &= 0\\
		(5-\lambda)(5-\lambda) - (3)(3) &= 0\\
		(5-\lambda)^2 - 9 &= 0\\
		\lambda^2 - 10\lambda + 16 &= 0
	\end{align}
	Using the quadratic formula:
	\begin{align}
		\lambda &= \frac{10 \pm \sqrt{100 - 64}}{2} = \frac{10 \pm \sqrt{36}}{2} = \frac{10 \pm 6}{2}\\
		\lambda_1 &= 8 \quad \text{(largest eigenvalue)}\\
		\lambda_2 &= 2 \quad \text{(smallest eigenvalue)}
	\end{align}
	For $\lambda = 2$, solve $(\mathbf{M} - 2\mathbf{I})\mathbf{v} = \mathbf{0}$:
	\begin{align}
		\myvec{5-2 & 3\\3 & 5-2}\myvec{v_1\\v_2} &= \myvec{0\\0}\\
		\myvec{3 & 3\\3 & 3}\myvec{v_1\\v_2} &= \myvec{0\\0}\\
		3v_1 + 3v_2 &= 0\\
		v_1 + v_2 &= 0 \implies v_1 = -v_2
	\end{align}
	Let $v_2 = t$, then $v_1 = -t$.\\
	The general eigenvector is:
	\begin{align}
		\mathbf{v} &= t\myvec{-1\\1} = t\myvec{1\\-1} \cdot (-1)\\
		\mathbf{v} &= \myvec{1\\-1}
	\end{align}
	The normalized eigenvector is:
	\begin{align}
		\hat{\mathbf{v}} &= \frac{\mathbf{v}}{\|\mathbf{v}\|}\\
		\|\mathbf{v}\| &= \sqrt{\mathbf{v}^\top \mathbf{v}}= \sqrt{2}\\
		\hat{\mathbf{v}} &= \frac{1}{\sqrt{2}}\myvec{1\\-1} = \myvec{\frac{1}{\sqrt{2}} \\ -\frac{1}{\sqrt{2}}}
	\end{align}
	$$\boxed{\text{The correct answer is (c) } \myvec{\frac{1}{\sqrt{2}} \\ -\frac{1}{\sqrt{2}}}}$$
	
	
	
\end{document}

