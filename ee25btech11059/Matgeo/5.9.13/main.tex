\let\negmedspace\undefined
\let\negthickspace\undefined
\documentclass[journal]{IEEEtran}
\usepackage[a5paper, margin=10mm, onecolumn]{geometry}
\usepackage{lmodern} 
\usepackage{tfrupee} 
\setlength{\headheight}{1cm}
\setlength{\headsep}{0mm}   

\usepackage{gvv-book}
\usepackage{gvv}
\usepackage{cite}
\usepackage{amsmath,amssymb,amsfonts,amsthm}
\usepackage{algorithmic}
\usepackage{graphicx}
\usepackage{textcomp}
\usepackage{xcolor}
\usepackage{txfonts}
\usepackage{listings}
\usepackage{enumitem}
\usepackage{mathtools}
\usepackage{gensymb}
\usepackage{comment}
\usepackage[breaklinks=true]{hyperref}
\usepackage{tkz-euclide} 
\usepackage{listings}                             
\def\inputGnumericTable{}                                 
\usepackage[latin1]{inputenc}                                
\usepackage{color} 
\usepackage{array}                                            
\usepackage{longtable}                                       
\usepackage{calc}                                             
\usepackage{multirow}                                         
\usepackage{hhline}                                           
\usepackage{ifthen}                                           
\usepackage{lscape}
\usepackage{xparse}

\bibliographystyle{IEEEtran}

\title{5.9.13}
\author{EE25BTECH11059 - Vaishnavi Ramkrishna Anantheertha}

\begin{document}
\maketitle

\renewcommand{\thefigure}{\theenumi}
\renewcommand{\thetable}{\theenumi}

\numberwithin{equation}{enumi}
\numberwithin{figure}{enumi} 









\textbf{Question}:
A shopkeeper has 3 varieties of pens \( A, B \) and \( C \). Meenu purchased 1 pen of each variety for a total of Rs 21. Jeevan purchased 4 pens of \( A \) variety, 3 pens of \( B \) variety and 2 pens of \( C \) variety for Rs 60. While Shikha purchased 6 pens of \( A \) variety, 2 pens of \( B \) variety and 3 pens of \( C \) variety for Rs 70. Using matrix method, find the cost of each variety of pen.\\
\textbf{Solution: }\\
\begin{table}[H]    
  \centering
  \begin{tabular}{|c|c|}
\hline
\textbf{Variable} & \textbf{Value} \\
\hline
$A$ & $(0,-\frac{3}{2})$ \\
\hline
$m$ & $\frac{1}{2}$ \\
\hline
\end{tabular}
  \caption{Variables Used}
  \label{tab:1.10.25}
\end{table}
Let unit cost matrix X be 
\begin{align}
                                     X = \myvec{
                                             a
                                              \\
                                              b
                                               \\
                                               c
                                              }
\end{align}
\begin{align}
 \myvec{
        1 & 1 & 1
        \\
        4 & 3 & 2
        \\
        6 & 2 & 3
            }
X
=
\myvec{
        21
        \\
        60
        \\
        70
        }
\end{align}
Solving it using a Augmented Matrix
\begin{align}
\augvec{3}{1}{
1 & 1 & 1 & 21 \\
4 & 3 & 2 & 60 \\
6 & 2 & 3 & 70}
&\xrightarrow{R_2 \to R_2 - 4R_1}
\augvec{3}{1}{
1 & 1 & 1 & 21 \\
0 & -1 & -2 & -24 \\
6 & 2 & 3 & 70} \\[6pt]
&\xrightarrow{R_3 \to R_3 - 6R_1}
\augvec{3}{1}{
1 & 1 & 1 & 21 \\
0 & -1 & -2 & -24 \\
0 & -4 & -3 & -56} \\[6pt]
&\xrightarrow{R_2 \to -1 \cdot R_2}
\augvec{3}{1}{
1 & 1 & 1 & 21 \\
0 & 1 & 2 & 24 \\
0 & -4 & -3 & -56} \\[6pt]
&\xrightarrow{R_1 \to R_1 - R_2}
\augvec{3}{1}{
1 & 0 & -1 & -3 \\
0 & 1 & 2 & 24 \\
0 & -4 & -3 & -56} \\[6pt]
&\xrightarrow{R_3 \to R_3 + 4R_2}
\augvec{3}{1}{
1 & 0 & -1 & -3 \\
0 & 1 & 2 & 24 \\
0 & 0 & 5 & 40} \\[6pt]
&\xrightarrow{R_3 \to \tfrac{1}{5} R_3}
\augvec{3}{1}{
1 & 0 & -1 & -3 \\
0 & 1 & 2 & 24 \\
0 & 0 & 1 & 8} \\[6pt]
&\xrightarrow{R_1 \to R_1 + R_3}
\augvec{3}{1}{
1 & 0 & 0 & 5 \\
0 & 1 & 2 & 24 \\
0 & 0 & 1 & 8} \\[6pt]
&\xrightarrow{R_2 \to R_2 - 2R_3}
\augvec{3}{1}{
1 & 0 & 0 & 5 \\
0 & 1 & 0 & 8 \\
0 & 0 & 1 & 8}\\
  \Vec{X}=\myvec{5
                \\
                 8
                \\
                8}
\end{align}

Therefore,\\
          cost of pen A = Rs $5$\\
          cost of pen B = Rs $8$\\
          cost of pen C = Rs $8$
\end{document}