\let\negmedspace\undefined
\let\negthickspace\undefined
\documentclass[journal]{IEEEtran}
\usepackage[a5paper, margin=10mm, onecolumn]{geometry}
%\usepackage{lmodern} % Ensure lmodern is loaded for pdflatex
\usepackage{tfrupee} % Include tfrupee package

\setlength{\headheight}{1cm} % Set the height of the header box
\setlength{\headsep}{0mm}     % Set the distance between the header box and the top of the text

\usepackage{gvv-book}
\usepackage{gvv}
\usepackage{cite}
\usepackage{amsmath,amssymb,amsfonts,amsthm}
\usepackage{algorithmic}
\usepackage{graphicx}
\usepackage{textcomp}
\usepackage{xcolor}
\usepackage{txfonts}
\usepackage{listings}
\usepackage{enumitem}
\usepackage{mathtools}
\usepackage{gensymb}
\usepackage{comment}
\usepackage[breaklinks=true]{hyperref}
\usepackage{tkz-euclide}
\usepackage{listings}
% \usepackage{gvv}
\def\inputGnumericTable{}
\usepackage[latin1]{inputenc}
\usepackage{color}
\usepackage{array}
\usepackage{longtable}
\usepackage{calc}
\usepackage{multirow}
\usepackage{hhline}
\usepackage{ifthen}
\usepackage{lscape}
\begin{document}

\bibliographystyle{IEEEtran}

\title{12.453}
\author{EE25BTECH11023 - Venkata Sai}
% \maketitle
% \newpage
% \bigskip
\maketitle
\renewcommand{\thefigure}{\theenumi}
\renewcommand{\thetable}{\theenumi}
\setlength{\intextsep}{10pt} % Space between text and floats

\numberwithin{align}{enumi}
\numberwithin{figure}{enumi}
\renewcommand{\thetable}{\theenumi}
\vspace{-1em}
\textbf{Question:}  \\
A 3 $\times$ 3 matrix $\vec{P}$ is such that, $\vec{P}^3$ = $\vec{P}$. Then the eigenvalues of $\vec{P}$ are\\
\textbf{Solution:}  \\
 Let $\lambda$ be the eigen value of the matrix $\vec{P}$ and $\vec{v}$ be the corresponding eigen vector
 \begin{align}
     \vec{P}\vec{v}=\lambda\vec{v} \\
 \end{align}
 Multiplying both sides with vector $\vec{P}$
 \begin{align}
     \vec{P}\brak{\vec{P}\vec{v}}&=\vec{P}\brak{\lambda\vec{v}} \\
     \vec{P}^{2}\vec{v}&=\lambda\brak{\vec{P}\vec{v}}\\
     \implies \vec{P}^{2}\vec{v}&=\lambda\brak{\lambda\vec{v}} =\lambda^2\vec{v}
 \end{align}
 Hence
 \begin{align}
     \vec{P}^{n}\vec{v}=\lambda^n\vec{v} \\
     \vec{P}^{3}\vec{v}=\lambda^3\vec{v}
 \end{align}
 \begin{align}
     \vec{P}^3&=\vec{P} \\
     \vec{P}^3&-\vec{P}=0 \\
     \lambda^3\vec{v}-\lambda\vec{v}=0  &\implies \brak{\lambda^3-\lambda}\vec{v}=0 \\
     \lambda^3-\lambda=0 &\implies \lambda\brak{\lambda^2-1}=0 \\
     \lambda=0\ &\text{and}\ \lambda^2-1=0 \\
     \lambda=0\ &\text{and}\ \lambda=\pm1
 \end{align}
 Hence the eigen values of $\vec{P} = 0,1,-1$
 \end{document}
