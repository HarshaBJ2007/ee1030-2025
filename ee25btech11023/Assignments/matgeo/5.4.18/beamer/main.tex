\documentclass{beamer}
\usepackage[utf8]{inputenc}

\usetheme{Boadilla}
\usecolortheme{lily}
\usepackage{amsmath,amssymb,amsfonts,amsthm}
\usepackage{mathtools}
\usepackage{txfonts}
\usepackage{tkz-euclide}
\usepackage{listings}
\usepackage{adjustbox}
\usepackage{gvv}
\usepackage{array}
\usepackage{tabularx}
\usepackage{lmodern}
\usepackage{circuitikz}
\usepackage{tikz}
\usepackage{graphicx}

\setbeamertemplate{footline}
{
  \leavevmode%
  \hbox{%
  \begin{beamercolorbox}[wd=\paperwidth,ht=2.25ex,dp=1ex,right]{author in head/foot}%
    \insertframenumber{} / \inserttotalframenumber\hspace*{2ex} 
  \end{beamercolorbox}}%
  \vskip0pt%
}

\usepackage{tcolorbox}
\tcbuselibrary{minted,breakable,xparse,skins}


\lstset{
%language=C,
frame=single, 
breaklines=true,
columns=fullflexible
}

\numberwithin{equation}{section}

\lstset{
  language=Python,
  basicstyle=\ttfamily\small,
  keywordstyle=\color{blue},
  stringstyle=\color{orange},
  numbers=left,
  numberstyle=\tiny\color{gray},
  breaklines=true,
  showstringspaces=false
}

\title{Problem 5.4.18}
\author{ee25btech11023-Venkata Sai}

\date{\today} 
\begin{document}

\begin{frame}
\titlepage
\end{frame}

\section*{Outline}
\begin{frame}
\tableofcontents
\end{frame}

\section{Problem}

\begin{frame}
\frametitle{Problem}
\setcounter{section}{1}
Using elementary transformations, find the inverse of the following matrix
\begin{align*}
\myvec{4&5\\3&4}
\end{align*}
\end{frame}
%\subsection{Literature}
\section{Solution}

\subsection{Augmented Matrix}
\begin{frame}
\frametitle{Augmented Matrix}
Given  
\begin{align}
\vec{A}=\myvec{4&5\\3&4}
\end{align}
Let $\vec{A}^{-1}$ be the inverse of $\vec{A}$.Then
\begin{align}
    \vec{A}\vec{A}^{-1}=\vec{I}
\end{align}
Augmented matrix of $\augvec{1}{1}{\vec{A} & \vec{I}}$ is given by
\begin{align}
    \augvec{2}{2}{4 & 5 & 1 & 0 \\ 3 & 4 & 0 & 1} \xrightarrow{R_2 \rightarrow 4R_2 - 3R_1}\augvec{2}{2}{4 & 5 & 1 & 0 \\ 0 & 1 & -3 & 4} 
    \end{align}
    \begin{align}
      \augvec{2}{2}{4 & 5 & 1 & 0 \\ 0 & 1 & -3 & 4}  \xrightarrow{R_1 \rightarrow R_1 - 5R_2}\augvec{2}{2}{4 & 0 & 16 & -20 \\ 0 & 1 & -3 & 4} 
    \end{align}
    \begin{align}
   \augvec{2}{2}{4 & 0 & 16 & -20 \\ 0 & 1 & -3 & 4} \xrightarrow{R_1 \rightarrow \tfrac{1}{4} R_1}\augvec{2}{2}{1 & 0 & 4 & -5 \\ 0 & 1 & -3 & 4}
\end{align}
\end{frame}
\subsection{Conclusion}
\begin{frame}
\frametitle{Conclusion}
 Hence the inverse of the matrix $\myvec{4&5\\3&4}$ is \myvec{4&-5\\-3&4}
\end{frame}
\section{C Code}
\begin{frame}[fragile]
\frametitle{C Code}
\begin{lstlisting}[language=C]
void get_system_coeffs(double* out_coeffs) {
     
    out_coeffs[0] = 4.0;
    out_coeffs[1] = 5.0;
   
    out_coeffs[2] = 3.0;
    out_coeffs[3] = 4.0;
}
    \end{lstlisting}
\end{frame}
\section{Python Code}
\begin{frame}[fragile]
\frametitle{Python Code for Solving}
\begin{lstlisting}[language=Python]
import ctypes
import numpy as np

lib = ctypes.CDLL('./code.so')
double_array_4 = ctypes.c_double * 4
lib.get_system_coeffs.argtypes = [ctypes.POINTER(ctypes.c_double)]
out_data_c = double_array_4()

lib.get_system_coeffs(out_data_c)
coeffs = list(out_data_c)
M = np.linalg.inv([
    [coeffs[0], coeffs[1]],
    [coeffs[2], coeffs[3]],
])

print(M)
\end{lstlisting}
\end{frame}
\end{document}
