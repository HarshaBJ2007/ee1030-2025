\let\negmedspace\undefined
\let\negthickspace\undefined
\documentclass[journal]{IEEEtran}
\usepackage[a5paper, margin=10mm, onecolumn]{geometry}
%\usepackage{lmodern} % Ensure lmodern is loaded for pdflatex
\usepackage{tfrupee} % Include tfrupee package

\setlength{\headheight}{1cm} % Set the height of the header box
\setlength{\headsep}{0mm}     % Set the distance between the header box and the top of the text

\usepackage{gvv-book}
\usepackage{gvv}
\usepackage{cite}
\usepackage{amsmath,amssymb,amsfonts,amsthm}
\usepackage{algorithmic}
\usepackage{graphicx}
\usepackage{textcomp}
\usepackage{xcolor}
\usepackage{txfonts}
\usepackage{listings}
\usepackage{enumitem}
\usepackage{mathtools}
\usepackage{gensymb}
\usepackage{comment}
\usepackage[breaklinks=true]{hyperref}
\usepackage{tkz-euclide} 
\usepackage{listings}
% \usepackage{gvv}                                        
\def\inputGnumericTable{}                                 
\usepackage[latin1]{inputenc}                                
\usepackage{color}                                            
\usepackage{array}                                            
\usepackage{longtable}                                       
\usepackage{calc}                                             
\usepackage{multirow}                                         
\usepackage{hhline}                                           
\usepackage{ifthen}                                           
\usepackage{lscape}
\begin{document}

\bibliographystyle{IEEEtran}

\title{12.349}
\author{EE25BTECH11023 - Venkata Sai}
% \maketitle
% \newpage
% \bigskip
\maketitle 
\renewcommand{\thefigure}{\theenumi}
\renewcommand{\thetable}{\theenumi}
\setlength{\intextsep}{10pt} % Space between text and floats

\numberwithin{align}{enumi}
\numberwithin{figure}{enumi}
\renewcommand{\thetable}{\theenumi}
\vspace{-1em}
\textbf{Question:}  \\
Let $T_1 , T_2 : \mathbb{R}_5 \rightarrow \mathbb{R}_3 $ be linear transformations such that rank($T_1$) = 3 and
nullity$\brak{T_2}$ = 3. Let $T_3 : \mathbb{R}_3 \rightarrow \mathbb{R}_3$ be a linear transformation such that $T_3 \circ T_1 = T_2$ .
Then rank$\brak{T_3}$ is \dots \hfill (MA 2014) \\
\textbf{Solution:}  \\
According to Rank-Nullity theorem,\\
For a linear transformation $T : \mathbb{R}_m \rightarrow \mathbb{R}_n $
\begin{align}
    \text{rank}\brak{T}+\text{nullity}\brak{T}=\dim\brak{\text{domain}}
\end{align}
where $\dim\mathbb{R}_m $ is the dimension of the domain i.e vector space $\mathbb{R}_m $ \\ \newline
Given $T_2 : \mathbb{R}_5 \rightarrow \mathbb{R}_3 $ and nullity$\brak{T_2}$=3 
\begin{align}
    \text{rank}\brak{T_2}+\text{nullity}&\brak{T_2}=\dim\mathbb{R}_5 \\
    \text{rank}\brak{T_2}&+3=5 \\
    \text{rank}\brak{T_2} &= 2
\end{align}
Given $T_1 : \mathbb{R}_5 \rightarrow \mathbb{R}_3 $ and rank($T_1$)=3
\begin{align}
    &\dim\brak{\text{Co-domain}} =3 \\
    &\text{rank}\brak{T_1}=\dim\brak{\text{Co-domain}} 
\end{align}
It is onto and hence
\begin{align}
   \dim\brak{\text{Im}\brak{T_1}}&= \dim\brak{\text{Co-domain}} \\
   \dim\brak{\text{Im}\brak{T_1}}&= 3 \implies \text{Im}\brak{T_1}=\mathbb{R}_3
\end{align}
where $\text{Im}\brak{T}$ is the Image space of the linear transformation $T$ \\
Given $T_3 : \mathbb{R}_3 \rightarrow \mathbb{R}_3 $ 
\begin{align}
    T_3 \circ T_1 &= T_2\\
   \brak{ T_3 \circ T_1}\brak{\mathbb{R}_5} &= \text{Im}\brak{T_2} \\
   T_3 \brak{T_1\brak{R_5}} &= \text{Im}\brak{T_2} \\
   T_3 \brak{\text{Im}\brak{T_1}} &= \text{Im}\brak{T_2} \\
   T_3 \brak{\mathbb{R}_3} &= \text{Im}\brak{T_2} \\
   \text{Im}\brak{T_3} &= \text{Im}\brak{T_2} \\
   \implies  \text{rank}(T_3) &= \text{rank}(T_2)
\end{align}
From \brak{4}
\begin{align}
    \text{rank}(T_3)=2
\end{align}
 \end{document}
