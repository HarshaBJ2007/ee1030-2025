\let\negmedspace\undefined
\let\negthickspace\undefined
\documentclass[journal]{IEEEtran}
\usepackage[a5paper, margin=10mm, onecolumn]{geometry}
%\usepackage{lmodern} % Ensure lmodern is loaded for pdflatex
\usepackage{tfrupee} % Include tfrupee package

\setlength{\headheight}{1cm} % Set the height of the header box
\setlength{\headsep}{0mm}     % Set the distance between the header box and the top of the text

\usepackage{gvv-book}
\usepackage{gvv}
\usepackage{cite}
\usepackage{amsmath,amssymb,amsfonts,amsthm}
\usepackage{algorithmic}
\usepackage{graphicx}
\usepackage{textcomp}
\usepackage{xcolor}
\usepackage{txfonts}
\usepackage{listings}
\usepackage{enumitem}
\usepackage{mathtools}
\usepackage{gensymb}
\usepackage{comment}
\usepackage[breaklinks=true]{hyperref}
\usepackage{tkz-euclide} 
\usepackage{listings}
% \usepackage{gvv}                                        
\def\inputGnumericTable{}                                 
\usepackage[latin1]{inputenc}                                
\usepackage{color}                                            
\usepackage{array}                                            
\usepackage{longtable}                                       
\usepackage{calc}                                             
\usepackage{multirow}                                         
\usepackage{hhline}                                           
\usepackage{ifthen}                                           
\usepackage{lscape}
\begin{document}

\bibliographystyle{IEEEtran}

\title{12.141}
\author{EE25BTECH11023 - Venkata Sai}
% \maketitle
% \newpage
% \bigskip
\maketitle 
\renewcommand{\thefigure}{\theenumi}
\renewcommand{\thetable}{\theenumi}
\setlength{\intextsep}{10pt} % Space between text and floats

\numberwithin{align}{enumi}
\numberwithin{figure}{enumi}
\renewcommand{\thetable}{\theenumi}

\textbf{Question:}  \\
Let $\vec{A}$ be a 3$\times$3 matrix. Suppose that the eigenvalues of $\vec{A}$ are -1, 0, 1 with respective
eigenvectors $(1, -1, 0)^\top , (1, 1, -2)^\top\ \text{and}\ (1, 1, 1)^\top$ . Then 6$\vec{A}$ equals 
\begin{enumerate}
\begin{multicols}{4}
    \item \myvec{-1&5&2\\5&-1&2\\2&2&2}
    \item \myvec{1&0&0\\0&-1&0\\0&0&0}
    \item \myvec{1&5&3\\5&1&3\\3&3&3}
\item \myvec{-3&9&0\\9&-3&0\\0&0&6}
\end{multicols}
\end{enumerate}
\textbf{Solution:}  \\
 For an invertible matrix $\vec{P}$
 \begin{align}
     \vec{A}=\vec{P}\vec{D}\vec{P}^{-1}
 \end{align}
 Given eigen values are
 \begin{align}
    \lambda_1=-1,\lambda_2=0,\lambda_3=1
 \end{align}
 Given eigen vectors are
 \begin{align}
\vec{x_1}=\myvec{1\\-1\\0},\vec{x_2}=\myvec{1\\1\\-2},\vec{x_3}=\myvec{1\\1\\1}
 \end{align}
 where
 \begin{align}
\vec{D}=\myvec{\lambda_1&0&0\\0&\lambda_2&0\\0&0&\lambda_3}=\myvec{-1&0&0\\0&0&0\\0&0&1}\\
\vec{P}=\myvec{\vec{x_1} & \vec{x_2} & \vec{x_3}}=\myvec{1&1&1\\-1&1&1\\0&-2&1} 
 \end{align}
 \begin{align}
  |\vec{P}|=1\brak{1+2}-1\brak{-1+0}+1\brak{2+0}=3+1+2=6 \neq 0
 \end{align}
 \begin{align}
     \vec{P}\vec{P}^{-1}=\vec{I}
 \end{align}
 Augmented matrix of $\augvec{1}{1}{\vec{P} & \vec{I}}$ is given by
 \begin{align}
     \augvec{3}{3}{
1 & 1 & 1  & 1 & 0 & 0 \\
-1 & 1 & 1 & 0 & 1 & 0\\
0 & -2 & 1 & 0 & 0 & 1}
& \xrightarrow{R_2\rightarrow \frac{1}{2}\brak{R_1+R_2}}  \augvec{3}{3}{
1 & 1 & 1  & 1 & 0 & 0 \\
0 & 1 & 1 & \frac{1}{2} & \frac{1}{2} & 0\\
0 & -2 & 1 & 0 & 0 & 1}   \\
\augvec{3}{3}{
1 & 1 & 1  & 1 & 0 & 0 \\
0 & 1 & 1 & \frac{1}{2} & \frac{1}{2} & 0\\
0 & -2 & 1 & 0 & 0 & 1} &\xrightarrow[R_1\rightarrow R_1-R_2]{R_3\rightarrow R_3+2R_2}\augvec{3}{3}{
1 & 0 & 0  &  \frac{1}{2} & -\frac{1}{2} & 0 \\
0 & 1 & 1 & \frac{1}{2} & \frac{1}{2} & 0\\
0 & 0 & 3 & 1 & 1 & 1} \\
\augvec{3}{3}{
1 & 0 & 0  &  \frac{1}{2} & -\frac{1}{2} & 0 \\
0 & 1 & 1 & \frac{1}{2} & \frac{1}{2} & 0\\
0 & 0 & 3 & 1 & 1 & 1}  &\xrightarrow{R_2\rightarrow R_2-\frac{1}{3}R_3} \augvec{3}{3}{
1 & 0 & 0  &  \frac{1}{2} & -\frac{1}{2} & 0 \\
0 & 1 & 0 & \frac{1}{6} & \frac{1}{6} & -\frac{1}{3}\\
0 & 0 & 1 & \frac{1}{3} & \frac{1}{3} & \frac{1}{3}} 
 \end{align}
 
 \begin{align}
    \vec{P}^{-1}=\myvec{\frac{1}{2} & -\frac{1}{2} & 0 \\\frac{1}{6} & \frac{1}{6} & -\frac{1}{3}\\ \frac{1}{3} & \frac{1}{3} & \frac{1}{3}}
 \end{align}
 \begin{align}
6\vec{A}=6\vec{P}\vec{D}\vec{P}^{-1}=\brak{\vec{P}\vec{D}}\brak{6\vec{P}^{-1}} 
\end{align}
\begin{align}
6\vec{A}&=\brak{\myvec{1&1&1\\-1&1&1\\0&-2&1}\myvec{-1&0&0\\0&0&0\\0&0&1}}\brak{6\myvec{\frac{1}{2} & -\frac{1}{2} & 0 \\\frac{1}{6} & \frac{1}{6} & -\frac{1}{3}\\ \frac{1}{3} & \frac{1}{3} & \frac{1}{3}}}\\
&=\brak{\myvec{1&1&1\\-1&1&1\\0&-2&1}\myvec{-1&0&0\\0&0&0\\0&0&1} }\myvec{3 & -3 & 0 \\1 & 1 & -2\\ 2 & 2 & 2}\\
&=\myvec{-1 & 0 & 1 \\ 1& 0 & 1 \\0 & 0 & 1}\myvec{3 & -3 & 0 \\1 & 1 & -2\\ 2 & 2 & 2}
=\myvec{-3+2& 3+2&0+2\\3+2&-3+2&0+2\\0+2&0+2&0+2} \\
&=\myvec{-1&5&2\\5&-1&2\\2&2&2}
\end{align}
\end{document}  
