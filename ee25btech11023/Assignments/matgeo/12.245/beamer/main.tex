\documentclass{beamer}
\usepackage[utf8]{inputenc}

\usetheme{Boadilla}
\usecolortheme{lily}
\usepackage{amsmath,amssymb,amsfonts,amsthm}
\usepackage{mathtools}
\usepackage{txfonts}
\usepackage{tkz-euclide}
\usepackage{listings}
\usepackage{multicol}
\usepackage{adjustbox}
\usepackage{array}
\usepackage{tabularx}
\usepackage{lmodern}
\usepackage{gvv}
\usepackage{circuitikz}
\usepackage{tikz}
\usepackage{graphicx}

\setbeamertemplate{footline}
{
  \leavevmode%
  \hbox{%
  \begin{beamercolorbox}[wd=\paperwidth,ht=2.25ex,dp=1ex,right]{author in head/foot}%
    \insertframenumber{} / \inserttotalframenumber\hspace*{2ex}
  \end{beamercolorbox}}%
  \vskip0pt%
}

\usepackage{tcolorbox}
\tcbuselibrary{minted,breakable,xparse,skins}




\providecommand{\nCr}[2]{\,^{#1}C_{#2}} % nCr
\providecommand{\nPr}[2]{\,^{#1}P_{#2}} % nPr
\providecommand{\mbf}{\mathbf}
\providecommand{\pr}[1]{\ensuremath{\Pr\left(#1\right)}}
\providecommand{\qfunc}[1]{\ensuremath{Q\left(#1\right)}}
\providecommand{\sbrak}[1]{\ensuremath{{}\left[#1\right]}}
\providecommand{\lsbrak}[1]{\ensuremath{{}\left[#1\right.}}
\providecommand{\rsbrak}[1]{\ensuremath{{}\left.#1\right]}}
\providecommand{\brak}[1]{\ensuremath{\left(#1\right)}}
\providecommand{\lbrak}[1]{\ensuremath{\left(#1\right.}}
\providecommand{\rbrak}[1]{\ensuremath{\left.#1\right)}}
\providecommand{\cbrak}[1]{\ensuremath{\left\{#1\right\}}}
\providecommand{\lcbrak}[1]{\ensuremath{\left\{#1\right.}}
\providecommand{\rcbrak}[1]{\ensuremath{\left.#1\right\}}}
\theoremstyle{remark}
\newcommand{\sgn}{\mathop{\mathrm{sgn}}}
\providecommand{\abs}[1]{\left\vert#1\right\vert}
\providecommand{\res}[1]{\Res\displaylimits_{#1}}
\providecommand{\norm}[1]{\lVert#1\rVert}
\providecommand{\mtx}[1]{\mathbf{#1}}
\providecommand{\mean}[1]{E\left[ #1 \right]}
\providecommand{\fourier}{\overset{\mathcal{F}}{ \rightleftharpoons}}
%\providecommand{\hilbert}{\overset{\mathcal{H}}{ \rightleftharpoons}}
\providecommand{\system}{\overset{\mathcal{H}}{ \longleftrightarrow}}
	%\newcommand{\solution}[2]{\textbf{Solution:}{#1}}
%\newcommand{\solution}{\noindent \textbf{Solution: }}
\providecommand{\dec}[2]{\ensuremath{\overset{#1}{\underset{#2}{\gtrless}}}}
\newcommand{\myvec}[1]{\ensuremath{\begin{pmatrix}#1\end{pmatrix}}}
\let\vec\mathbf

\lstset{
%language=C,
frame=single,
breaklines=true,
columns=fullflexible
}

\numberwithin{equation}{section}

\lstset{
  language=Python,
  basicstyle=\ttfamily\small,
  keywordstyle=\color{blue},
  stringstyle=\color{orange},
  numbers=left,
  numberstyle=\tiny\color{gray},
  breaklines=true,
  showstringspaces=false
}

\title{Problem 12.245}
\author{ee25btech11023-Venkata Sai}

\date{\today}
\begin{document}

\begin{frame}
\titlepage
\end{frame}

\section*{Outline}
\begin{frame}
\tableofcontents
\end{frame}

\section{Problem}

\begin{frame}
\frametitle{Problem}
Which one of the following matrices has the same eigenvalues as that of \myvec{1&2\\4&3}
\begin{enumerate}
\begin{multicols}{4}
    \item \myvec{3&4\\1&2}
    \item \myvec{1&4\\2&3}
    \item \myvec{4&2\\1&3}
\item \myvec{2&4\\1&3}
\end{multicols}
\end{enumerate}
\end{frame}
%\subsection{Literature}
\section{Solution}


\subsection{Equation}
\begin{frame}
\frametitle{Equation}
 Let the given matrix be
\begin{align}
    \myvec{1&2\\4&3}
\end{align}
Characteristic equation of Matrix is given by
\begin{align}
    \vec{A}=\myvec{a&b\\c&d}\\
    |\vec{A}-\lambda\vec{I}|=0
    \end{align}
    \begin{align}
    \mydet{\myvec{a&b\\c&d}-\myvec{\lambda&0 \\0&\lambda}}=0\\
    \mydet{\myvec{a-\lambda&b\\c&d-\lambda}}=0\\
    \brak{a-\lambda}\brak{d-\lambda}-bc=0\\
\lambda^2-a\lambda-d\lambda+ad-bc=0
    \end{align}
\end{frame}
\subsection{Verification}
\begin{frame}
\frametitle{Verification}
 \begin{align}
     \lambda^2-\brak{a+d}\lambda+ad-bc=0 \\
 \lambda^2-\brak{tr\vec{A}}\lambda+\det{\vec{A}}=0
\end{align}
where $\lambda$ is the eigen value and tr$\vec{A}$ is the trace of $\vec{A}$
\begin{align}
    \vec{A}&=\myvec{1&2\\4&3}\\
    tr\vec{A}=1+3=4,&\det{\vec{A}}=3-8=-5
\end{align}
Option \brak{1}
\begin{align}
  \vec{V}=\myvec{3&4\\1&2}\\
  tr\vec{V}=3+2=5,&\det{\vec{V}}=6-4=2
\end{align}
Not equal to the given matrix $\vec{A}$. Hence the eigen values are not same \\
 \end{frame}
\begin{frame}
\frametitle{Verification}
Option \brak{2}
\begin{align}
  \vec{V}=\myvec{1&4\\2&3}\\
  tr\vec{V}=1+3=4,&\det{\vec{V}}=3-8=-5
\end{align}
Equal to the given matrix $\vec{A}$. Hence the eigen values are  same \\
Option \brak{3}
\begin{align}
  \vec{V}=\myvec{4&2\\1&3}\\
  tr\vec{V}=4+3=7,&\det{\vec{V}}=12-2=10
\end{align}
Not equal to the given matrix $\vec{A}$. Hence the eigen values are not same
Option \brak{4}
\begin{align}
  \vec{V}=\myvec{2&4\\1&3}\\
  tr\vec{V}=2+3=5,&\det{\vec{V}}=6-4=2
\end{align}
 \end{frame}
\subsection{Conclusion}
\begin{frame}[fragile]
\frametitle{Conclusion}
 Not equal to the given matrix $\vec{A}$. Hence the eigen values are not same \\

Hence option \brak{2} is the correct answer
\end{frame}

\section{C Code}
\begin{frame}[fragile]
\frametitle{C Code}
\begin{lstlisting}[language=C]
void get_matrices(double* data) {
    // Given Matrix
    data[0] = 1.0; data[1] = 2.0;
    data[2] = 4.0; data[3] = 3.0;
    // Option 1
    data[4] = 3.0; data[5] = 4.0;
    data[6] = 1.0; data[7] = 2.0;
    // Option 2
    data[8] = 1.0; data[9] = 4.0;
    data[10] = 2.0; data[11] = 3.0;
    // Option 3
    data[12] = 4.0; data[13] = 2.0;
    data[14] = 1.0; data[15] = 3.0;
    // Option 4
    data[16] = 2.0; data[17] = 4.0;
    data[18] = 1.0; data[19] = 3.0;
}

\end{lstlisting}
\end{frame}

\section{Python Code}
\begin{frame}[fragile]
\frametitle{Python Code for Solving}
\begin{lstlisting}[language=Python]
import ctypes
import numpy as np

lib = ctypes.CDLL('./code.so')
double_array_20 = ctypes.c_double * 20
lib.get_matrices.argtypes = [ctypes.POINTER(ctypes.c_double)]

out_data_c = double_array_20()
lib.get_matrices(out_data_c)
data = np.array(list(out_data_c))
options = data.reshape(5, 2, 2)

given = options[0]

trace = np.trace(given)
det = np.linalg.det(given)

match_index = -1
\end{lstlisting}
\end{frame}
 \begin{frame}[fragile]
\frametitle{Python Code for Solving}
\begin{lstlisting}
for i in range(1,4):
    if np.isclose(np.trace(options[i]),trace) and np.isclose(np.linalg.det(options[i]),det):
        print("Option",i,"is the correct answer ")
        break
\end{lstlisting}
\end{frame}

\end{document}
