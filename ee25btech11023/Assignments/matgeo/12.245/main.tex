\let\negmedspace\undefined
\let\negthickspace\undefined
\documentclass[journal]{IEEEtran}
\usepackage[a5paper, margin=10mm, onecolumn]{geometry}
%\usepackage{lmodern} % Ensure lmodern is loaded for pdflatex
\usepackage{tfrupee} % Include tfrupee package

\setlength{\headheight}{1cm} % Set the height of the header box
\setlength{\headsep}{0mm}     % Set the distance between the header box and the top of the text

\usepackage{gvv-book}
\usepackage{gvv}
\usepackage{cite}
\usepackage{amsmath,amssymb,amsfonts,amsthm}
\usepackage{algorithmic}
\usepackage{graphicx}
\usepackage{textcomp}
\usepackage{xcolor}
\usepackage{txfonts}
\usepackage{listings}
\usepackage{enumitem}
\usepackage{mathtools}
\usepackage{gensymb}
\usepackage{comment}
\usepackage[breaklinks=true]{hyperref}
\usepackage{tkz-euclide}
\usepackage{listings}
% \usepackage{gvv}
\def\inputGnumericTable{}
\usepackage[latin1]{inputenc}
\usepackage{color}
\usepackage{array}
\usepackage{longtable}
\usepackage{calc}
\usepackage{multirow}
\usepackage{hhline}
\usepackage{ifthen}
\usepackage{lscape}
\begin{document}

\bibliographystyle{IEEEtran}

\title{12.245}
\author{EE25BTECH11023 - Venkata Sai}
% \maketitle
% \newpage
% \bigskip
\maketitle
\renewcommand{\thefigure}{\theenumi}
\renewcommand{\thetable}{\theenumi}
\setlength{\intextsep}{10pt} % Space between text and floats

\numberwithin{align}{enumi}
\numberwithin{figure}{enumi}
\renewcommand{\thetable}{\theenumi}

\textbf{Question:}  \\
Which one of the following matrices has the same eigenvalues as that of \myvec{1&2\\4&3}
\begin{enumerate}
\begin{multicols}{4}
    \item \myvec{3&4\\1&2}
    \item \myvec{1&4\\2&3}
    \item \myvec{4&2\\1&3}
\item \myvec{2&4\\1&3}
\end{multicols}
\end{enumerate}
\textbf{Solution:}  \\
Let the given matrix be
\begin{align}
    \myvec{1&2\\4&3}
\end{align}
Characteristic equation of Matrix is given by
\begin{align}
    \vec{A}=\myvec{a&b\\c&d}\\
    |\vec{A}-\lambda\vec{I}|=0
    \end{align}
    \begin{align}
    \mydet{\myvec{a&b\\c&d}-\myvec{\lambda&0 \\0&\lambda}}=0\\
    \mydet{\myvec{a-\lambda&b\\c&d-\lambda}}=0\\
    \brak{a-\lambda}\brak{d-\lambda}-bc=0\\
    \lambda^2-a\lambda-d\lambda+ad-bc=0  \\
    \lambda^2-\brak{a+d}\lambda+ad-bc=0 \\
    \lambda^2-\brak{tr\vec{A}}\lambda+\det{\vec{A}}=0
\end{align}
where $\lambda$ is the eigen value and tr$\vec{A}$ is the trace of $\vec{A}$
\begin{align}
    \vec{A}&=\myvec{1&2\\4&3}\\
    tr\vec{A}=1+3=4,&\det{\vec{A}}=3-8=-5
\end{align}
Option \brak{1}
\begin{align}
  \vec{V}=\myvec{3&4\\1&2}\\
  tr\vec{V}=3+2=5,&\det{\vec{V}}=6-4=2
\end{align}
Not equal to the given matrix $\vec{A}$. Hence the eigen values are not same \\
Option \brak{2}
\begin{align}
  \vec{V}=\myvec{1&4\\2&3}\\
  tr\vec{V}=1+3=4,&\det{\vec{V}}=3-8=-5
\end{align}
Equal to the given matrix $\vec{A}$. Hence the eigen values are  same \\
Option \brak{3}
\begin{align}
  \vec{V}=\myvec{4&2\\1&3}\\
  tr\vec{V}=4+3=7,&\det{\vec{V}}=12-2=10
\end{align}
Not equal to the given matrix $\vec{A}$. Hence the eigen values are not same \\
Option \brak{4}
\begin{align}
  \vec{V}=\myvec{2&4\\1&3}\\
  tr\vec{V}=2+3=5,&\det{\vec{V}}=6-4=2
\end{align}
Not equal to the given matrix $\vec{A}$. Hence the eigen values are not same \\

Hence option \brak{2} is the correct answer
\end{document}
