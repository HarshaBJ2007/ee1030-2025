\documentclass{beamer}
\usepackage[utf8]{inputenc}

\usetheme{Boadilla}
\usecolortheme{lily}
\usepackage{amsmath,amssymb,amsfonts,amsthm}
\usepackage{mathtools}
\usepackage{txfonts}
\usepackage{tkz-euclide}
\usepackage{listings}
\usepackage{multicol}
\usepackage{adjustbox}
\usepackage{array}
\usepackage{tabularx}
\usepackage{lmodern}
\usepackage{gvv}
\usepackage{circuitikz}
\usepackage{tikz}
\usepackage{graphicx}

\setbeamertemplate{footline}
{
  \leavevmode%
  \hbox{%
  \begin{beamercolorbox}[wd=\paperwidth,ht=2.25ex,dp=1ex,right]{author in head/foot}%
    \insertframenumber{} / \inserttotalframenumber\hspace*{2ex} 
  \end{beamercolorbox}}%
  \vskip0pt%
}

\usepackage{tcolorbox}
\tcbuselibrary{minted,breakable,xparse,skins}




\providecommand{\nCr}[2]{\,^{#1}C_{#2}} % nCr
\providecommand{\nPr}[2]{\,^{#1}P_{#2}} % nPr
\providecommand{\mbf}{\mathbf}
\providecommand{\pr}[1]{\ensuremath{\Pr\left(#1\right)}}
\providecommand{\qfunc}[1]{\ensuremath{Q\left(#1\right)}}
\providecommand{\sbrak}[1]{\ensuremath{{}\left[#1\right]}}
\providecommand{\lsbrak}[1]{\ensuremath{{}\left[#1\right.}}
\providecommand{\rsbrak}[1]{\ensuremath{{}\left.#1\right]}}
\providecommand{\brak}[1]{\ensuremath{\left(#1\right)}}
\providecommand{\lbrak}[1]{\ensuremath{\left(#1\right.}}
\providecommand{\rbrak}[1]{\ensuremath{\left.#1\right)}}
\providecommand{\cbrak}[1]{\ensuremath{\left\{#1\right\}}}
\providecommand{\lcbrak}[1]{\ensuremath{\left\{#1\right.}}
\providecommand{\rcbrak}[1]{\ensuremath{\left.#1\right\}}}
\theoremstyle{remark}
\newcommand{\sgn}{\mathop{\mathrm{sgn}}}
\providecommand{\abs}[1]{\left\vert#1\right\vert}
\providecommand{\res}[1]{\Res\displaylimits_{#1}} 
\providecommand{\norm}[1]{\lVert#1\rVert}
\providecommand{\mtx}[1]{\mathbf{#1}}
\providecommand{\mean}[1]{E\left[ #1 \right]}
\providecommand{\fourier}{\overset{\mathcal{F}}{ \rightleftharpoons}}
%\providecommand{\hilbert}{\overset{\mathcal{H}}{ \rightleftharpoons}}
\providecommand{\system}{\overset{\mathcal{H}}{ \longleftrightarrow}}
	%\newcommand{\solution}[2]{\textbf{Solution:}{#1}}
%\newcommand{\solution}{\noindent \textbf{Solution: }}
\providecommand{\dec}[2]{\ensuremath{\overset{#1}{\underset{#2}{\gtrless}}}}
\newcommand{\myvec}[1]{\ensuremath{\begin{pmatrix}#1\end{pmatrix}}}
\let\vec\mathbf

\lstset{
%language=C,
frame=single, 
breaklines=true,
columns=fullflexible
}

\numberwithin{equation}{section}

\lstset{
  language=Python,
  basicstyle=\ttfamily\small,
  keywordstyle=\color{blue},
  stringstyle=\color{orange},
  numbers=left,
  numberstyle=\tiny\color{gray},
  breaklines=true,
  showstringspaces=false
}

\title{Problem 12.661}
\author{ee25btech11023-Venkata Sai}

\date{\today} 
\begin{document}

\begin{frame}
\titlepage
\end{frame}

\section*{Outline}
\begin{frame}
\tableofcontents
\end{frame}

\section{Problem}

\begin{frame}
\frametitle{Problem}
Let $\vec{A}=\myvec{1&2\\2&1}, \vec{X}=\myvec{1&a\\b&0}$ and $\vec{Y}=\myvec{3&1\\3&2}$.If $\vec{AX=Y}$.Then $a+b$ equals
\end{frame}
%\subsection{Literature}
\section{Solution}

 
\subsection{Inverse}
\begin{frame}
\frametitle{Inverse}
Given
  \begin{align}
\vec{A}=\myvec{1&2\\2&1}, \vec{X}=\myvec{1&a\\b&0}\ \text{and}\ \vec{Y}=\myvec{3&1\\3&2} 
\end{align}
\begin{align}
\vec{A}\vec{X}=\vec{Y} \\
\vec{X}=\vec{A}^{-1}\vec{Y}
  \end{align}
  Augmented matrix of $\augvec{1}{1}{\vec{A} & \vec{I}}$ is Inverse by
  \begin{align}
      \augvec{2}{2}{1& 2 & 1 & 0 \\ 2 & 1 & 0 & 1}& \xrightarrow{R_2\rightarrow R_2-2R_1} \augvec{2}{2}{1& 2 & 1 & 0 \\ 0 & -3 & -2 & 1} \\
      \augvec{2}{2}{1& 2 & 1 & 0 \\ 0 & -3 & -2 & 1}&\xrightarrow{R_2 \rightarrow \frac{2}{3}R_2} \augvec{2}{2}{1& 2 & 1 & 0 \\ 0 & \frac{2}{3}\brak{-3} & \frac{2}{3}\brak{-2} & \frac{2}{3}\brak{1}}
      \end{align}
      \begin{align}
      \augvec{2}{2}{1& 2 & 1 & 0 \\ 0 &-2 & -\frac{4}{3} & \frac{2}{3}} \xrightarrow{R_1 \rightarrow R_1+R_2} \augvec{2}{2}{1& 0 & 1-\frac{4}{3} & \frac{2}{3} \\ 0 &-2 & -\frac{4}{3} & \frac{2}{3}} 
  \end{align}
\end{frame}
\subsection{Conclusion}
\begin{frame}
\frametitle{Conclusion}
\begin{align}
    \augvec{2}{2}{1& 0 & 1-\frac{4}{3} & \frac{2}{3} \\ 0 &-2 & -\frac{4}{3} & \frac{2}{3}} \xrightarrow{R_2\rightarrow -\frac{1}{2}R_2} \augvec{2}{2}{1& 0 & -\frac{1}{3} & \frac{2}{3} \\ 0 &1 & \frac{2}{3} & -\frac{1}{3}}
\end{align}
 \begin{align}
      \vec{A}^{-1}=\myvec{-\frac{1}{3} & \frac{2}{3} \\\frac{2}{3} & -\frac{1}{3}}
  \end{align}
  \begin{align}
      \vec{X}=\myvec{-\frac{1}{3} & \frac{2}{3} \\\frac{2}{3} & -\frac{1}{3}}\myvec{3&1\\3&2} =\myvec{-\frac{1}{3}\brak{3}+\frac{2}{3}\brak{3}&-\frac{1}{3}\brak{1}+\frac{2}{3}\brak{2} \\
      \frac{2}{3}\brak{3}-\frac{1}{3}\brak{3}&\frac{2}{3}\brak{1}-\frac{1}{3}\brak{2}  } 
      \end{align}
      \begin{align}
      \myvec{1&a\\b&0}&=\myvec{1& 1 \\1&0} 
  \end{align}
  Hence $a=1,b=1\implies a+b=1+1=2$ 
 \end{frame}
 \begin{frame}[fragile]
 \section{C code}
\frametitle{C code}
\begin{lstlisting}[language=C]
void get_matrices_data(double* out_data) {
    out_data[0] = 1.0;
    out_data[1] = 2.0;
    out_data[2] = 2.0;
    out_data[3] = 1.0;
    out_data[4] = 3.0;
    out_data[5] = 1.0;
    out_data[6] = 3.0;
    out_data[7] = 2.0;
}
\end{lstlisting}
\end{frame}
\section{Python code}
\begin{frame}[fragile]
\frametitle{Python Code for Solving}
\begin{lstlisting}[language=Python]
import ctypes
import numpy as np

def solve_matrix_equation():

    lib = ctypes.CDLL('./code.so')
    double_array_8 = ctypes.c_double * 8
    lib.get_matrices_data.argtypes = [ctypes.POINTER(ctypes.c_double)]

    out_data_c = double_array_8()
    lib.get_matrices_data(out_data_c)
    raw_data = np.array(list(out_data_c))

    A = raw_data[0:4].reshape((2, 2))
    Y = raw_data[4:8].reshape((2, 2))


\end{lstlisting}
\end{frame}
 \begin{frame}[fragile]
\frametitle{Python Code for Solving}
\begin{lstlisting}[language=Python]

   X = np.linalg.solve(A, Y)

    return X
if __name__ == '__main__':
    solution_X = solve_matrix_equation()

    a = solution_X[0, 1]
    b = solution_X[1,0]

    print(f"\nThis corresponds to: a={a:.2f}, b={b:.2f}")


\end{lstlisting}
\end{frame}
\end{document}
